\chapter{Conclusions}
This is the second chapter of the present dissertation. It is more interesting than the first one, for it is the last one.

\section{Limitations}

\section{Future Work}

\subsection{Adult Subject Population and Images}

As the prognosis for patients with CHD improves, their life expectancy also increases. The aging CHD population presents new questions about the connection between CHD and neurocognitive challenges associated with aging. As patients age, there is an expectation that their images will contain less motion for a time. If a patient begins to show signs of cognitive impairment due to aging, it can be expected that their images will begin to contain more motion as their neurocognitive state deteriorates. 

We include a cohort of adult subjects over a wide range of ages in our study. The purpose of using images from this cohort is to demonstrate the generalizability of the DAG-based framework to adult patients as well as its use in different clinical populations. This cohort is being studied as part of an ongoing, prospective study of CHD and neurodevelopment. The data collected for these subject includes rs-fMRIs, behavioral, and clinical data from healthy and CHD adult subjects. 
