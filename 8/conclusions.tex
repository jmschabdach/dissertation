\chapter{Conclusions}
\label{ch:fin}

The two primary goals of this work were to compare two volume registration techniques and to characterize motion in different clinical groups. Five sets of images were used for these experiments: a simulated data set, brain images from three clinical cohorts, and a set of placenta images. 

An rs-fMRI simulation tool was developed to generate data with a known BOLD signal, background signal noise, and patient motion. The base for the structural information in the simulated sequences was the proton density image from the MNI average brain data, while the regions of interest used to generate the simulated BOLD signal were taken from the 90 functional ROI atlases created by Shirer et al. \cite{Shirer2012}. 

The clinical images were obtained as part of prospective, long-term studies of CHD and brain development. The neonatal subjects were recruited from a single site while the fetal subjects were recruited at one of two sites, and the preadolescent subjects were recruited at one of 12 sites in the United States. The sequences differed in length depending on the age group and scanning site. In addition to the images, basic demographic information was gathered for each subject.

All rs-fMRIs from the simulated and clinical data sets underwent both the traditional registration and the DAG-based registration. The original sequences and both types of registered sequences were compared. Four metrics were used for these comparisons. The FD and DVARS metrics were calculated between every image volume $i$ and $i+1$ in each sequence. The Dice and MI similarities were calculated for every possible pair of image volumes $i$ and $j$ in a single sequence. 

The FD and DVARS metrics were compared to the usability thresholds set forth by Power et al. \cite{Power2012}. These thresholds state that an rs-fMRI has sufficiently low quantities of motion if more than 50\% of the volumes in the sequence have FD $<$ 0.2 mm and DVARS $<$ 2.5\% signal intensity units. The number of image volumes meeting these thresholds for each sequence type (original, traditionally registered, and DAG-registered) were calculated and compared using two-sample t-tests. 

Comparing the FD and DVARS metrics for the simulated, preadolescent, neonatal, and fetal brain images to the usability threshold showed that the two registration types had comparable effects with respect to recovering image volumes to the usability standards. Between these four data sets, there were slightly different impacts of each registration type on each cohort, but the registration techniques did not recover statistically significantly different numbers of image volumes. The placenta images, however, demonstrated a statistically significant increase in the number of image volumes meeting the DVARS threshold and the pair of thresholds after registration.

The rs-fMRIs were also compared using matrices of similarity metrics where the value at every row $i$ and column $j$ was the similarity between volumes $i$ and $j$ in the same image sequence. The purpose of these matrices was to better characterize the positional and signal changes due to motion throughout the image sequence. Analysis of the sequence duration motion in the simulated data set showed mixed results. For the simulated sequences, the DAG-based registration was better at improving the range and quartile values of the distribution of the original similarity metrics than the traditional registration was.

One additional analysis on the simulated images was used to compare the traditional and DAG-based registration techniques. The registered images underwent independent component analysis. The components for each image were compared to the DMN ROI to identify the component which correlated best with the simulated BOLD signal. The best matching components were compared voxelwise to the DMN ROI to determine the number of true positive, false positive, true negative, and false negative identifications of component voxels as belonging to the DMN ROI. The DAG-based registration had a higher true positive rate and a lower false positive rate (0.623 and 0.00996) than the traditional registration did (0.587 and 0.104).

After analyzing the volume registration techniques, the patterns of motion in the clinical brain images were characterized into two categories of interest. No detectable groups were identified when the metrics for the original image sequences were clustered and labeled according to disease status. However, analysis of the clusters labeled according to age group suggests that there are detectable differences in the patterns of motion between subjects from different age groups.
