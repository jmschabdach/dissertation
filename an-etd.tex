\documentclass[dvipdfm,final]{pittetd}%   If you want to use dvipdfm. The file is to be normally processed (LaTeX), and 
%                                   then the program dvipdfm applied to it. This will create the PDF file with bookmarks
%                                   and links. It will also try to convert any PS graphics included.
%\documentclass[pdftex]{pittetd}    If you want to use PDFLaTeX instead. This will create the PDF file directly.
%                                   Processing time can be longer. No PS graphics conversion will take place 
%                                   automatically.
%
%Other options: ma, ms, for Master's. 
%11pt, 10pt, font size (12pt is default). 
%final, makes pittetd's warnings (about things that might go against the Format Guidelines) 
%into error messages. 
%Option 'sectionletters' numbers the chapters with Roman numerals (I, II, etc.), sections with 
%letters (A, B), subsections with numbers (1, 2), and subsubsections with lowercase letters (a, b). 
%The four levels of the enumerate environment receive the same treatment. Within the
%text, however, cross references (\ref} produce `the whole thing,' something like I.A.1 
%instead of only 1.
%
\usewithpatch{graphicx}%            Better \usewithpatch than \usepackage because it makes pittetd look for any 
%                                   available patch for the package. 
\usepackage{amsmath,amsthm}%        But you can't use \usewithpatch for several packages as in this line. The search for
%                                   patches has to be then forced through:
\patch{amsmatch}
\patch{amsthm}
%
%\patch{pittdiss}                   If you started writing your thesis with the pittdiss class, this patch makes 
%                                   pittetd interpret pittdiss commands.
%\patch{pitthesis}                  Analogous for the pitthesis class.
\title[A pittetd-thesis sample]{Sample file for a thesis with the `pittetd' class}% The optional argument is the %                                   version of the title that will appear in Acrobat Reader's Document Info dialog box.
\author{I. M. Student}
\degree{Previous degree, institution, year}
%\date{July 20th 1967}%             This date is the date of the thesis defense. Default is \today
%\year{1967}                        pittetd will use the current year unless otherwise indicated. So this command is not
%                                   necessary.
\keywords{\LaTeX, pittetd, theses, format}% This list appears in the field 'Keywords' of Acrobat Reader's Document Info
%                                   dialog box, and also, optionally, after the abstract.
\subject{Sample file}%              This fills in the 'Subject' field in Acrobat Reader's Document Info dialog box.
\school{Department of Mathematics}%    The name of the school will be preceeded by 'the' unless otherwise specified, as in:
%\school[certain]{department}
%
%\chapterfloats%                    Un-comment this to get figures and tables numbered within chapters.
\begin{document}
\maketitle
%
% For the committee membership page, you have to provide the names and affiliations of the members. The first one will 
% be treated by pittetd as the committee chair (thesis/dissertation advisor).
\committeemember{Chair's name, Departmental Affiliation}
%\coadvisor{Second advisor, Dept. Aff.}%         This is used if there are two advisors.
\committeemember{Second member's name, Dept.\ Aff.}
\committeemember{Third member's name, Dept.\ Aff.}
% etc., as many as needed. For master's theses, the committee may be omitted, naming only the advisor.
\school{Mathematics Department}
\makecommittee
%\copyrightpage                     Uncomment this to get a copyright page.
\begin{abstract}
This document is a sample file for the creation of ETD's at Pitt through \LaTeX.
\end{abstract}
% If you say \begin{abstract}[Keywords:] instead of the simple \begin{abstract}, a list of the keywords is appended.
% The list comes from the \keywords command above.
% The starred version \begin{abstract*} typesets the word `ABSTRACT' on the top of the page
\tableofcontents
%\listoftables                      Pittetd will complain if you tell it to create a list of tables when there are no
%                                   tables (as in this sample file). Uncomment this command if you have tables.
%\listoffigures                     Obvious analogous for figures.
%\preface
% This is the text of the preface, with acknowledgments, dedication, etc. It is optional, and you create, as shown, by 
% just saying \preface and starting the preface's actual text. Note that 'foreword' is no longer acceptable as title
% for this preliminary.
%
%Conventions, such as notation (nomenclature) and abbreviations, don't receive their own preliminary page. They can be included as an appendix, or as part of the introduction.
%
\chapter{Introduction}%             And when you type '\chapter', pittetd understands that this is the body of the 
%                                   document. 
We begin by saying that we do not really have much to say, but for the sake of clarity we divide our topic in chapters.

\chapter{Second chapter}
The topics treated in this chapter can be somewhat obscure. For humanitarian considerations, the chapter will be subdivided.
\section{FIRST SECTION}%            Remember to capitalize the sections (otherwise, the bookmark will be lowercase)
The topic treated here, given its complexity, merits an additional subdivision.
\subsection{First subsection of the section}
This is well-known topic, and we shall discuss it no more.
\subsection{Second subsection of the section}
This is a very complicated topic and we shall discuss it in our next paper.
\chapter{Conclusions}
This is the second chapter of the present dissertation. It is more interesting than the first one, for it is the last one.
%
%\appendix                          After this command, chapters will be formatted as appendices. For example:
%\chapter{Raw data}
%
%\safebibliography{etdbib}          \safebibliography is used the same way as \bibliography, but gives pittetd
%                                   a greater chance to succeed in formatting the bibliography when non-standard
%                                   BibTeX styles are used.
\end{document}
