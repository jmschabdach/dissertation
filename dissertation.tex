\documentclass[pdftex,final]{pittetd}
%final, makes pittetd's warnings (about things that might go against the Format Guidelines) into error messages. 
%Option 'sectionletters' numbers the chapters with Roman numerals (I, II, etc.), sections with 
%letters (A, B), subsections with numbers (1, 2), and subsubsections with lowercase letters (a, b). 
%The four levels of the enumerate environment receive the same treatment. Within the
%text, however, cross references (\ref} produce `the whole thing,' something like I.A.1 
%instead of only 1.

% Packages included in PittETD Template
% auto use this package and check for patches
\usewithpatch{graphicx} 
% manually use these packages
\usepackage{amsmath}
% manually check for patches
\patch{amsmath}
%\patch{amsthm}

% Jenna's commonly used packages
\usepackage{amssymb}
\usepackage{amsmath}
%\usepackage{amsthm}
\usepackage{graphics} % for improved inclusion of graphics
%\usepackage{wrapfig} % to include figure with text wrapping around it
%\usepackage{subcaption}
%\usepackage[margin=10pt,font=small,labelfont=bf]{caption}
\usepackage{algorithm}
%\usepackage{algorithmic}
\usepackage{algpseudocode}

% Bibliography
\bibliographystyle{apalike}

\title[Global Volume Registration for Motion Correction in rs-fMRIs]{Global Volume Registration for Multiple Motion Types in Resting-State Functional Magnetic Resonance Images}
% The optional argument is the 
% version of the title that will appear in Acrobat Reader's Document Info dialog box.
\author{Jenna Marie Schabdach}
\degree{B. S. Electrical Engineering, Drexel University, 2016\\M. S. Electrical Engineering, Drexel University, 2016\\M. S. Biomedical Informatics, University of Pittsburgh, 2018}

%\date{July 20th 1967}%             This date is the date of the thesis defense. Default is \today
% pittetd will use the current year unless otherwise indicated. So this command is not necessary.
\keywords{\LaTeX, pittetd, theses, format}% This list appears in the field 'Keywords' of Acrobat Reader's Document Info
%                                   dialog box, and also, optionally, after the abstract.
\subject{Schabdach Biomedical Informatics Dissertation}%              This fills in the 'Subject' field in Acrobat Reader's Document Info dialog box.
\school{Department of Biomedical Informatics}%    The name of the school will be preceeded by 'the' unless otherwise specified, as in:
%\school[certain]{department}

\begin{document}

\year{2019}  
%\chapterfloats%                    Un-comment this to get figures and tables numbered within chapters.
\maketitle
%
% For the committee membership page, you have to provide the names and affiliations of the members. The first one will 
% be treated by pittetd as the committee chair (thesis/dissertation advisor).
\committeemember{Dr. Ashok Panigrahy, Department of Biomedical Informatics}
%\coadvisor{Second advisor, Dept. Aff.}%         This is used if there are two advisors.
\committeemember{Second member's name, Dept.\ Aff.}
\committeemember{Third member's name, Dept.\ Aff.}
% etc., as many as needed. For master's theses, the committee may be omitted, naming only the advisor.
\school{School of Medicine}
\makecommittee
%\copyrightpage                     Uncomment this to get a copyright page.
\begin{abstract}
INSERT ABSTRACT HERE
\end{abstract}
% If you say \begin{abstract}[Keywords:] instead of the simple \begin{abstract}, a list of the keywords is appended.
% The list comes from the \keywords command above.
% The starred version \begin{abstract*} typesets the word `ABSTRACT' on the top of the page
\tableofcontents
%\listoftables                      Pittetd will complain if you tell it to create a list of tables when there are no
%                                   tables (as in this sample file). Uncomment this command if you have tables.
%\listoffigures                     Obvious analogous for figures.
%\preface
% This is the text of the preface, with acknowledgments, dedication, etc. It is optional, and you create, as shown, by 
% just saying \preface and starting the preface's actual text. Note that 'foreword' is no longer acceptable as title
% for this preliminary.
%
%Conventions, such as notation (nomenclature) and abbreviations, don't receive their own preliminary page. They can be included as an appendix, or as part of the introduction.


% Add in chapters here
\chapter{Introduction}

Resting-state functional magnetic resonance imaging (rs-fMRI) measures the blood oxygen level dependent signal in an organ or organ system. This property makes rs-fMRI an invaluable tool for evaluating a patient's neurodevelopmental status or examining functional networks in his brain. To gather enough data to fully evaluate these networks, a series of image volumes must be acquired over a period of several minutes. In a standard rs-fMRI (?), one new image volume is obtained approximately once every two to three seconds. To gather high quality data on such a short timescale, the rs-fMRI suffers from two major limitations: rs-fMR images have low physical resolution and are highly susceptible to motion. The first limitation can be addressed by obtaining an MR image with high physical resolution and registering the rs-fMRI to this structural image, but the second limitation requires the patient to remain as still as possible for the entire duration of the scan. This task is particularly difficult for populations of certain ages or populations who suffer from conditions that affect neurodevelopment. As a result, it is common for an image from a member of one of these populations to contain too much motion to be used in clinical or research applications.

Various behavioral and XX protocols have been developed in an attempt to prevent patients from moving during MRI scans, though many of these protocols are not applicable to younger populations. In particular, a neonate or fetus cannot understand instructions to stay still, and young children who can understand the command have difficulty following it. Sedation is not advisable for these young populations. After a rs-fMR image is acquired, however, it is possible to reduce the positional effects of motion in the image sequence.

Limitations of traditional methods

DAG-based registration

Apply DAG-based registration to neonates and preadolesents

Real goal is to develop a method of registering fetal brain and placental images so that we can further examine the relationship between placental oxygen levels and fetal brain development. Longitudinally, this technique can be used to determine how placental oxygen flow and fetal brain development impact a patient over the course of his or her life. Once the relationship between the placenta and fetal brain development is better understood, we can determine a set of neuroprotective interventions to employ for at-risk patients before they are born.
\chapter{Background}
The topics treated in this chapter can be somewhat obscure. For humanitarian considerations, the chapter will be subdivided.

\section{RESTING-STATE NETWORKS}

The idea of a neuronal network which operated when a person is at rest was proposed in 2001, and then confirmed in 2003 \cite{Raichle2001} \cite{Greicius2003}. Resting-state networks are recorded using resting-state functional magnetic resonance images (rs-fMRIs). rs-fMRIs are sequences of image volumes acquired over a period of a few minutes while the patient is in a task-free state. The image volumes themselves have relatively low spatial resolution when compared to structural MRIs, but their temporal resolution is significantly higher as a new volume is acquired every two to three seconds. Each volume records the blood oxygen level dependent (BOLD) signals within the brain at that point in time. 

The BOLD signals in rs-fMRI image sequences are analyzed using a process called functional connectivity analysis. Functional connectivity analysis identifies patterns and networks of brain activity. Because the patient is not performing a specific task during a rs-fMRI acquisition, these resting-state networks have the potential to reveal valuable information about a patient's neurodevelopmental status. Some functional connectivity analysis studies have lead to the discoveries of links between specific disruptions in these naturally occurring networks and neurodevelopmental diseases such as autism and attention deficit hyperactivity disorder \cite{Assaf2010} \cite{Zang2007}. With further refinements of both acquisition techniques and characterization of these functional networks, clinicians may be able to use rs-fMRI in early detection protocols to evaluate the neurodevelopmental status of infants and neonates, and in personalized care by identifying patients who may benefit from certain therapies or neuroprotective interventions.

\section{MEASURING THE EFFECTS OF MOTION}

Due to their low spatial and high temporal resolutions, rs-fMRIs are highly susceptible to motion. Even the smallest movement can alter the position of the patient enough to cause the voxels to record signals from different brain regions and tissue types. Even if the movement does not significantly change the recorded position of the subject, it impacts the established spin gradients, which introduces artifacts into the image sequence. Movements cause the orientation of existing spin gradients to change, and the gradients require time to realign to the magnetic field. This recovery time often results in a decrease in the global signal in frames obtained over the following 8-10 seconds, which can affect the functional connectivity analysis \cite{Power2014}.

The effects of motion on rs-fMRIs can be clearly divided into two categories: the effect on patient position and the effect on the recorded BOLD signal.

The effect of motion on patient position is measured in terms of the difference in position between temporally neighboring image volumes. The difference in position is determined using metrics calculated by performing rigid volume registration on the two volumes. 
In rigid volume registration, one volume is chosen as the reference volume and the other is considered the moving volume. The reference volume remains stationary while the moving volume is translated and rotated in three-dimensional space on top of it. The registration is considered successfully complete when the position of the patient in the moving volume matches the position in the reference volume. The three translation and three rotation parameters used to achieve this alignment are used to calculate the positional change between the image volumes, which is often called the framewise displacement (FD). 

Several researchers have proposed different methods for calculating the FD. Power et al., Jenkinson et al., and Dosenbach et al. each propose a slightly different method for calculating the FD \cite{Power2012} \cite{Jenkinson2002} \cite{Dosenbach2017}. All three FD calculations produce correlated metrics: the FD metric proposed by Power et al. produces measurements approximately twice as large as the metric proposed by Jenkinson et al., and Dosenbach et al. reported a high correlation between their FD and Power’s FD \cite{Yan2013a} \cite{Dosenbach2017}.

Signal change 
Even the smallest movement can impact the recorded BOLD signal for the following eight to ten seconds [CITE].


An rs-fMRI must (contain only low amounts of motion) for functional connectivity analysis to be successful. There is some debate a

\section{MOTION PREVENTION}

Sedation can be used for some medical image acquisitions, but it is not appropriate for all patient populations or imaging modalities. 

\section{MOTION CORRECTION}

\section{VOLUME REGISTRATION}

Liao et al. suggested that a rs-fMRI sequence could be viewed as a hidden Markov model, and reflected this idea in their suggested registration framework \cite{Liao2016}. Their framework uses the transformation of the previous volume to the reference volume as the initial transformation for the current volume and the reference volume. 



%\section{MAGNETIC RESONANCE IMAGING}
%
%Magnetic resonance imaging is a technique used in medicine to obtain in vivo images of different organ systems. The patient is placed in a static magnetic field and electromagnetic pulses are used to briefly induce a secondary magnetic field.
%
%One type of MRI is resting state functional MRI (rs-fMRI). rs-fMRI is 
%
%
%\section{THE CHALLENGE OF MOTION}%            Remember to capitalize the sections (otherwise, the bookmark will be lowercase)
%
%Patient movement is a challenge across all modalities of medical imaging, but rs-fMRI is particularly sensitive to it. Various preventative, 
%
%An rs-fMRI is considered usable if it contains at least 5 minutes of low motion data.
%
%\subsection{•}
%
%
%
%\subsection{First subsection of the section}
%This is well-known topic, and we shall discuss it no more.
%\subsection{Second subsection of the section}
%This is a very complicated topic and we shall discuss it in our next paper.\textsl{•}


\chapter{Conclusions}
This is the second chapter of the present dissertation. It is more interesting than the first one, for it is the last one.


%
%\appendix                          After this command, chapters will be formatted as appendices. For example:
%\chapter{Raw data}
%
\safebibliography{sources}          
%\safebibliography is used the same way as \bibliography, but gives pittetd
%                                   a greater chance to succeed in formatting the bibliography when non-standard
%                                   BibTeX styles are used.
\end{document}
