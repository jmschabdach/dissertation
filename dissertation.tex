\documentclass[hidelinks,pdflatex,final]{pittetd}

% Packages included in PittETD Template
% auto use this package and check for patches
\usewithpatch{graphicx} 
% manually use these packages
\usepackage{amsmath,amsthm}%        But you can't use \usewithpatch for several packages as in this line. The search for
%                                   patches has to be then forced through:
\patch{amsmatch}
\patch{amsthm}

% Jenna's commonly used packages
%\usepackage{amsthm}
\usepackage{graphics} % for improved inclusion of graphics
%\usepackage{wrapfig} % to include figure with text wrapping around it
\usepackage{subcaption} % fit multiple graphics in one figure
%\usepackage[margin=10pt,font=small,labelfont=bf]{caption}
\usepackage{algorithm}
%\usepackage{algorithmic}
\usepackage{algpseudocode}
\usepackage{hyperref}
\usepackage{multirow}
\usepackage{listings}
\usepackage{apalike}
\usepackage{soul}
\usepackage{color}

\lstset{
  language=Python,
  basicstyle=\ttfamily,
  numbers=left,
  numberstyle=\tiny,
  numbersep=5pt,
  showstringspaces=false,
}

\def\sectionautorefname{Section}
\def\chapterautorefname{Chapter}

% Bibliography
\bibliographystyle{apalike}

\title[Correction, Validation, and Characterization of Motion in Resting-State Functional Magnetic Resonance Images of Pediatric Patients]{Correction, Validation, and Characterization of Motion in Resting-State Functional Magnetic Resonance Images of Pediatric Patients}
% The optional argument is the 
% version of the title that will appear in Acrobat Reader's Document Info dialog box.
\author{Jenna Marie Schabdach}
\degree{B. S., Drexel University, 2016\\M. S., Drexel University, 2016\\M. S., University of Pittsburgh, 2018}
\date{March 31, 2020}%             This date is the date of the thesis defense. Default is \today
\year{2020}
% pittetd will use the current year unless otherwise indicated. So this command is not necessary.
\keywords{resting-state fMRI, medical imaging, motion correction}% This list appears in the field 'Keywords' of Acrobat Reader's Document Info
%                                   dialog box, and also, optionally, after the abstract.
\subject{J Schabdach Biomedical Informatics Dissertation}%              This fills in the 'Subject' field in Acrobat Reader's Document Info dialog box.
\school{School of Medicine}

\begin{document}
  
%\chapterfloats%                    Un-comment this to get figures and tables numbered within chapters.
\maketitle
%
% For the committee membership page, you have to provide the names and affiliations of the members. The first one will 
% be treated by pittetd as the committee chair (thesis/dissertation advisor).
\committeemember{Dr. Douglas Landsittal, Department of Biomedical Informatics}
%\coadvisor{Second advisor, Dept. Aff.}%         This is used if there are two advisors.
\committeemember{Dr. Ashok Panigrahy, Department of Biomedical Informatics}
\committeemember{Dr. Gregory Cooper, Department of Biomedical Informatics}
\committeemember{Dr. Rafael Ceschin, Department of Pediatric Radiology,\\ Children's Hospital of Pittsburgh of UPMC}

% etc., as many as needed. For master's theses, the committee may be omitted, naming only the advisor.
\school{School of Medicine}
\makecommittee

\copyrightpage                     

\begin{abstract}
Understanding a patient’s neurodevelopmental status is valuable for many research and clinical applications. Neurodevelopmental evaluations can be performed through psychological testing or individual assessments with a psychologist. However, these approaches are not applicable in all cases. Resting-state functional magnetic resonance images (rs-fMRIs) can be used to study neuronal networks that are active when the patient is in a task-free state. These image sequences are highly sensitive to motion. Techniques have been developed to prevent patients from moving, monitor motion during the scan, and correct motion after the scan. We focus on the first step of retrospective motion correction: volume registration.

The purpose of volume registration is to align the contents of all of the image volumes in the image sequence to those of a single volume. Traditionally, all image volumes are directly registered to the chosen stationary image volume, but this approach does not account for significant differences in patient position across the sequence. We developed a registration framework based on directed acyclic graphs (DAG). We treat the volumes in the sequence as nodes in a graph where pairs of subsequent volumes are connected via directed edges. This perspective better models the relationships between volumes and accounts for them during registration.

We applied both registration frameworks to a set of simulated images as well as neurological rs-fMRIs from three clinical populations of patients who are healthy or have congenital heart disease (CHD). The original and registered sequences were compared with respect to their local and global motion. These motion parameters were used to determine how many sequences had statistically significant differences in their motion distributions before and after registration. The metrics of the original sequences were clustered to identify age group-specific motion patterns.

The registration frameworks had different effects on each age group. We found that the neonatal subjects contained the least amount of motion, while the fetal subjects contained the most motion. The DAG-based registration was most effective at reducing motion in the fetal images. Our clustering analysis showed that the different age groups have different global motion parameters, though patterns related to CHD status could not be detected.


\end{abstract}

% If you say \begin{abstract}[Keywords:] instead of the simple \begin{abstract}, a list of the keywords is appended.
% The list comes from the \keywords command above.
% The starred version \begin{abstract*} typesets the word `ABSTRACT' on the top of the page
\tableofcontents
\listoftables                     
\listoffigures                     

\preface

\section{Dedication}
To the patients being treated in children's hospitals all over the world, their families, and the people working to help them.         

\clearpage

\section{Acknowledgements}

This project would not have been possible without the assistance and support of many people. First, to my committee members: Dr. Doug Landsittal, Dr. Ashok Panigrahy, Dr. Greg Cooper, and Dr. Rafael Ceschin. Your guidance and suggestions were truly invaluable. Second, to the members of PIRC, especially: Julia Wallace, Vince Lee, and Nancy Beluk, for helping wrangle the data and answering questions about behavioral techniques used during pediatric scans; Dr. Vincent Schmithorst, for the many conversations about the intricacies of MR physics; Billy Reynolds, for the fetal data and for pulling the occasional perfect figure seemingly out of thin air; and Samuel Cho, for helping me develop the data simulation and aptly naming it ``SPECTr''. Third, this document would not be the same without the Writing Accountability Groups organized and run by Dr. Moriah Kirdy through the Pitt Writing Center. Moriah's dedication and organization is (thankfully) contagious. 

A successful dissertation is only partly made up of the research and writing. The rest is managing people, places, and paperwork. Many thanks to Toni Porterfield and Tami Robinson for helping me keep my administrative ducks in a row and to Barbara Karnbauer for helping coordinate schedules. 

Of course, I would like to thank my friends and family. Ryan, for keeping me grounded, and for rock climbing, burritos, and ice cream. My friends in and around Philadelphia and Pittsburgh, for the laughs, game sessions, and more rock climbing. Mom and Dad, thanks for supporting me the way you always do. Jonathan and David, I'm over the moon you were able to make my defense. 

Finally, special thanks to everyone involved with Extra Life, a charity that raises money for kids being helped by Children's Miracle Network Hospitals. As promised, here is the shout out to those who donated to my page, especially to Leo Chan, Elliot Guiso, Bonnie Young, John Maloney, Justine White, and Mike West.

% This is the text of the preface, with acknowledgments, dedication, etc. It is optional, and you create, as shown, by 
% just saying \preface and starting the preface's actual text. Note that 'foreword' is no longer acceptable as title
% for this preliminary.
%
%Conventions, such as notation (nomenclature) and abbreviations, don't receive their own preliminary page. They can be included as an appendix, or as part of the introduction.


% Add in chapters here
\chapter{Introduction}
\label{ch:intro}

Patient motion is a critical cause of data loss in medical imaging. Many approaches have been developed to prevent and reduce the impact of motion before, during, and after image acquisition. The effectiveness of these techniques vary between patient populations. If the effects of a patient's movements cannot be removed from an image, that image is considered to have been corrupted by motion and is deemed unusable.

Though patient motion is a problem across the entire medical imaging domain, we focus specifically on resting state functional magnetic resonance imaging (rs-fMRI). rs-fMRIs measure the blood oxygen level dependent signal in an organ or organ system. Areas of an organ with greater activity require more oxygen than less active areas. When used to examine the brain, the signals recorded by the rs-fMRI are used as an effective approximation of the amount of activity occurring in different areas of the brain. The term ``resting state'' means that the patient is not performing any particular task, so any activity that occurs is from underlying networks connecting different areas of the brain. 

\section{Resting-state Functional Magnetic Resonance Images}

When an area of the brain is active, it uses more oxygen than the surrounding regions. Functional MRIs (fMRI) are sensitive to signals emitted by deoxygenated hemoglobin. The blood oxygen level dependent (BOLD) signal recorded by the fMRI reveal regions of the brain which are active at the same time. These combinations of regions are called neuronal networks. 

Many neuronal networks exist, but most of them are considered to be task related. In 2001, Raichle et al. suggested the existence of a neuronal network which operated when a person is at rest \cite{Raichle2001}. Their theory was confirmed by Greicius et al. in 2003 \cite{Greicius2003}. Because the patient is not performing a specific task when they are in a resting state, the resting-state networks have the potential to reveal valuable information about a patient's neurodevelopmental status.

To gather enough data to fully evaluate these networks, a series of image volumes must be acquired over a period of several minutes. A fMRI taken of a patient in a resting, task-free state, is called a resting-state fMRI (rs-fMRI). rs-fMRIs are sequences of image volumes acquired over a period of a few minutes. The image volumes themselves have relatively low spatial resolution when compared to structural MRIs, but their temporal resolution is significantly higher as a new volume is acquired every two to three seconds. 

The BOLD signals in rs-fMRI image sequences are analyzed using a process called functional connectivity analysis. Functional connectivity analysis identifies patterns and networks of brain activity. Some functional connectivity analysis studies have lead to the discoveries of links between specific disruptions in these naturally occurring networks and neurodevelopmental diseases such as autism and attention deficit hyperactivity disorder \cite{Assaf2010} \cite{Zang2007}. With further refinements of both acquisition techniques and characterization of these functional networks, clinicians may be able to use rs-fMRI to evaluate the neurodevelopmental status of CHD patients and to identify patients who may benefit from certain therapies or neuroprotective interventions.

To gather high quality data on such a short timescale, the rs-fMRI suffers from two major limitations: rs-fMR images have low physical resolution and are highly susceptible to motion. The first limitation can be addressed by obtaining an MR image with high physical resolution and registering the rs-fMRI to this structural image, but the second limitation is a significant problem. 

\section{Motion Effects, Prevention, and Correction}

There are three effects of motion on an rs-fMRI scan: the positional effect, the spin history effect, and the susceptibility effect. The impact of motion on the position of the patient means that a given voxel will not record electromagnetic signal from the same location in the brain for the duration of the scan. When the patient moves, the molecules which where in the area activated by the MRI scanner also move. Molecules which were not activated are now in the area where signal is being recorded, which results in a decrease in signal. At the next activation, an molecules which were moved out of the previous area of activation may be further activated, which can result in a increase in signal. These spin history effects impact the image sequence for several frames, but dissipate. The suspectibility effects impact the signal recorded by the scanner for the duration of the scan. The difference in the susceptibility of different tissues is most prominent at the tissue interfaces. When the patient moves, the tissue interfaces move and contribute or detract from the signal at new locations. 

In general, the best way to prevent the effects of motion is to prevent motion itself. Various clinical, behavioral, and technical protocols have been developed in an attempt to prevent patient motion from impacting the rs-fMR image as it is acquired. Sedation can be used to immobilize a patient during a scan, but requires additional personal to perform safely and involves a greater time commitment from the patient. Sedation is also not recommended for use in young children and fetal patients. Behavioral and educational techniques can be employed to prepare a patient for stressors he may experience during a medical imaging scan, but these approaches do not prevent the patient from moving out of boredom, discomfort, or distress. Several groups have developed techniques to compensate for motion as an image is acquired, but these techniques often require additional scanner-compatible equipment and can only be utilized during the scan. Sedation or intra-scan motion monitoring approaches are difficult to integrate with MR scanners due to the constraints of MR safety requirements. 

Additional processing is needed to remove motion from an image after the scan is acquired. Many methods have been developed to mitigate the effects of motion after the rs-fMRI is acquired. While different post-acquisition motion correction pipelines utilize different processing techniques, they begin with global volume registration. Global volume registration is the process used to align all volumes in a rs-fMRI sequence into the same physical space. Traditionally, all volumes in the sequence are registered directly to one volume. This approach can be effective in images where the subject remains relatively still throughout the duration of the scan, but is not as successful in images containing high quantities of patient movement.

We have developed an alternative volume registration framework which takes into account the spatiotemporal relationships between sequential volumes in the rs-fMRI sequence and uses these relationships during the registration process. Herein, we evaluate it further in the context of a complete motion correction pipeline across healthy and disease populations at various stages of life. In addition to reducing the effects of patient motion on image quality, we are also interested in the patient motion itself. We believe there are relationships between different motion patterns, patient age, and clinical outcomes, and we have explored these relationships throughout our experiments.

\section{Data}

The disease population we used for our study is a population with a variety of congenital heart defects. Congenital heart defects (CHDs) have many presentations, and all cause problems in a patient's heart structure and the structure of the surrounding vessels. It has been found that the development of cardiac problems \textit{in utero} is often linked to delays in patient neurodevelopment. Research in the area of CHD and neurodevelopment has often focused on younger populations. However, treatment of CHDs has evolved over the past fifty years with the result that many CHD patients live to adulthood: every year, approximately 1.35 million children are born with a congenital heart defect and it is currently estimated that about 12 to 34 million adults are living with CHD \cite{VanderLinde2011}. Researchers have recognized the burden of neurocognitive disorders on the aging CHD population and are now starting to investigate the relationships between CHD and neurocognitive outcomes.

The process for objectively identifying neurocognitive disorders is still under development. Psychologists have developed and validated surveys to estimate a patient's neurocognitive status. These surveys vary with the child's. Initially, a parent fills out the survey on behalf of his infant or toddler child. When the child has reached certain developmental milestones, the parent and child might both fill out different portions of a different survey. At some point, the child can fill out his own survey. Psychologists may meet with the patient and his parents to determine a diagnosis. These survey based methods are highly subjective, and objective methods based on rs-fMRIs are being explored. 

Eventually, clinicians will be able to develop a lifespan approach to managing CHD and neurocognitive disorders. As a community, we are still in the data-gathering stage of this research. We cannot afford to lose rs-fMRI scans of healthy or CHD patients in any stage of life because of motion. For these reasons, a cohort of healthy and CHD patient images are an ideal data set for our motion correction work. 

We were able to access data from three clinical cohorts. The first cohort is a set of healthy and CHD neonatal subjects scanned at our primary study site. The second cohort is a set of healthy and CHD preadolescent subjects enrolled in a multicenter study. Some subjects were scanned at our primary study site while others were scanned at one of the 11 other participating sites. The third cohort is a set of healthy and CHD fetal subjects scanned at our primary site.

Many rs-fMRI studies struggle to obtain enough low-motion scans to come to statistically significant conclusions. We develop a tool to address this challenge by simulating rs-fMRIs using existing average brain and functional region of interest templates. The simulated sequences contain brain signal, scanner noise, and motion. We then use this tool to simulate additional images with which to test the registration frameworks and to determine the effects of volume registration on brain signal.


%In addition to evaluating a global volume registration framework in the context of a fully motion correction pipeline, we also investigate the relationships between a patient's motion and their clinical outcomes, specifically to further the study of CHD and neurological disorders across the lifetime of the patient. %This analysis is valuable for each of our subject populations for different reasons. For the preadolescent patients, we are investigating the relationship between in-scan motion, neurocognitive development, and congenital heart disease status. In the fetal images, we investigate the relationship between FETAL BRAIN DEVELOPMENT AND PLACENTAL GROWTH. % need to run this part by Ashok.
%These investigations use both supervised and unsupervised machine learning techniques to determine what relationships a computer can detect between these pieces of information.

%For our study, we will use our large set of neurological rs-fMRIs of both healthy  and CHD neonatal, preadolescent, and adult subjects. We also have a set of neurological and placental rs-fMRIs for healthy and CHD fetal patients. We will apply both the tradition and novel registration frameworks to all images in our different cohorts and evaluate the impact of each framework on each image after passing it through a complete motion correction pipeline. The original, registered, and motion corrected images will be used to address the aims discussed in this chapter.

%Our aims for this project are as follows:
%\begin{itemize}
%\item \textbf{Aim 1.} Evaluate the impact of global volume registration on simulated and clinical data within a complete motion correction pipeline.
%\item \textbf{Aim 2.} Study the motion patterns in the different populations to formally describe demographic related motion patterns.
%\item \textbf{Aim 3.} Employ machine learning techniques to (a) measure the impact of motion on image harmonization in multi-center studies, and (b) evaluate the relationship between motion and cognitive, clinical, and behavioral outcomes of CHD patients.
%\end{itemize}

\section{Experiments}

Both the traditional and novel registration techniques are applied to the clinical and simulated images. We compare the original and registered images with respect to the amount of motion removed from each sequence. We use two metrics to measure the effects of motion and three metrics to measure the similarity of the images. 

The metrics used to measure motion are the change in patient position (framewise displacement, FD) and the change in overall signal (derivative fo the variance of the root mean square of the signal, DVARS) between one time point and the next. These metrics are only applied between every pair of chronological image volumes in a sequence. They are then compared to two thresholds commonly recognized in the field of motion in rs-fMRIs to determine the usability of an image sequence.

The three similarity metrics are the correlation ratio, the mutual information, and the Dice coefficient. The similarity metrics are used to compare every volume in a sequence to every other volume in the same sequence, which produces a two-dimensional matrix for each metric. Changes in the patterns of the matrices for the same metrics for the same images are used to determine the overall impact of registration on the similarities between volumes in an image sequence.

While correcting motion within an rs-fMRI is important both for clinical use and research applications, we are also interested in the motion itself. Through discussion with radiologists and researchers who work with rs-fMRIs, we have developed the following hypothesis: \textbf{Neonatal patients on average exhibit less motion than preadolescent patients, who exhibit less motion than fetal patients.} We evaluate this hypothesis using unsupervised machine learning techniques to identify clusters of similar patients. The unsupervised machine learning techniques will also be applied to age group level cohorts to identify patients with common patterns of motion in each cohort.

\section{Summary}


The remainder of this document is laid out as follows. We elaborate on resting-state functional magnetic resonance images (rs-fMRIs) and their use for investigating functional brain networks in Chapter \ref{ch:mri}. In Chapter \ref{ch:moco}, we perform a breadth-wise review of the effects of motion, methods to prevent motion, and methods to mitigate the effects of motion.
Chapter \ref{ch:mopa} transitions into methods for analyzing MRIs, machine learning techniques, and our approach to statistical analysis. 

Chapter \ref{ch:data} discusses congenital heart disease, its relationship with neurological conditions, and methods for evaluating neurological conditions. It also contains information about the scans and demographics information for each clinical cohort as well as details about the simulation. 

Chapter \ref{ch:results} contains the results of our experiments and statistical analyses for the registration experiments and the machine learning experiments. Chapter \ref{ch:discussion} contains a discussion of these results, and Chapter \ref{ch:fin} contains a metadiscussion of this study as a whole.
\chapter{rs-fMRIs and Patient Motion}
\label{ch:mri}

This chapter discusses rs-fMRIs and how they are affected by patient motion. Specific topics include the structure of rs-fMRIs, sources of motion, current methods for preventing and managing motion in rs-fMRIs, quantifying motion, and usability criteria for using images corrupted by motion.

\section{Structure of an rs-fMRI}

A rs-fMRI scan produces a four dimensional image series. The first three dimensions are length, width, and depth and encompass the physical space occupied by the patient's head. The information in these three dimensions is interpreted as a three dimensional, volumetric image. The fourth dimension is time. The temporal dimension interacts with the spatial dimensions such that the contents of that image volume change with time. This concept of a 4D image can be illustrated in two different ways. The first is an ordered list of 3D image volumes. The second is a single 3D image volume where the value of each voxel is a temporal signal. 

rs-fMRIs are discrete representations of continuous data. A new image volume of the patient's brain is acquired every two to three seconds. The image volume is composed of a three dimensional version of a pixel called a voxel (volume element). Just as the ``distance'' between each image volume encompasses a certain amount of time, each voxel encompasses a small volume of physical space. The transformations between the continuous physical and temporal dimensions and the discrete physical and temporal dimensions are the spatial and temporal resolutions. 

An rs-fMRI is considered to have relatively low spatial resolution but high temporal resolution. The physical size of a single voxel seems small at about 4 mm$^3$, but this resolution is not granular enough to capture details about activity within small structures of the brain. The activity information recorded during a rs-fMRI must be combined with the detailed anatomic information from a structural MRI to know precisely which areas of the brain are active at each point in time. A structural MRI volume takes much longer to acquire than a rs-fMRI volume, which can be obtained every two to three seconds. Unfortunately, the patient's position and neural activity can change faster than the image volume can be acquired. As a result, a temporal resolution of two to three seconds is not fast enough to actively compensate for sources of noise which confound the BOLD signal. 

\section{*Sources of Motion in rs-fMRIs}

During every medical imaging scan, the patient will naturally perform small, automatic movements due to regular bodily functions. Minuscule movements caused by cardiac activity may disrupt scans with high spatial resolution or with high sensitivity to the movement of blood molecules. Larger movements caused by respiration result in motion artifacts in images of the thoracic and abdominal cavities. 

Other motions occur on a larger and more conscious scale. It is important to note that different populations may exhibit more of certain macro-motions than others. The patient may fidget or shift his gaze when he becomes bored in the scanner. If the patient falls asleep during a scan, there may be slight movement as the body relaxes and retenses if the patient wakes. Certain MRI protocols are known to produce loud sounds: during one of these protocols, the patient may become surprised and react by jumping. Additionally, clausterphobic patients or patients who feel secure around specific people that are not allowed in the scanner room may become agitated. 


\section{Effects of Motion}

Due to their low spatial and high temporal resolutions, rs-fMRIs are highly susceptible to all types of motion outlined in the previous section. The effects of motion on rs-fMRIs can be clearly divided into two categories: the effect on patient position and the effect on the recorded BOLD signal.

\subsection{Positional Effects of Motion}

The smallest motion can alter the position of the patient enough to cause the voxels to record signals from different brain regions and tissue types. The technique used for analyzing rs-fMRIs, called functional connectivity analysis, assumes that the contents of one voxel at two different time points both contain signal from a single point in the brain. This assumption is vital in the process of inferring networks of neuronal activity. 

The effect of motion on patient position is measured in terms of the difference in the positions of the contents of temporally neighboring image volumes. The difference in position is determined using metrics calculated by performing rigid volume registration on the two volumes. 
In rigid volume registration, one volume is chosen as the reference volume and the other is considered the moving volume. The reference volume remains stationary while the moving volume is translated and rotated in three-dimensional space on top of it. The registration is considered complete when the position of the patient in the moving volume matches the position in the reference volume. % Maybe move these sentences to methods?

The moving volume can undergo linear or nonlinear transformations. Linear transformations include translation, rotation, and affine transformations along all three spatial dimensions as well as a scaling transformation. These transformations move the image volume as a whole: all voxels in the moving image remain in the same location relative to their neighbors. On the other hand, nonlinear transformations have the ability to warp the contents of the moving volume so that it better matches the contents of the reference volume. Nonlinear transformations are more complex than linear transformation. They involve additional image processing steps such as smoothing and voxel interpolation.

Even in cases when nonlinear transformations are used, the registration process begins with the translation and rotation transformations. The three translation and three rotation parameters used to achieve the best alignment are used to calculate the positional change between the image volumes. The positional change between temporally neighboring volumes is called the framewise displacement (FD).  

Several researchers have proposed slightly different methods for calculating the FD. Power et al., Jenkinson et al., and Dosenbach et al. each propose a slightly different method for calculating the FD \cite{Power2012} \cite{Jenkinson2002} \cite{Dosenbach2017}. All three FD calculations produce correlated metrics: the FD metric proposed by Power et al. produces measurements approximately twice as large as the metric proposed by Jenkinson et al., and Dosenbach et al. reported a high correlation between their FD and Power’s FD \cite{Yan2013a} \cite{Dosenbach2017}. Herein, we use Power et al.'s version of the FD metric.

\subsection{Signal Effects of Motion}

In addition to changing the recorded position of the subject, motion impacts the established spin gradients, which introduces artifacts into the image sequence.

%According to principles of quantum mechanics, elementary particles intrinsically have two types of angular momentum: orbital angular momentum and spin angular momentum. The same types of elementary molecules have the same spin angular momentum, which is denoted by a spin quantum number specific to that particle type. A spin quantum number 

During an ideal MRI scan, the patient is sitting in the scanner and all molecules are aligned with the primary magnetic field $B_0$ in a relaxed state. Then, a radiofrequency (RF) pulse is applied to the field. The purpose of the pulse is to excite the molecules in a certain volume of physical space to orient the molecules to align to a secondary field in a different plane. When the pulse ends, the molecules precess back to their orientation in $B_0$. As they do, their small magnetic fields induce electric currents on the RF coil. The currents are received by the scanner as signals in frequency space. The volume of the space intended to be excited is known, and the signal produced by the induced electric current is used in conjunction to reconstruct the image in voxel space.

However, when the patient moves, the volume of space which was thought to be excited is not actually excited: some other volume of space, which may or may not overlap with the intended volume of space, is excited instead. Because the MRI scanner has no way to know this assumption is not true, it does not know that not all of the molecules in its intended area are relaxed and correctly aligned to the $B_0$ field at the end of the RF pulse. The scanner proceeds with the next RF pulse, which excites a new set of ``relaxed molecules'', some of which are still excited from the previous pulse. As a result, the signals produced in the second RF pulse are different than they should be. For example, signals that are smaller than they should be result in dark shadows within motion affected volumes of the sequence.

The previous few paragraphs in this section describe how motion disrupts the magnetic spin gradients present in the patient during an rs-fMRI scan. The spin gradients need time to recover to the correct magnetic field orientation, and up to eight to ten seconds may pass before the recovery is complete \cite{Power2014}. While the spin gradients are reorienting, the recorded BOLD signal will vary more than usual between temporally neighboring volumes. These variations are more difficult to quantify than the positional effects of motion.

One popular metric to measure changes in the BOLD signal due to patient motion was developed by Smyser et al. in 2010. Their metric is called DVARS, which measures the temporal \textbf{d}erivative of the root mean squared \textbf{var}iance over the voxels between two volumes \cite{Smyser2010}. Power et al. explain the steps to calculate DVARS in a separate study \cite{Power2012}. The DVARS value is calculated in two steps. The first step uses backward differences to approximate the derivative of the BOLD signal change between volumes $J_i$ and $J_{i-1}$ at every point $\vec{x}$ contained in both image volumes:

\begin{equation}
\frac{\partial}{\partial t} J_i(\vec{x}) \approx J_i(\vec{x}) - J_{i-1}(\vec{x}).
\end{equation}

% FD(J_i) = | \Delta d_{ix} | + | \Delta d_{iy} | + | \Delta d_{iz} | \\ + | \Delta \alpha_i | + | \Delta \beta_i | + | \Delta \gamma_i |

The second step calculates the root mean square of the approximated derivatives for all $N$ points $\vec{x}$:

\begin{equation}
DVARS(J_i) = \sqrt{ \frac{1}{N} \sum_{\vec{x} \in J_i, J_{i-1}} \left( \frac{\partial}{\partial t} J_i(\vec{x}) \right)^2 }.
\end{equation}

DVARs measures the change in BOLD signal intensity, which is highly related to motion-induced spin gradient changes. 

\section{Motion Prevention}

Various techniques and protocols have been developed to prevent patients from moving during the image acquisition process. Not all of these techniques are suitable for all patient populations, and some techniques have been designed specifically for certain populations.

\subsection{Pre-Scan: Education}

Educational material can be used to help the patient understand what to expect during an MRI scan as well as to teach the patient different behavioral coping strategies. The education materials can be used either before or upon arrival at the imaging facility. Most of the formal literature focuses on informative, distraction, and behavioral techniques to use during pediatric MRI scans, though many of the following approaches could be adapted for use with adults.

% Pediatric patients
In a review of the available literature, Alexander found several commonly used techniques to educate pediatric patients before and comfort or distract pediatric patients during radiology procedures \cite{Alexander2012}. Tools such as educational coloring books and short videos can expose patients to the types of equipment they can expect to see using a familiar, engaging medium. Pediatric patients can learn coping strategies to employ during the scan such as breathing techniques, imagery, and positive statements. Alexander notes that allowing a pediatric patient to choose a behavioral coping strategy gives the patient a sense of control and may encourage the patient to cooperate during the MRI acquisition.

Mock scanners and MRI simulators can also help the patient feel more comfortable during the scan. Barnea-Goraly et al. showed that both a commerical MRI simulator and a low-tech mock scanner desensitized pediatric patients between four and ten years of age to the MRI scanner with the results that 92.3\% of the acquired images could be used in high-resolution anatomical studies \cite{Barnea-Goraly2014}. 

% distraction
Several groups have investigated the role of auditory and visual distraction during an MRI acquisition. Headphones with music and stories or MR compatible video goggles can distract patients from the tedium of the scan \cite{Alexander2012} \cite{Barnea-Goraly2014} \cite{Harned2001}. Khan et al. found that a relatively simple moving light show can be helpful in distracting younger patients \cite{Khan2007}. Garcia-Palacios et al. performed a case study comparing the efficacy of music and immersive virtual reality tools as distractions during a mock scan \cite{Garcia-Palacios2007}. They suggest that immersive virtual reality may help decrease patient anxiety during a scan more effectively than music alone. As virtual reality technology improves, it may join headphones and MR compatible video goggles as an available distraction method.

Another helpful source of distraction for pediatric patients could be the patient's parent or parents. Having a parent involved with the scanning process may calm the patient and encourage him to cooperate; however, parental distress can further upset an anxious patient and complicate the scanning process \cite{Alexander2012}. 

These techniques for educating the patient and helping the patient cope with the anxiety that can accompany an MRI scan all depend on the ability of the patient to understand instructions and communicate with the scan team. Due to the gap in communication abilities, these techniques are not useful for young patients such as neonates, infants, toddlers, and possibly elementary school aged children. Other patient populations, such as those with developmental delays and neurobehavioral disorders, may also have difficulty adhering to these protocols. Even in patients with developed and intact communication skills, the techniques outlined here do not actively prevent the patient from moving during the scan: they only help the patient feel more comfortable with the MRI environment.

\subsection{During Scan: Sedation}

Sedation can be used to help a patient tolerate an MRI scan. Murphy and Brunberg retrospectively analyzed seven weeks of data from the MR department and found that 14.2\% of their adult patients required some form of sedation \cite{Murphy1997}. In a study about claustrophobia and MR acquisitions, Dewey et al. report that out of 55,734 patients who underwent MRI scans, a total of 1004 patients experienced claustrophobia and 610 of these patients required intravenous sedation before their scans \cite{Dewey2007}. Even though sedation allowed the patients mentioned in this paragraph to undergo an MRI scan, the authors of both studies note that sedation can result in adverse events and advise the reader to avoid patient sedation if possible.


Sedation can be used with pediatric patients, though the risks are more significant than with adult patients. Studies have shown that sedation for pediatric imaging can lead to hypoxemia and inappropriate sedation levels during image acquisition \cite{Malviya2000}. Pediatric patients can also expect ``motor imbalance and gastrointestinal effects,'' as well as agitation and restlessness for a period of hours after waking from sedation.

A report from the American Academy of Pediatrics and the American Academy of Pediatric Dentistry outlines the minimum set of criteria needed for a pediatric patient to be sedated for a procedure \cite{Cote2016}:
\begin{itemize}
\item The patient must be a suitable candidate for sedation based on their medical history and medical needs.
\item The patient's health status must be evaluated and verified by the sedation team prior to the procedure.
\item Informed consent must be obtained prior to the procedure.
\item Instructions for what to expect and how to transport the patient home safely must be provided to the patient's responsible adult.
\item At least one responsible adult must be with the patient at the medical facility, though the report recommends that two adults are present for patients who travel to and from the facility using car seats. This practice ensures that one adult can monitor the patient after the procedure while the other adult drives.
\item The patient's food and drink intake prior to the procedure should be taken into account to minimize the risk of pulmonary aspiration.
\item The clinician administering the sedation must have immediate access to emergency facilities, personnel, and equipment, and should monitor the patient for adverse events including respiratory events, seizures, vomiting, and allergic reactions.
\item There must be a clear protocol outlined for immediately accessing these emergency services.
\item Emergency equipment and drugs appropriate for the patient's size and age must be immediately available in case the patient needs to be resuscitated.
\item The information about the procedure must be correctly documented.
\item The facility should have a dedicated recovery area, and the status of the patient should be recorded when he is discharged. The patient should not be discharged if his level of consciousness and oxygen saturation do not meet recognized guidelines.
\item The patient may be held at the facility for prolonged monitoring after the procedure.
\end{itemize}
\noindent This report clearly states that the levels of monitoring suggested above should serve as minimum levels of involvement: clinicians should increase patient monitoring as needed for complex cases. Rutman has a similar and detailed perspective on patient monitoring during and after sedation, adding that two independent medical personnel should be present during the scan and one should be present until the patient is discharged \cite{Rutman2009}. Rutman also notes that all sedation and monitoring equipment must be MR compatible, which is a simple but important safety constraint. This constraint makes sedation less advisable if the appropriate equipment is not available.

Sedation in neonatal and infant populations is not recommended. The  U.~S.~Food and Drug Administration (FDA) issued a warning in late 2016 about repeated use of sedation or general anesthesia for patients under three years of age or for pregnant women during their third trimester \cite{FDA2016}. The warning states that while a single, relatively short exposure to sedative and anesthetic drugs is unlikely to impact the patient, the effects of prolonged exposure to these drugs are still being studied. Studies of sedative and anesthetic drugs in multiple animal models have shown that these drugs can lead to loss of nerve cells in the brain when the animals undergo prolonged, repeated exposure to them during period of brain development. More data is needed to determine if this effect translates to humans.

\subsection{During Scan: Feed and Sleep Protocols}

Neither sedation nor educational and behavioral techniques are appropriate to use with neonatal patients, but rs-fMRIs in neonates and infants are invaluable  in studying early brain development and neurological diseases \cite{Smyser2015}. A set of protocols have been developed specifically for scanning neonates without sedation. These protocols are referred to as ``feed and sleep'' or ``feed and bundle'' protocols.

Windram et al. describe a protocol in which the infant is deprived of food for four hours prior to the scan \cite{Windram2011}. At the scanning facility, the patient is fed by his mother, swaddled, and placed in a vacuum-bag immobilizer for the duration of the scan. 

Rather than deprive the patient of food prior to the scan, Gale et al.'s protocol recommends timing the scan so that the patient is fed after arrival on site and less than 45 minutes before the scan \cite{Gale2013}. The patient's ears are protected from the noise of the MR scanner by a layer of dental putty followed by headphones, and held in place by a hat. The patient is the swaddled and placed in the scanner once he is asleep. Additional foam padding is used to cushion the patient's head and provides extra noise protection.

Mathur et al. describe a protocol similar to the previous two: the patient's feeding schedule is adjusted so that he feeds 30-45 minutes before the scan time, and he is swaddled, given ear protection, and placed in a vacuum-bag immobilizer \cite{Mathur2008}.

When performed correctly, these protocols are generally successful and the neonatal patient will sleep for the duration of the MRI scan. However, the patient may shift slightly while asleep or may wake up and move mid-scan.

% application of rs-fMRI to evaluate neonates neurodevelopmental status of infants and neonates, and in personalized care by identifying patients who may benefit from certain therapies or neuroprotective interventions \cite{Smyser2015}.

\section{Prospective Motion Correction}

Since motion cannot be completely eliminated from rs-fMRI scans, different approaches have developed for correcting for the effects of motion after the scan. These approaches can be divided into two groups: those which monitor the patient's motion during the scan and those which work solely on the acquired sequences.

\subsection{Optical Motion Correction}

Several groups have developed methods for actively accounting for changes in the patient's position during an MRI scan. Optical-based methods record the patient's position using a combination of markers placed on the patient and one or more MR compatible optical cameras placed the scanner bore. The changes in the patient position from one time point to the next are used to update the MR parameters in real-time. Real-time updates of the MR parameters result in decreased spatial and spin-history effects of motion in the acquired sequences.

The first report of successful prospective motion correction using optical cameras and markers was by Zaitsev et al. in 2006 \cite{Zaitsev2006}. Their dual camera system was located outside of the MRI scanner and focused on the patient inside the system. Four reflective markers were attached to a modified mouthpiece originally designed for patient immobilization. Changes in the translation and rotation of the patient were recorded and processed during the exam. The processed changes were sent in real-time to the MRI scanner which used them to update the gradient orientations, RF frequencies, and RF phases at every time point during the acquisition process.

Aksoy et al. simplify this approach by using a single in-bore optical camera and replacing the 3D markers with a small 2D chessboard grid \cite{Aksoy2008}. Properties intrinsic to the camera as well as information about the camera's placement within the MRI scanner were recorded prior as part of a calibration process. During the scan, patient movements recorded using the optical camera were used to calculate the relationship between the patient's position at the current time point in the physical space and the patient's position at the initial time point in the MR space. The transformation needed to translate between these two positions was calculated on a laptop and passed to the MRI scanner to correct for motion in real-time. The camera used to record the position of the chessboard is mounted on the head coil. If the patient moves his head significantly, the camera will only be able to record the position of part of the chessboard marker. This limitation makes it difficult for the computer vision processing to identify the independent features on the standard chessboard. 

Forman et al. modified the chessboard marker to improve its flexibility \cite{Forman2011}. To differentiate between the different blocks in the chessboard, they added a unique, machine readable symbol to each black block in the chessboard. The symbols were chosen to be unique even in the event of rotation so that the identification of each block would be robust to rotation movements. The chessboard marker was embedded with MR-detectable agar so that the position of the marker could be detected in the MRI scan as well as by the in-bore camera. At each point during the scan, the image recorded by the in-bore camera was sent to a computer independent from the MRI controller. The independent computer detected the blocks of the chessboard and identified their spatial locations using the symbols contained within them. Their positions were checked by confirming the locations of the symbols with respect to each other. The confirmed locations of the corners of the black boxes were used to estimate the position of the patient, which was then sent to the MRI controller so that the magnetic gradients and RF hardware could be updated for the time point. The authors note that the latency of the system is a significant limitation to their system, but overall they experienced an increase in the accuracy of the estimates of the patient's position.

Several companies have developed commercial products for prospective motion correction in neurological images. KintetiCor's system uses a high resolution camera and a physical marker to detect motion \cite{kineticor}. The camera's resolution allows it to detect respiratory and cardiac motion through changes in skin displacement on the patient's forehead. The physical marker consists of pair of rectangles containing several concentric circles which are connected via a bridge across the nose. Any patient movement is reflected in the movement of the markers, which is also tracked through the camera. Both the camera system and the marker are MR compatible. Another company, TracInnovations, uses a stereo camera system to track all patient motion \cite{tracinnovations}. At the start of the scan, the stereo camera obtains a point cloud of the patient's position at that time. The points in the point cloud are averaged together to create a primary marker. Small facial motions, cardiac motion, and respiratory motion, are monitored using the point cloud. Larger head motions are monitored using both the point cloud and the primary marker. These two systems both allow prospective motion correction to be turned on or off: if the prospective motion correction is off, the system will still acquire the motion parameters so that the motion can be corrected retrospectively.

% Limitation: MR safe equipment
% Limitation: measurements must be made  
% Limitation: only rigid body motion. 
The methods discussed above have a few limitations due to the optical camera setups. For precise real-time motion correction, the camera or cameras must be carefully placed so that the position of the marker on the patient can be recorded. They must have a clear line of sight, which means they will be in the same room as the MRI scanner, if not within the scanner bore. The cameras and markers must be MR compatible, and the positions of the cameras and markers in physical space relative to the visual markers on the patient must be known. These positions are vital for the calculations used to measure the motions. Even if the motion measurements are accurate, the changes in position that are recorded and used to adapt the scan parameters will only be true for rigid body motion of the body part to which the markers are attached: any distortion of soft tissue will not be accurately accounted for during the motion correction unless the camera system was specifically built for and trained to do so. 
% To incorporate
% Limitation: optical trackers must have clean line of sight to subject 
% Limitation: systems using markers attached to patient suffer from imperfect attachment of the marker to the subject, where the marker slips or moves during the scan independently from the subject, 
% Look up Oliver Speck?

\subsection{External Sensors}

Cameras are not the only type of external sensor that can be used to measure motion during a rs-fMRI scan. 

There is a class of sensors which can take advantage of electrophysics properties of an MRI scanner. These sensors include wired nuclear magnetic resonance field probes, wireless inductivity coupled markers, and off-resonance markers. %CITATIONS AND DETAILS. 
The fact that these sensors directly interact with the magnetic field of the MR scanner means that protocols using these sensors must be modified to account for them. As a result of the protocol modification, the scan time might need to be extended.

%CHECK DETAILS AND CITATION 
As mentioned earlier in this chapter, respiration is a source of patient motion. Since respiration is relatively periodic, it can be monitored and accounted for within a scan protocol via gating. Gating prevents an image from being acquired unless the patient is in the expected state. In the case of respiration, the expected state is either complete inhalation or exhalation. The state of a patient's respiration can be tracked using respiration bellows. After acquiring the MRI sequence, volumes in the sequence can be grouped depending on when they were recorded in the breathing cycle. By only using volumes recorded during the same stage of the breathing cycle, the effects of respiratory motion can be mitigated.

%Cardiac: pulse oximeter (delay relative to central pulse) 
%ECG: less reliable at high field strengths (Frauenrath et al JMRI 2012; 36: 364-72
%Acoustic cardiac triggering: 
%Frauenrath et al, Investigative Radiology 2009,
%Frauenrath et al J Cardiovascular MR 2010

%Additional sensors complicate the scan setup

%Detection using RF coils? Changes conductivity and RF power (Buikman et al MRI 1988)

Ultimately, the addition of extra sensors complicate the process and set up of a rs-fMRI scan.

\subsection{Image Signal Motion Monitoring}

Dosenbach et al. have developed a tool to evaluate motion in rs-fMRI sequences as they are acquired \cite{Dosenbach2017}. It registers each volume to the initial volume of the rs-fMRI sequence immediately after the new volume is recorded. The parameters produced by this registration are used to calculate the framewise displacement between pairs of volumes, which is then compared to a set of displacement thresholds associated with the scan quality. The number of volumes that meet each threshold is used to determine how many more volumes are needed to obtain five minutes of low-motion volumes. This method for assessing the quality of a scan in real time is useful for ensuring images are acquired with a sufficient number of low-motion volumes. It can also aid the technologists in determining whether to prematurely terminate a scan, which may be desirable if the amount of time needed to obtain enough low-motion volumes is greater than the amount of time remaining for the patient in the scanner. 


\subsection{General Limitations of Prospective Motion Correction}

All types of prospective motion correction introduce a delay into the scanning process. The delay is due to the additional processing of some metrics to determine the patient's position, the transmission of these metrics to the MR scanner, and the adjustments the scanner makes to its next set of measurements. These alterations to the image acquisition during prospective motion correction actively change the image as it is acquired.
Maclaren et al. note that while prospective motion correction reduces imhomogeneities in the $B_0$ field, the $B_0$ field will still change when the patient moves and may change while the motion correction is occuring \cite{Maclaren2013}. % SO WHAT?

In order to view a scan not impacted by prospective motion correction, the patient often must undergo a second scan. It may be wise to build the second image acquisition into the same scan period as the prospectively motion corrected scan: unsuccessful prospective motion correction has the potential to drastically corrupt the acquired scan \cite{Zaitsev2017}.

Finally, though prospective motion correction has great power for managing motion during a scan, it cannot be used to recover motion-corrupted data in existing data sets.


\section{Retrospective Motion Correction}

Many groups have put significant effort into developing techniques for motion correction after the scan is acquired. Here, we discuss several commonly used techniques: volume registration, denoising, and filtering. % and give examples of pipelines which utilize these tools.

\subsection{Volume Registration}

The rs-fMR image is stored in computer memory as a set of 3D matrices. The values in corresponding cells of each matrix are considered to be aligned in this digital space (voxel space). The voxel space is defined by the imaging protocol and relates to the physical space through the spatial resolution of the image. Even though the spatial and voxel spaces for the image align, the contents of the image volumes may be misaligned due to patient movement. Because we cannot assume that an image is completely motion-free, we cannot directly compare the contents of each image volume in the rs-fMRI sequence. However, we can use image registration to align the contents of the image volumes to reduce the impact of motion on patient position.

Image registration is the process of morphing the contents of one image so that they overlap optimally with another image. The morphing operations include translation, rotation, scaling, skewing, and nonlinear adjustments. The linear and affine operations in this list should be used to perform rigid body registrations for organs such as the brain. Nonlinear operations can be used to fine-tune the alignment of more pliable organs such as the liver. All morphing operations are applied to one image repeatedly until it's contents optimally match those of the static reference image as determined by a chosen similarity metric. 

One of the earliest examples of image registration was described by Friston et al. in 1995 \cite{Friston1995}. They performed image registration on positron emission tomography (PET) scans and MRI scans of a human brain. During the registration process, one scan was designated as the ``reference'' image, which remained stationary, and the other scan was designated as the ``object'' image, which was transformed to match the reference image. Constraining the alignment process to transforming a single image into the coordinates of the other image rather than transforming both images into an independent coordinate frame simplifies the registration process.

\begin{figure}
\centering
\includegraphics[width=.7\textwidth]{2/traditional-registration.png}
\caption{The traditional approach to volume registration in an rs-fMRI sequence consists of registering all volumes in the sequence to a single reference volume.}
\label{fig:ch4:traditional-reg}
\end{figure}

When performing image registration on a sequence of image volumes, one volume must be chosen as the reference volume for the entire sequence. All other volumes in the sequence are registered to this volume. An example of this process can be seen in Figure \ref{fig:ch4:traditional-reg}. In subsequent work, Friston et al. used the first volume in the rs-fMRI sequence as the universal reference image \cite{Friston1996}. Common choices for the reference volume include the volume with the least FD to all other volumes in the sequence, a volume produced by averaging all volumes in the sequence, or the first volume in the sequence \cite{Friston1996} \cite{Liao2005}. In our implementation, we chose to use the first volume in the sequence as the reference volume.

One drawback to this traditional approach to volume registration is that it only minimizes the differences between all the image volumes in the sequence and the reference volume. The key word here is minimizes: minimizing differences between image volumes does not mean that there are no differences between the image volumes. Image registration is an optimization problem, and its goal is to find the overlap between a pair of volumes with as few differences as possible either within a defined time period or until the optimization cost does not change above a certain tolerance for a certain amount of time. These practical constraints on optimization problems mean that there may still be differences between other pairs of image volumes in the sequence that do not include the reference volume. 

Variations on Friston et al.'s framework have been developed over the last two decades. Liao et al. suggested that a rs-fMRI sequence could be viewed as a hidden Markov model, and reflected this idea in their suggested registration framework \cite{Liao2016}. They still use the first volume in the image sequence as the reference volume. Their framework uses the transformation of the previous volume to the reference volume to initialize the transformation for the current volume and the reference volume. 

It has been demonstrated that image registration across the entire image sequence reduces the effects of motion on the image sequence, though they do note that motion also effects the image due to changes in the spin history of the image. These effects are not correctable by global volume registration alone and will be discussed later in this chapter.

\subsection{Denoising}

Denoising techniques can be applied to a rs-fMRI after global volume registration is completed. They consist of regressions of various confound variables. 

Regression of the global signal (global signal regression, GSR) corrects for variance between temporal signals within a voxel and for the mean BOLD signal across all voxels \cite{Power2014} \cite{Satterthwaite2013} \cite{Yan2013} \cite{Yan2013a}. GSR has been shown to reduce spuriously increased long-distance correlations in functional connectivity studies, but may inadvertently weaken shorter-distance connections \cite{Jo2013} \cite{Power2014}  \cite{Satterthwaite2012}. 

Other regression parameters have been investigated. Commonly, the six rigid realignment parameters and their first order derivatives are suggested as regression parameters \cite{Power2012} \cite{Satterthwaite2012} \cite{VanDijk2012}. More recently, researchers have also incorporated the rigid realignment parameters from surrounding timpoints \cite{Power2014} \cite{Satterthwaite2013} \cite{Yan2013a}.

, signals from white matter or cerebral spinal fluid \cite{Power2014} \cite{Satterthwaite2013} \cite{Yan2013a} \cite{Jo2010}, and components identified using principal or independent component analysis \cite{Pruim2015} \cite{Salimi-Khorshidi2014} \cite{Behzadi2007}. Regression of each of these sets of parameters has been shown to reduce the effects of motion in the sequence but not remove them entirely \cite{Power2015} \cite{Parkes2017}. 



Patriat et al. performed a robust comparison of different regression parameters on their MotSim motion data set \cite{Patriat2017}. They included rigid realignment parameters, but also used parameters obtained by performing principle component analysis (PCA) on the image sequences. PCA generates a set of linear, uncorrelated components that reflect the main features of a patient's motion. The list of parameter combinations included 
\begin{itemize}
\item 12mot: The six rigid realignment parameters and their first derivatives,
\item 12for: The first 12 principal components of the whole brain before realignment,
\item 12back: The first 12 principal components of the whole brain after realignment,
\item 12both: The first 12 principal components of the whole brain both before and after realignment,
\item 24mot: the six rigid realignment parameters of the current volume, the six rigid realignment parameters of the previous volume, and the square of these rigid realignment parameters,
\item 24both: the first 24 principle components of the whole brain before and after realignment.
\end{itemize}

\noindent They found that the features extracted from the image sequence using PCA explained more variance in the image sequence (measured using $R^2$) than the rigid realignment parameters. They showed that increasing the number of regressors increased the amount of variance explained, but with diminishing returns. While their work is promising, their experiment was performed on a simulated data set using healthy subject data and required an accurate estimate of the subject's head motion.

\subsection{Filtering}

Filtering, which is also referred to as censoring, involves the identification and removal or interpolation of volumes containing high quantities of motion. Two popular techniques are scrubbing and spike regression. Power et al.’s scrubbing technique removes volumes with more than 0.2 mm of FD \cite{Power2012}. Spike regression identifies volumes with large FD and replaces them with interpolated volumes \cite{Satterthwaite2013}. Unfortunately, these filtering techniques ultimately result in the loss of data as volumes are removed from the sequence. A third technique called despiking detects signal spikes at the voxel level and interpolates over the spikes \cite{Jo2013} \cite{Patel2014}. Despiking does not remove volumes, but could accidentally remove valuable signals. 

\subsection{Spin History Distortion Correction}

A number of post-acquisition methods have been developed specifically to correct for distortions due to the impact of motion on the magnetic field. The usability of these dynamic distortion correction methods has been studied in a few specific cases, but their generalizability has yet to be confirmed in a broader range of fMRI studies \cite{Zaitsev2017}.

\section{Image Usability}


%The FD metric only measures the positional effects of motion, not the variations in signal in individual voxels caused by motion. Changes in signal between volumes can be measured using the temporal derivative of the variance in the BOLD signal intensity (DVARS) between neighboring volumes \cite{Power2012} \cite{Smyser2015}. 

Even though the effects of motion on the patient position and the recorded signal can be measured, we still need gold standard criteria to determine whether an image containing motion can be used. Patients move slightly due to breathing and cardiac function, and the BOLD signal naturally fluctuates over time. Some motion is expected; however, we need to know how much motion can be present in the image before it is considered to be corrupted by it. Power et al. established thresholds for FD and DVARS to determine the usability of a pair of images:
\begin{itemize}
\item FD less than or equal to 0.2 mm from previous volume, and
\item DVARS less than or equal to 25 units on a normalized scale of [0, 1000] signal units \cite{Power2014}
\end{itemize}

Image volumes that meet these criteria are considered to be low-motion.

The time span of low-motion data is highly debated. van Dijk et al. established that approximately five minutes of low-motion data is sufficient for use in functional connectivity analysis \cite{VanDijk2012}. However, a recent study by Laumann et al. suggests that at least 10 minutes of low-motion data is essential for obtaining high-quality results \cite{Laumann2015}. From a practical standpoint, it is difficult to obtain even five minutes of low motion data from certain patient populations, so radiology technicians and neuroimaging study designers are often content with the five minute time standard. 

\section{Summary}

Resting-state fMRIs are four dimensional images which record BOLD signal in active areas of the brain. The BOLD signal can be used to evaluate the functional connnectivity of different underlying networks in a patient's brain. Since rs-fMRIs are highly sensitive to motion, clinicians and psychologists have devised techniques to inform patients about what they can expect during an MRI scan as well as different coping mechanisms to help them remain calm during the scan. These techniques do not prevent the patient from moving, but approaches that do are not always appropriate to use during a rs-fMRI scan. Techniques and algorithms to prospectively and retrospectively remove motion from rs-fMRIs have also been developed, though they are not always successful in removing the effects of motion. Ultimately, the amount of motion present in the rs-fMRI sequence dictates whether or not the sequence can be used in clinical or research applications.

In the next chapter, we will discuss rs-fMRIs in the context of our chosen population of CHD patients.
\chapter{METHODS: Motion Correction}
\label{ch3:moco}

In the previous chapter, we discuss several techniques used to retrospectively correct motion. Motion correction pipelines may use denoising and filtering, but all pipelines begin with volume registration. In this chapter, we discuss a different approach to volume registration, how it compares to traditional volume registration, and how volume registration fits into a motion correction pipeline. 

%In this section, we discuss the two registration frameworks we apply to our rs-fMRIs: the traditional global volume registration framework and the DAG-based global volume registration framework. The registration frameworks will later be evaluated in comparison to each other, but will also be evaluated in the context of a complete motion correction pipeline. The motion correction pipeline of choice, ICA, will also be discussed in this section.

\section{Directed Acyclic Graph Based Volume Registration}

As discussed previously, the major drawback to Friston et al.'s approach to volume registration is that it only minimized the positional differences between the reference volume and the rest of the sequence. This drawback demonstrates an inability for the traditional approach to account for relationships in the patient's position throughout the scan. Intuitively, we know that the patient's position at any volume in the scan is more similar to his position in the immediately previous or subsequent volume than to another randomly chosen volume in the image.

In our proposed framework, we wish to account for these spatiotemporal relationships between temporally neighboring volumes in the sequence. To accomplish this goal, we start by viewing the rs-fMRI sequence as a directed acyclic graph (DAG). A DAG consists of a set of nodes and edges. Each edge has a direction associated with it and connects a pair of nodes. Since a DAG contains no cycles, there is no possible path back to a node once it has been traversed. 

\begin{figure}
\centering
\includegraphics[width=.7\textwidth]{3/dag-chain.png}
\caption{A rs-fMRI can be viewed as a directed acyclic graph where each volume is a node and the edges connect from each volume $i$ to the following volume $i+1$.}
\label{ch3:fig:dag-chain}
\end{figure}

In the case of an rs-fMRI, each volume can be considered a node. The relationship between each pair of temporally neighboring volumes is represented as a directed edge connecting the node for the first volume to the node for the next volume. The acyclic nature of the DAG means that once a patient was in a specific position, he will never return to that exact same position with the exact same neurons firing. The position of the subject and his brain activity as measured by the BOLD signal may be similar in subsequent image volumes, but it will never be precisely the same. The perspectives of an rs-fMRI sequence as a set of images and of the sequence as a DAG can be seen in Figure \ref{ch3:fig:dag-chain}.

The cost of transitioning from one node to the next in our DAG has a parallel representation to the combination of the positional transformation needed to align volume $i$ to volume $i+1$ and the signal change between the volumes. This representation can be written as 

\begin{equation}
J_{i+1} = \phi_{i,i+1} J_i + \delta s_{i,i+1} + \epsilon
\end{equation}

\noindent{where $J_i$ and $J_{i+1}$ are volumes $i$ and $i+1$, $\phi_{i,i+1}$ is a matrix of transformation parameters that must be applied to $J_i$ to achieve the patient’s position in $J_{i+1}$, $\delta s_{i,i+1}$ is the natural change in BOLD signal, and $\epsilon$ is the change in BOLD signal due to motion. Currently, there is no way to estimate the natural change in BOLD signal and the change in BOLD signal due to motion without incorporating additional information about the MRI scanner and the patient that is not included in a rs-fMRI. We simplify our representation of the relationship between two volumes to}

\begin{equation}
J_{i+1} = \phi_{i,i+1} J_i + \epsilon^*
\end{equation}

\noindent{where $\epsilon^*$ is the change in the BOLD signal that cannot be accounted for after aligning the patient’s position in the two volumes. Here, we use the notation $\epsilon^*$ to represent the generic error change in BOLD signal across any pair of volumes.}

After aligning two volumes $i$ and $i+1$, we will then align volumes $i+1$ and $i+2$:

\begin{equation}
\begin{split}
J_{i+2} & = \phi_{i+1,i+2} J_{i+1} + \epsilon^* \\
& = \phi_{i+1,i+2} (\phi_{i,i+1} J_i + \epsilon^*) +\epsilon^*\\
& = \phi_{i+1,i+2} \phi_{i,i+1} J_i + \epsilon^{*'}\\
\end{split}
\end{equation}

\noindent Traditional volume registration assumes that 

\begin{equation}
\phi_{i,i+2} = \phi_{i+1,i+2} \phi_{i,i+1}
\end{equation}

\begin{figure}
\centering
\includegraphics[width=.7\textwidth]{3/dag-registration.png}
\caption{The traditional approach to volume registration in an rs-fMRI sequence consists of registering all volumes in the sequence to a single reference volume.}
\label{ch3:fig:dag-reg}
\end{figure}

\noindent{and calculates $\phi_{i,i+2}$ directly. We argue that this assumption is not true in all cases. Rather than directly calculate $\phi_{0,i}$ and use it to align volume $i$ to the reference volume as the traditional method does, we calculate each component $\phi$ that is a factor of $\phi_{0,i}$. Each component $\phi_{i,i+1}$ is combined with the preceding $\phi_{0,i}$s to recursively align volume $i+1$ to the reference volume without making the large and often inaccurate transformations required by directly calculating $\phi_{0,i+1}$.} This process is outlined in Figure \ref{ch3:fig:dag-reg}.

\section{Independent Component Analysis}

The purpose of image registration is purely to ensure the position of the patient throughout the entire rs-fMRI is consistent. After registration, the image still contains BOLD source signals and noise signals caused by factors other than brain activity. The challenge of separating these combined signals is called blind source separation (BSS). 

We chose to focus on an independent component analysis (ICA) approach for solving the BSS problem. The specific technique we use has been described by Beckmann and Smith as probabilisic ICA. This section aims to provide an overview of the probabilistic ICA technique. For further details, please refer to the technical reports by the FMRIB group \cite{Beckmann2004} \cite{Woolrich2004} \cite{Beckmann} \cite{Smith2004}.

Probabilistic ICA is a linear regression model which performs mixing in the original data space and assumes the true BOLD signal has been confounded by Gaussian noise. These constraints mean that BSS can be solved in three steps:

\begin{enumerate}
\item Estimate a joint subspace consisting of source and noise signals and an noise subspace orthogonal to the joint subspace,
\item Estimate the independent sources in the joint subspace, and 
\item Assess the statistical significance of the independent sources.
\end{enumerate} 

Probabilistic ICA treats the voxel intensity values in every frame of the image sequence as a matrix of $V$ voxels across $n$ time points. For each voxel $v_i \in V$, the observed signal in that voxel can be modeled as

\begin{equation}
\label{ch3:eq:ica01}
\vec{x_i} = A \vec{s_i} + \mu + \vec{\eta_i}
\end{equation}

\noindent This equation allows three different types of signals to contribute to the observed voxel values $\vec{x_i}$ for a given voxel across all $n$ timepoints in the sequence. The first type of signal is a vector of non-Gaussian source signals $\vec{s_i}$ across all $n$ timepoints. The source signals are modulated by mixing matrix $A$ whose shape is the number of time points $n$ by the number of source signals $q$. The second type of signal is an offset denoted by $\mu$. The offset constrains the observed signals to be centered around the mean of all observed signals. The third type of signal $\vec{\eta_i}$ is a vector of noise throughout the duration of the sequence. To summarize, probabilistic ICA explicitly assumes that the observed signal in a given voxel can be divided into non-Gaussian source signals, isotropic Gaussian noise signals, and some offset. This assumption makes it easier to separate a source signal from a noise signal: a noise signal will have a Gaussian distribution while a source signal will not. \textbf{The goal of probabilistic ICA is to identify the source signals, $\vec{s}$.}

With this combination of signals in mind, we can write the covariance matrix of the observed data $x$ as

\begin{equation}
\label{ch3:eq:cov-01}
R_x = \langle x_i x_i^T \rangle = AA^T + \sigma ^2 I
\end{equation}

\noindent where $A$ is the mixing matrix, $\sigma^2$ is the standard deviation of the noise, and $I$ is $n$x$n$ identity matrix. The covariance matrix of the observed data $R_x$ can be calculated, but $A$ and $\sigma^2$ are both unknown. The noisy observed data is transformed with respect to the noise sources using a process called whitening. The whitening with respect to noise enforces the assumption of noise following an isotropic Gaussian distribution with a mean of zero and a standard deviation of $\sigma^2$.

The mixing matrix $A$ can be estimated using maximum likelihood estimation. Beckmann and Smith use singular value decomposition of the observed data $X = U(N\Lambda)^{\frac{1}{2}} V$ to model the estimator of $A$:

\begin{equation}
\hat{A}_{ML} = U_q(\Lambda_q - \sigma ^2 I_q)^{\frac{1}{2}}Q^T
\end{equation}

\noindent where $U_q$ contains the eigenvectors associated with the $q$ largest eigenvalues, $\Lambda_q$ contains the $q$ largest eigenvalues, and $Q$ is a $q$x$q$ orthogonal rotation matrix in the whitened observation space such that $QQ^T = I$. %This estimator assumes that the number of source signals $q$ is known. In a noiseless system, the number of source signals is equivalent to the rank of the covariance matrix of the observed data. However, rs-fMRIs are inherently noisy. In this more general case, the number of sources is more complex to determine. Beckmann and Smith suggest that the number of source signals is the same as the number of eigenvalues of the covariance matrix that violate the ``sphericity assumption of the isotropic Gaussian noise model'' \cite{Beckmann2004}. 
The eigenvectors and eigenvalues can be calculated from $X$, but $\sigma$ and $Q$ remain unknown. As noted earlier, the matrix $Q$ is an orthogonal rotation matrix which, when applied to the whitened data $\tilde{x}$, has the same effect of applying an unmixing matrix to the observed data: 

\begin{equation}
\label{ch3:eq:unmixing-01}
W \vec{x} = Q \tilde{x} = \hat{s}
\end{equation}

\noindent Both matrix-vector multiplications serve to estimate individual source signals $\hat{s}$. The estimated source signals are identified by projecting the whited data $\tilde{x}$ onto each row $r$ of the unmixing matrix $Q$ a total of $q$ times:

\begin{equation}
\hat{s_r} = Q_{r,:} \tilde{x}
\end{equation}

\noindent where the $Q_{r, :}$ represents row $r$ of matrix $Q$.  \textit{(Note: A key assumption in this step is that the rows of the unmixing matrix are mutually orthogonal so that they cover the entire space of signal sources. Additional steps described by Beckmann and Smith can be taken to incoporate prior information about the voxels into this step \cite{Beckmann2004}.)}

At this point, the standard deviation of the noise $\sigma^2$ and the source signals are unknown. We can solve the following system of equations jointly to resolve these two unknown quantities:

\begin{equation}
\hat{s}_{ML} = (\hat{A}^T\hat{A})^{-1}\hat{A}^Tx = \hat{W}x = Q \tilde{x}
\end{equation}

\begin{equation}
\hat{\sigma}_{ML}^2 = \frac{1}{n-q} \sum_{l=q+1}^p \lambda_l .
\end{equation}

Solving these equations is an iterative process. First, the mixing matrix and source signals are estimated. These estimations are used to calculate the corresponding estimator of the standard deviation of the noise. Then, the residual noise $\hat{\eta}_i$ at each voxel $v_i$ is calculated:

\begin{equation}
\label{ch3:eq:res-01}
\hat{\eta}_i = (I - \hat{W}^T\hat{W}) x_i.
\end{equation}

\noindent Recalling from Equation \ref{ch3:eq:ica01} how probabilistic ICA views a signal, Equation \ref{ch3:eq:res-01} becomes: 

\begin{equation}
\hat{\eta}_i = (I - \hat{W}^T\hat{W}) A + (I - \hat{W}^T\hat{W}) \eta
\end{equation}

When the correct number of sources has been identified, the estimated mixing matrix will fully span the source signal space. Then, the residual noise will only be related to the true noise:

\begin{equation}
\hat{\eta}_i = 0 + (I - \hat{W}^T\hat{W}) \eta
\end{equation}

Upon reaching this stage in the probabilistic ICA technique, the source signals have been approximated. The source signals are called spatial independent component maps. Normalizing the values in these maps by the variance of the noise produces $Z$-statistic maps. $Z$-statistic maps can be analyzed to identify voxels with statistically significant activations. These activations are attributed to BOLD signal.

One of the major limitations of ICA is that it is highly data driven. It assumes the dataset contains a sufficiently large number of images, each with a sufficiently large number of voxels. Even assuming an ideal data set, the true value of the mixing matrix is dependent on the observed data \cite{Beckmann2004}. Fluctuations in the data can lead to deviations of the residual noise in certain voxels from the true noise. These deviations can produce \textit{type-I} and \textit{type-II} errors when examining the $Z$-statistic maps to identify statistically significantly activated voxels.
 
Additionally, the developers of probabilistic ICA note that not all noise follows the isotropic Gaussian assumption. Noise based in the patient's physiology is likely to be structured in a way that is non-Gaussian. The non-Gaussian noise signals can still be separated from the BOLD source signals, but only if these noise signals are not highly correlated with the source signals.

\section{Motion Correction Pipeline and Implementation}

Both the traditional and novel volume registration techniques were applied independently to each image from the subject cohorts described in Chapter \ref{ch:data}. After registration, three versions of each image existed: the original BOLD sequence, the sequence modified using traditional volume registration, and the sequence modified using the novel registration method.

The registration algorithms applied to rigid tissue types used affine registration with two degrees of granularity. When applied to soft tissue types (ie, placenta),  three nonlinear transformations with increasing granularities were performed after the affine registrations. The exact parameters used for each volume registration can be seen in Appendix A\ref{appendix:registration-params}. The registration frameworks were implemented in Python using the nipype (Neuroimaging in Python Pipelines and Interfaces) library \cite{Gorgolewski2011}. Volume registration used the ANTs (Advanced Normalization Tools) tools as a backend \cite{Avants2014}.

After performing volume registration to ensure the patient is in the same physical space throughout the image sequence, the image sequence may still contain artifacts due to motion. Our registered sequences underwent motion correction via a well-established motion correction pipeline. We chose to use the independent component analysis (ICA) pipeline outlined by Beckmann and Smith \cite{Beckmann2004}. The motion corrected sequences produced by FMRIB's MELODIC tool were saved alongside the original and registered sequences.  

\section{Evaluating Registered and Motion Corrected Sequences Against Gold Standard Usability Thresholds}

The main goal of motion correction is to reduce the effects of motion on the image so that it is usable. The gold standards for rs-fMRI usability as established by Power et al. are that the FD and DVARs metrics must change less than 0.2 mm and 2.5\% normalized voxel units between at least 50\% of the neighboring volumes. The FD and DVARs metrics between each pair of subsequent image volumes were calculated for the original, registered, and motion corrected sequences. The metrics for each sequence were then compared to the gold standard image usability thresholds. This comparison answers the key question of how each registration framework impacts an established motion correction pipeline.

Additionally, a smaller comparison of the registered sequences was conducted. This comparison evaluates the immediate impact of the registration algorithm on the image sequence. It is highly unlikely that an entire image sequence would meet the Power et al. usability thresholds after only the initial step of a motion correction pipeline, but it is valuable to examine the impact of a volume registration algorithm at each stage of the pipeline. 

\textbf{Implementation.} We calculated the FD and DVARS metrics defined by Power et al. using the FSLMotionOutliers tool \cite{Power2012}. 


\chapter{METHODS}
\label{ch:methods}

\section{Motion Correction}

In the previous chapter, we discuss several techniques used to retrospectively correct motion. Motion correction pipelines may use denoising and filtering, but all pipelines begin with volume registration. In this section, we discuss a different approach to volume registration, how it compares to traditional volume registration, and how volume registration fits into a motion correction pipeline. 

%In this section, we discuss the two registration frameworks we apply to our rs-fMRIs: the traditional global volume registration framework and the DAG-based global volume registration framework. The registration frameworks will later be evaluated in comparison to each other, but will also be evaluated in the context of a complete motion correction pipeline. The motion correction pipeline of choice, ICA, will also be discussed in this section.

\subsection{Directed Acyclic Graph Based Volume Registration}

As discussed previously, the major drawback to Friston et al.'s approach to volume registration is that it only minimized the positional differences between the reference volume and the rest of the sequence. This drawback demonstrates an inability for the traditional approach to account for relationships in the patient's position throughout the scan. Intuitively, we know that the patient's position at any volume in the scan is more similar to his position in the immediately previous or subsequent volume than to another randomly chosen volume in the image.

In our proposed framework, we wish to account for these spatiotemporal relationships between temporally neighboring volumes in the sequence. To accomplish this goal, we start by viewing the rs-fMRI sequence as a directed acyclic graph (DAG). A DAG consists of a set of nodes and edges. Each edge has a direction associated with it and connects a pair of nodes. Since a DAG contains no cycles, there is no possible path back to a node once it has been traversed. 

\begin{figure}
\centering
\includegraphics[width=.7\textwidth]{4/dag-chain.png}
\caption{A rs-fMRI can be viewed as a directed acyclic graph where each volume is a node and the edges connect from each volume $i$ to the following volume $i+1$.}
\label{ch4:fig:dag-chain}
\end{figure}

In the case of an rs-fMRI, each volume can be considered a node. The relationship between each pair of temporally neighboring volumes is represented as a directed edge connecting the node for the first volume to the node for the next volume. The acyclic nature of the DAG means that once a patient was in a specific position, he will never return to that exact same position with the exact same neurons firing. The position of the subject and his brain activity as measured by the BOLD signal may be similar in subsequent image volumes, but it will never be precisely the same. The perspectives of an rs-fMRI sequence as a set of images and of the sequence as a DAG can be seen in Figure \ref{ch4:fig:dag-chain}.

The cost of transitioning from one node to the next in our DAG has a parallel representation to the combination of the positional transformation needed to align volume $i$ to volume $i+1$ and the signal change between the volumes. This representation can be written as 

\begin{equation}
J_{i+1} = \phi_{i,i+1} J_i + \delta s_{i,i+1} + \epsilon
\end{equation}

\noindent{where $J_i$ and $J_{i+1}$ are volumes $i$ and $i+1$, $\phi_{i,i+1}$ is a matrix of transformation parameters that must be applied to $J_i$ to achieve the patient’s position in $J_{i+1}$, $\delta s_{i,i+1}$ is the natural change in BOLD signal, and $\epsilon$ is the change in BOLD signal due to motion. Currently, there is no way to estimate the natural change in BOLD signal and the change in BOLD signal due to motion without incorporating additional information about the MRI scanner and the patient that is not included in a rs-fMRI. We simplify our representation of the relationship between two volumes to}

\begin{equation}
J_{i+1} = \phi_{i,i+1} J_i + \epsilon^*
\end{equation}

\noindent{where $\epsilon^*$ is the change in the BOLD signal that cannot be accounted for after aligning the patient’s position in the two volumes. Here, we use the notation $\epsilon^*$ to represent the generic error change in BOLD signal across any pair of volumes.}

After aligning two volumes $i$ and $i+1$, we will then align volumes $i+1$ and $i+2$:

\begin{equation}
\begin{split}
J_{i+2} &= \phi_{i+1,i+2} J_{i+1} + \epsilon^* \\
&= \phi_{i+1,i+2} (\phi_{i,i+1} J_i + \epsilon^*) +\epsilon^*\\
&= \phi_{i+1,i+2} \phi_{i,i+1} J_i + \epsilon^{*'}\\
\end{split}
\end{equation}

Traditional volume registration assumes that 

\begin{equation}
\phi_{i,i+2} = \phi_{i+1,i+2} \phi_{i,i+1}
\end{equation}

\begin{figure}
\centering
\includegraphics[width=.7\textwidth]{4/dag-registration.png}
\caption{The traditional approach to volume registration in an rs-fMRI sequence consists of registering all volumes in the sequence to a single reference volume.}
\label{ch4:fig:dag-reg}
\end{figure}

\noindent{and calculates $\phi_{i,i+2}$ directly. We argue that this assumption is not true in all cases. Rather than directly calculate $\phi_{0,i}$ and use it to align volume $i$ to the reference volume as the traditional method does, we calculate each component $\phi$ that is a factor of $\phi_{0,i}$. Each component $\phi_{i,i+1}$ is combined with the preceding $\phi_{0,i}$s to recursively align volume $i+1$ to the reference volume without making the large and often inaccurate transformations required by directly calculating $\phi_{0,i+1}$.} This process is outlined in Figure \ref{ch4:fig:dag-reg}.

\subsection{Motion Correction Pipeline}

After performing volume registration to ensure the patient is in the same physical space throughout the image sequence, the image sequence may still contain artifacts due to motion. Many pipelines exist for correcting motion in registered rs-fMRIs. 

\section{Implementation: Tools and Libraries}

The registration frameworks described in this section were implemented in Python using the nipype (Neuroimaging in Python Pipelines and Interfaces) library \cite{Gorgolewski2011}. Affine volume registration was performed using ANTs (Advanced Normalization Tools) \cite{Avants2014}. The metric used to estimate the dissimilarity between the pairs of volumes being registered was cross-correlation with a local window size of 5 voxels. 

To calculate metrics, we used several existing tools. FLIRT (FMRIB’s Linear Image Registration Tool) was used to calculate the correlation ratio between each possible pair of volumes in the sequences \cite{Jenkinson2001} \cite{Jenkinson2002}. We then used the average and standard deviation of the correlation ratio distribution of each image to compare the images. We calculated the FD and DVARS metrics defined by Power et al. using the FSLMotionOutliers tool \cite{Power2012}. These metrics were calculated for each image and were used for evaluation of the efficacy of the registration frameworks.
\chapter{DATA}
\label{ch5:data}

The data used to test the hypothesis and aims introduced in the previous chapter are drawn from a set of simulated rs-fMRI sequences and four clinical groups. In this chapter, we will first discuss the mechanism built to simulate brain activity, scanner noise, and motion in rs-fMRI sequences. Then we will discuss the clinical images. Because motion causes problems in MR images across all stages of life, we used images from cohorts of healthy and CHD fetal, neonatal, and preadolescent subjects as well as images of aging subjects enrolled in the Alzheimer's Disease Neuroimaging Initiative (ADNI) study.

\section{Simulated Sequences}

\subsection{SPECTr: Simulated Phantom Emulating Cranial Transformations}

Every MRI scanner is different, so a stand-in model for an organ or tissue type is often used to calibrate an MRI scanner. The model is designed to have specific physical properties which mimic the physical properties of the organ or tissue. These properties can be accurately measured during the design process of this model so that the radiologist or researcher looking at images of the model can know the ground truth of the model. Because these models mimic true organs and tissues, they are called phantoms. 

We will generate a simulated phantom image using the rs-fMRI of a healthy adult male. A single volume will be selected from the rs-fMRI sequence. This volume will be duplicated to create a generated image with 150 instances of the same volume. This sequence will be our base phantom sequence. 

A copy of the base phantom sequence will be made and a subvolume in the same location of every volume will be selected. In the subvolume of each frame, a small amount of noise generated using a normal Gaussian distribution will be added to simulate changes in blood oxygen level-dependent signal over time. The noise will be generated from a normal Gaussian distribution will be added to each frame. This image sequence will be referred to as our BOLD phantom sequence.

\subsection{Simulated Images}

\section{Congenital Heart Disease Cohorts}

Congenital heart defects and congenital heart disease (CHD) both refer to defects in the heart or the vessels around the heart which formed during  fetal development. Heart defects affect how blood moves into, through, and away from the heart. However, cardiac conditions are not the only complications CHD must deal with. In recent years, researchers have found that there is a relationship between CHD and neurocognitive disorders. The CHD rs-fMRIs used in this study were gathered as part of ongoing studies of the relationship between CHD and neurodevelopment. Data from these studies was obtained through studies approved by the IRB at the Children's Hospital of Pittsburgh of UPMC and the University of Pittsburgh. The data is stored and accessed in compliance with all HIPPA policies.

\subsection{CHD and Neurodevelopment}

CHD consists of a variety of defects affect the vessels and chambers of the heart. It has a worldwide prevalence of about 8 per 1000 live births, meaning about 1.35 million children are born with CHD every year. Since the survivability of CHD has increased from 10\% to 90\%, the medical community is faced with a growing, aging population of CHD patients. Many of these patients also suffer from neurocognitive disorders that co-occur with CHD. The neurocognitive disorders are usually diagnosed using at least one of many psychological survey-based evaluations, but these methods are subjective. rs-fMRIs could be used to identify patients who have functional connectivity patterns associated with different neurocognitive disorders, and eventually may be used to identify patients who are at risk for developing these disorders.

CHD can affect any combination of heart chambers and blood vessels with varying degrees of severity. The lesions prevent the cardiopulmonary system as a whole from functioning correctly, but pinpointing and treating the defects effectively can be a complex process.
 
% Causes of CHD: genetic syndromes, single gene mutations, environmental exposure, and unknown
There are a number of genetic and environmental factors associated with different presentations of CHD \cite{Mozaffarian2016}. Genetic conditions such as Down syndrome, Turner syndrome, 22q11 deletion syndrome, Williams syndrome, and Noonan syndrome are associated with different CHD presentations. Maternal behaviors such as smoking and binge drinking are known to cause heart problems in the fetus. Other maternal risk factors are obesity, folate deficiency, and living at a high altitude. Paternal exposure to phthalates, anesthesia, sympathomimetic medications, pesticides, and solvents may increase the risk of the fetus for developing CHD. While there are quite a few factors in this list, there are many CHD cases whose causes are unknown.

\begin{figure}
\centering
\includegraphics[width=0.6\textwidth]{5/chd-defects-usa.png}
\caption{Table of prevalences of congenital heart defects borrowed temporarily from \cite{Mozaffarian2016}.}
\label{ch5:fig:usa-defects-prev}
\end{figure}

% Diagnosis
The process of diagnosing CHD can begin before birth. A specialized ultrasound test called fetal echocardiography can detect heart abnormalities as early as the second trimester of the pregnancy. Additional tests, such as amniocentesis and follow-up ultrasounds may be used to determine treatment options before the patient is born. Generally, severe CHD cases present and are detected at earlier stages, but minor defects may not become apparent until the patient is older. Tests used to diagnose CHD in postnatal patients include electro- and echo-cardiograms, chest x-rays, pulse oximetry, exercise stress tests, computed tomography or MRI scans, and cardiac catheterization. Treatment of different defects varies from monitoring and medication to surgery and cardiac implants.

The incidence of CHD in live births vary across countries and continents. The United States reports approximately 4-10 CHD case per 1000 live births. Europe and Asia see about 6.9 and 9.3 CHD cases per 1000 live births \cite{Mozaffarian2016}. 
In China, the incidence of CHD ranges from 8.98 to 11.1 per 1000 live births \cite{Zhao2019} \cite{Qu2016}. 
A pair of studies from Iran report incidences of 8.6 and 12.3 per 1000 live births, though the studies note that they were performed in different geographical locations with different populations within the country \cite{Nikyar2011} \cite{Rahim2008}.
One report from Dharan reports an incidence of 5.8 per 1000 patients admitted to a tertiary care hospital over a 12 month period \cite{Shah2008}. A study of newborns at one hospital in New Delhi, India claims an incidence of 3.9 per 1000 live births, though this rate may be a poor estimate as there is a significant delay between patient birth and referral to a cardiac center in India \cite{Khalil1994} \cite{Saxena2005}.

These incidence rates should be analyzed with some caution. In many cases, the reported rates were based on medical records. Medical records are not always correct. Additionally, the only way for a person to have a medical record is for him to go to a medical center. Not everyone who has CHD is able to seek medical help, often because of their geographical locations or their income. Even if a patient is able to seek medical help, the availability of proper cardiac care varies between and within countries. 

As screening tools become more effective and more widespread, it is expected that incidence rates will increase as defects are detected earlier. Generally, the earlier a defect is detected, the earlier it can be treated. Early detection and treatment means more CHD patients will live to adulthood. Currently, Webb et al. estimate that at least 12 to 34 million adults have CHD, and this number is expected to increase \cite{Webb2015}.

It is important to note that each defect type has a different prevalence, a different treatment plan, and different expected outcomes. A breakdown of prevalence rates of some of the most common lesion types can be seen in Figure \ref{ch5:fig:usa-defects-prev}. % See (16)
Once a patient is diagnosed with one of these defects, the specific nature of his case must be clearly documented. The documentation of CHD using the International Classification of Diseases, Ninth Revision, Clinical Modification (ICD-9-CM) has 25 high level codes representing various presentations of CHD, but these codes used alone are often not sufficient for describing a patient's true condition \cite{Mozaffarian2016}. Additional ICD-9-CM codes should be used to communicate the finer details of a patient's condition. 

\begin{figure}
\centering
\includegraphics[width=0.7\textwidth]{5/CHD-burden-webb.png}
\caption{Estimated CHD burden in World Health Organization (WHO) regions using incidence rates of approximately 12/1000 and 4/1000 in children and adults, respectively \cite{Webb2015}.}
\label{ch5:fig:CHD-burden}

\end{figure}

%Depending on the cause of the 
The financial burden of CHD varies depending on the defect. Certain defects require complex, expensive surgical repairs while others can be treated with less expensive approaches \cite{Mozaffarian2016}. The burden of CHD across the globe was outlined by Webb et al. Their figure illustrating the prevalence of CHD and the availability of funds with which to treat it can be see in Figure \ref{ch5:fig:CHD-burden}. As the overall mortality of CHD declines, the burden of CHD is expected to increase \cite{Mozaffarian2016}.

% Complications and risks
Unfortunately, the cost of treating CHD alone is not the only burden a patient must undergo. Patients with CHD are also at increased risk for heart failure and infections \cite{Mozaffarian2016}. Children with CHD are at 19-fold risk for stroke compared to their healthy counterparts \cite{Fox2015}. In a study of Swedish citizens born between 1970 and 1993, Giang et al performed a study compared the prevalence of cardiac conditions in patients with and without CHD \cite{Giang2018}. They found that patients who had a CHD diagnosis were at about eight times higher risk for intracerebral hemorrhage and subarachnoid hemorrhage than their non-CHD counterparts. The CHD patients were also more likely to suffer from arrhythmia and heart failure.

% When are patients diagnosed?
% Expected lifespan
% Treatment plan
% Financial burden


\subsection{Potential Causes of Neurodevelopmental Complications}

Early research in this area focuses on the neurodevelopmental status of neonatal patients pre- and post-surgical intervention. One theory was that some factor or factors in the surgical intervention caused brain injuries in the patients. This idea proved to be inaccurate when researchers began detecting neurological malformations \textit{in utero}.

In a systematic review of available literature regarding prenatal and postnatal presurgical CHD cases and neurodevelopmental outcomes, Mebius et al. identify two theories about the causality of  neurodevelopmental delays and CHD \cite{Mebius2017}. The first theory is that abnormalities in the cardiac system prevent the developing brain from receiving enough oxygen and nutrients, which disrupts prenatal brain development. The second theory is that faulty genetic pathways used during both cardiac and brain development cause both conditions to co-occur. However, 11 articles Mebius et al. found during their review that are related to bloodflow through the umbilical artery suggest a third theory. During the prenatal period, a fetus receives oxygen from the mother via the placenta. If the placenta was not functioning correctly, it could lead to the fetus receiving not enough oxygen. Lower quantities of oxygen throughout prenatal development could potentially cause problems both in brain and cardiac growth. The 11 articles have contradictory results, but some researchers are currently investigating the role of the placenta in CHD and prenatal brain development.

%\subsection{Common Disorder Combinations}

\subsection{Aging}

Survival of CHD patients to adulthood has increased from 10\% to 90\% over the last several decades. The impact of the combination of CHD and neurological conditions throughout a patient's lifetime is starting to be explored. The aging of the CHD population has also sparked interest in the relationships between CHD and adult-stage neurological disorders such as dementia and Alzheimer's. 

%To address:
%\begin{itemize}
%\item Common combinations
%\item Joint treatment?
%\item Additional risks?
%\item Joint financial and emotional burden on caretakers? 
%\item CHD, neuro, and aging? Dementia/Alzheimer's? Recent data ... MIND neuroimaging ancillary r01 dec 17
%\end{itemize}

\subsection{Identifying Neurocognitive Disorders}

\subsubsection{Patient Surveys}

Surveys known to be used for studying the relationship between CHD and neurodevelopment are

\begin{itemize}
\item National Institute of Health Toolbox (3 - 85 years): ``Performance tests of cognitive, motor, and sensory function and self-reported measures of emotional function for adults and children in the general population and those living with a chronic condition''.

\item Sue Beers (4 - 18 years [not inclusive of 18 years]): WASI-II, NEPSY-2, WRAML-2, D-KEFS, WISC-IV, Grooved Pegboard, BRIEF, Beery-Buktenica VMI, ASRS, Conners-3, BASC-II, ABAS-II, PedsQL General, PedsQL Cardiac, Pictoral Scale Self Perception Profile.

\item SVR-III NDT (9 - 13 years [not inclusive of 13 years]): WIAT, NEPSY, WRAML, D-KEFS, WISC-V, Grooved Pegboard, BRIEF, Beery-Buktenica VMI, ASRS, Conners ADHD Index, BASC-II, ABAS-3, PedsQL General, PedsQL Cardiac

\item Bayley Scales of Infant and Toddler Development -III (1 - 24 months): Subtests include cognitive, language, social-emotional, motor, and adaptive behavior tests \cite{Mebius2017}.

\item Battelle Developmental Inventory (Birth - 8 years [not inclusive of 8 years]): Subsets include cognition, communication, social-emotional development, physical development, and adaptive behavior.

\item Developmental Assessment of Young Children (Birth - 6 years [not inclusive of 6 years]): Subtests include cognition, communication, social-emotional development, physical development, and adaptive behavior.

\item Preschool Language Scale + Receptive-Expressive Emergent Language (Birth - 3 years): Total language, auditory comprehension, expressive communication, articulation, receptive language, expressive language, and inventory of vocabulary words.

\item Peabody Developmental Motor Scales (Birth - 5 years): Subtests include reflexes, stationary, locomotion, object manipulation, grasping, visual-motor integration
\end{itemize}

The goal of these surveys is to compare the patient's cognitive function and neurological functions to expected milestones. Certain deviations from certain milestones are indicative of different disorders.

\subsubsection{Neurological Images}

When an area of the brain is active, it uses more oxygen than the surrounding regions. Functional MRIs (fMRI) are sensitive to signals emitted by deoxygenated hemoglobin. The blood oxygen level dependent (BOLD) signal recorded by the fMRI reveal regions of the brain which are active at the same time. These combinations of regions are called neuronal networks. 

Many neuronal networks exist, but most of them are considered to be task related. In 2001, Raichle et al. suggested the existence of a neuronal network which operated when a person is at rest \cite{Raichle2001}. Their theory was confirmed by Greicius et al. in 2003 \cite{Greicius2003}. Because the patient is not performing a specific task when they are in a resting state, the resting-state networks have the potential to reveal valuable information about a patient's neurodevelopmental status.

A fMRI taken of a patient in a resting, task-free state, is called a resting-state fMRI (rs-fMRI). rs-fMRIs are sequences of image volumes acquired over a period of a few minutes. The image volumes themselves have relatively low spatial resolution when compared to structural MRIs, but their temporal resolution is significantly higher as a new volume is acquired every two to three seconds. 

The BOLD signals in rs-fMRI image sequences are analyzed using a process called functional connectivity analysis. Functional connectivity analysis identifies patterns and networks of brain activity. Some functional connectivity analysis studies have lead to the discoveries of links between specific disruptions in these naturally occurring networks and neurodevelopmental diseases such as autism and attention deficit hyperactivity disorder \cite{Assaf2010} \cite{Zang2007}. With further refinements of both acquisition techniques and characterization of these functional networks, clinicians may be able to use rs-fMRI to evaluate the neurodevelopmental status of CHD patients and to identify patients who may benefit from certain therapies or neuroprotective interventions.

\subsection{CHD Subjects}

\subsubsection{Neonatal Subject Population and Images}

Neonatal subjects are recruited as part of a prospective observational study. The subjects were scanned using a 3T Skyra (Siemans AG, Erlangen, Germany). They were unsedated during the scans and a ``feed and bundle'' protocol was used to prevent motion during the scans \cite{Windram2011}. The newborns were positioned in the coil to minimize head tilting. Newborns were fitted with earplugs (Quiet Earplugs; Sperian Hearing Protection, San Diego, CA) and neonatal ear muffs (MiniMuffs; Natus, San Carlos, CA). An MR-compatible vital signs monitoring system (Veris, MEDRAD, Inc. Indianola, PA) was used to monitor neonatal vital signs. All scans were performed using a multi-channel head coil. The parameters for the resting-state BOLD MR scans were FOV=240 mm and TE/TR=32/2020 ms with interplane resolution of 4x4 mm, slice thickness of 4 mm, and 4 mm space between slices. The acquired images contained 150 volumes where each volume consisted of 64x64x32 voxels$^3$.

\subsubsection{Preadolescent Subject Population and Images}

As part of a multicenter study of CHD in preadolescents, we collected rs-fMRIs from nine sites throughout the United States. These images were of patients in the age range of 9 to 13 years who either had CHD or were healthy with no neurocognitive impairments. In addition to the MRI scans, subjects who participated in this study were asked to participate in additional testing  either to determine their neurocognitive outcome status or to perform genetic analyses. % (GET DETAILS FROM NANCY).

%\begin{itemize}
%\item How were the images gathered?
%\item How were the patients recruited?
%\item What are the imaging protocol details?
%\item What other information was collected?
%\end{itemize}

\subsubsection{Fetal Subject Population and Images}

%Real goal is to develop a method of registering fetal brain and placental images so that we can further examine the relationship between placental oxygen levels and fetal brain development. Longitudinally, this technique can be used to determine how placental oxygen flow and fetal brain development impact a patient over the course of his or her life. Once the relationship between the placenta and fetal brain development is better understood, we can determine a set of neuroprotective interventions to employ for at-risk patients before they are born.

Fetal subjects have different constraints on their physical environment than neonates, preadolescents, and adults. As a result, they exhibit unique patterns of motion. The previous subject cohorts discussed in this chapter have the following commonalities: the subject experiences the full effects of gravity, the subject is lying on his back in an MRI scanner, and the subject's head motion is limited by the head coil within the MRI. Any motion in these images is a direct result of the subject himself moving, whether passively (cardiac motion and breathing) or actively (fidgeting or looking around).

A fetal subject is scanned in vivo. He is suspended in amniotic fluid within his mother. The amniotic fluid has buoyancy that reduces the effects of gravity and allows a fetal subject significant freedom of movement. The fetus can rotate, shift, and flip in ways that can only be accomplished when floating in a body of water. The properties of the uterus constrain the physical space in which a motion could occur, but not as much as the head coil and gravity do to the other patient cohorts. A fetus is not guaranteed to be in any specific position at the start of the scan: the scan begins when the mother is ready, not when the fetus achieves a certain pose. 

The fetal subjects underwent fetal echocardiography scans in a cardiac clinic to determine whether they were healthy or had a form of CHD. They were then scanned on an MRI scanner. Images of the fetal brain and the placenta were acquired for each subject. 

We are interested in both the fetal brain and placental images for our work because of the relationship between placenta and brain development. However, these organs have very different physical properties. The fetal brain is a rigid structure floating and moving within the amneotic fluid. It undergoes translation and rotation as a single unit due to passive and active maternal and fetal motions. The placenta, on the other hand, is anchored in place on the uterine wall. It may undergo small translations or rotations due to maternal motion, but it will respond differently to fetal motion. Fetal motions cause nonlinear deformations of the pliable placenta that can only be adequately accounted for using nonlinear registration algorithms. Nonlinear registrations have the potential to deform brain images into physically impossible shapes, so the fetal brain and placenta were manually segmented in their respective images so that each organ could undergo independent motion correction. 

The segmenters were one of a group of four researchers. While one researcher trained the other three group members, the interrater agreement between them is still being determined.

%\begin{itemize}
%\item Fetal patients scanned between XX and XX weeks gestational age. 
%\item Imaging protocol details?
%\item What other information is collected about fetus and/or mom?
%\end{itemize}

\section{Aging Brain Subjects}

The final data set we use is from the Alzheimer's Disease Neuroimaging Initiative (ADNI) database (adni.loni.usc.edu). The ADNI was launched in 2003 as a public-private partnership, led by Principal Investigator Michael W. Weiner, MD. The primary goal of ADNI has been to test whether serial magnetic resonance imaging (MRI), positron emission tomography (PET), other biological markers, and clinical and neuropsychological assessment can be combined to measure the progression of mild cognitive impairment (MCI) and early Alzheimer's disease (AD). For up-to-date information, see www.adni-info.org.



%\begin{itemize}
%\item How were the images gathered?
%\item How were the patients recruited?
%\item What are the imaging protocol details?
%\item What other information was collected?
%\end{itemize}

%The second adult cohort comes from the Alzheimer's Disease Neuroimaging Initiative (ADNI) dataset. The ADNI study has been working since 2004 to further Alzheimer's research by gathering, analyzing, and sharing clinical, imaging, genetic, and biochemical biomarkers from the elderly population. The group gathers data from 63 sites in the United States and Canada. During the second phase of the study, sites who have a Philips MRI system gathered resting-state fMRIs from their subjects. This data is freely available to academic researchers through the LONI Image and Data Archive.

\section{Purpose for each cohort}

\subsection{Simulated Phantom}

The phantom experiments will be used to probe the volume registration technique. By applying the DAG-based and traditional registration techniques to the base phantom sequence, we will be able to evaluate the degrees of positional and signal change errors each technique may introduce into the registration process. After determining the baseline error, we will apply both registration techniques to the BOLD phantom sequence. The registered versions of the BOLD phantom sequence will be compared to each other and to the original BOLD phantom sequence to determine how well each registration retains the BOLD signal.

This particular experiment will be one of the first to investigate how much true BOLD signal is preserved through motion correction. One of the major drawbacks to existing motion correction pipelines is that they remove signal along with noise. In clinical data, there is no way to know the ground truth signal contained within the image; however, simulated phantom images have a de facto known ground truth signal. The design for this experiment can be used to evaluate how much BOLD signal is recovered by other motion correction pipelines, and how close the recovered signal is to the signal of interest.

\subsection{Clinical Images}
%For each subject in a cohort of 74 healthy neonatal subjects, at least one rs-fMRI was acquired. The images underwent motion correction using a pipeline developed by Power et al. \cite{Power2014}. The motion corrected image were compared to Power et al.'s thresholds of acceptability for their FD and DVARS metrics \cite{Power2014}. Of the 74 subjects in this cohort, 17 subjects had rs-fMRIs which did not meet Power et al.'s usability criteria. These high motion images were used in our study.

\textbf{Neonatal Cohort.} Our set of neonatal subjects includes a cohort of 74 healthy neonates. Each subject in this cohort underwent an MRI scan, and the rs-fMRIs obtained during this process were compared to Power et al.'s positional and signal change usability thresholds. Of the 74 subjects, 17 of them had rs-fMRIs which did not meet the usability criteria. These high motion images were used to test the feasibility of the DAG-based volume registration framework. 

These images were ideal for the feasibility study for three reasons. First, the neonates were healthy, which eliminates disease status as a confounding variable in the analysis of the registered images. Second, the neonates in this study were scanned using a feed and sleep protocol. Because the neonates were asleep during the scan, they generally did not move very much. The high-motion neonates are an obvious exception to this concept, but many of the high-motion images contained long periods where the subject was stationary. Evaluating the DAG-based framework on data with various patterns of motion and different periods of low and high motion allowed us to explore the effects of the DAG-based algorithm in different combinations of motion features. Third, these images were too corrupted by motion to be used in other analyses. Applying both the DAG-based framework and the traditional registration framework to these images provided the opportunity to compare the performances of both registration frameworks to each other in the context of the usability gold standard thresholds. 

\textbf{Preadolescent Cohort.} The multicenter imaging study of preadolescent subjects provides a unique opportunity to evaluate the efficacy of the DAG-based framework on a large subject cohort containing variable amounts of motion. The outcome of this experiment will be used in the next experiment to determine if there are any site-specific or vendor-specific variables influencing patient motion.

\textbf{Adult Cohort.} The adult cohort encompass many clinical outcomes and a wider age range than the other clinical populations. 

\textbf{Fetal Cohort.} As the fetal subjects have both neurological and placental images, their data will be used to examine the impact of volume registration on different organ types.

\section{Summary}


% FIX CONTENTS
\chapter{RESULTS}
\label{ch:results}

This chapter is divided into two main sections. The first section focuses on the comparison of the two motion correction techniques. The second section focuses on the results of the machine learning algorithms applied to the metrics extracted from the images.

\section{Comparison of Volume Registration Methods}

\subsection{Overview}

Each of the clinical images underwent volume registration using both registration methods outlined in Chapter \ref{moco}. The FD and DVARs metrics were calculated for every pair of subsequent volumes $i$ and $i+1$ in the original sequences, the traditionally registered sequences, and the DAG-registered sequences. Then the sequences were comprehensively compared to themselves. For every volume in each sequence, the Dice metric, the mutual information, and the correlation ratio were calculated for the volume and every other volume in the sequence.

The simulated data underwent the same analyses as the clinical images, with one addition. 

\subsubsection{Preadolescent Cohort}

% First: FD and DVARs
Histograms of the distributions of the FD and DVARs metrics forthe original, traditionally registered, and DAG-registered images can be seen in Figure REF.

The FD and DVARS values also considered to be distribution functions representing the effects of no registration, traditional registration, and DAG-based registration. These distributions were compared using the Kolmogorov-Smirnov test, which compares the empirical distribution functions of two samples. There were statistically significant differences between the FD and DVARS values of all sequences at $p < 2.2*10^{-16}$. Statistics calculated for the FD and DVARS value histograms of both motion correction methods can be seen in Table \ref{tab:hists}.

%Each rs-fMRI sequence in the cohort underwent registration using both frameworks. For each sequence, the correlation ratio between every possible pair of volumes was calculated. A set of metrics of the correlation ratio matrices for each sequence can be seen in Table \ref{tab:crm-stats}. This table shows that the original sequences generally have higher average correlation ratios and contain more variation in their correlation ratios than the globally registered images. The registration methods were able to reduce the mean and variability of the correlation ratios across all subjects in the cohort who had original correlation ratio averages of at least 0.035.

\begin{table}[th]
\centering
\caption{The number of frames recovered by each global volume registration framework for each threshold.}
\label{tab:thresholds}
\begin{tabular}{|l|r|r|r|}
\hline
\textbf{Threshold} & \textbf{None} & \textbf{Traditional} & \textbf{DAG-based} \\ \hline
FD (0.2 mm)        & 966           & 175                  & 569                \\ \hline
DVARS (25 units)   & 781           & 78                   & 297                \\ \hline
Both               & 619           & 61                   & 258                \\ \hline
Both (\%)          & 24.27\%       & 2.39\%               & 10.11\%            \\ \hline
\end{tabular}
\end{table}


The FD and DVARS values were compared to the usability thresholds. 

Power et al.'s usability thresholds were used to determine how many volumes were recovered by each framework \cite{Power2014}. Table \ref{tab:thresholds} shows the number of volumes meeting each threshold, with the traditional and DAG-based frameworks recovering 2\% and 10\% of volumes, respectively. These results show that the DAG-based registration technique produces sequences with lower FD and DVARS value than the traditional global registration method does.

\section{Motion Patterns}

\chapter{DISCUSSION}
\label{ch:discussion}

\section{Comparison of Volume Registration Methods}

Resting-state BOLD MR images are used to evaluate the functional architecture of a patient's brain. Because resting-state BOLD images are highly susceptible to motion, development of strong post-acquisition motion correction techniques is vital. Current pipelines for mitigating motion after sequence acquisition vary in terms of efficacy and effectiveness, but all begin with global volume registration. In this study, we compared the corrective performance of two global volume registration methods, the traditional framework and a novel DAG-based framework, on a set of 17 neonatal rs-fMRIs. 

The correlation ratio matrices, FD, and DVARS values were calculated for each sequence. The decrease in the mean and standard deviations of the correlation ratio matrices for the registered sequences indicate that global volume registration reduces some effects of motion in rs-fMRIs. The histograms of the FD and DVARS values in the registered sequences show that the DAG-based method was better able to correct volumes to meet Power et al’s thresholds than the traditional registration method. These results indicate that the DAG-based global registration method is better able to reduce the effects of motion than the traditional global registration method when correcting motion in neonatal images. While no entire sequences were recovered, some high-motion volumes within each sequence were recovered by the DAG-based registration method that were not recovered by the traditional registration method. 

\subsection{Relation to Existing Work}
To the best of our knowledge, the only other study that has used a variant of the DAG-based method was performed by Liao et al \cite{Liao2016}. Liao et al’s dataset consisted of 10 fetal rs-fMRIs. In each of these sequences, the fetal brain, fetal liver, and placenta were manually segmented in the first volume of the sequence as well as in five other randomly chosen volumes. These overlap of these manual segmentations before and after registration as measured using the Dice coefficient was used to quantify the amount of motion in each sequence. Even though the Dice coefficients increase more in each sequence after Liao et al.’s registration than after traditional registration, their measure of positional change fails to quantify any changes in position between any other pairs of volumes that do not have manual segmentations. 


\section{Aim 2: Describing Motion}

Satterthwaite et al. note that motion is often correlated with patient age in adolescent populations and specifically designed a study of adolescents ages 8-23 such that patient age and motion were uncorrelated (Satterthwaite 2012a?).

\section{Limitations}

\textbf{Fetal Scans.} The fetal scans were manually segmented, the masks used in the segmentation were created to be uniform across the whole sequence. The intention behind this process is to remove all voxel values not associated with the organ of interest. However, fetal motion is highly variable. It is possible for a subject to rotate in any direction. The subject may drastically change position in the middle of the scan, possibly several times. The masks created by the annotators were created using  SOFTWARE, which allows masks to be applied to an entire sequence. The masks were forced to be created to ensure the fetal brain or placenta would be inside the masked area at all times. The masks may cover some area that does not belong to the organ of interest.

This limitation is unique to the fetal images. Existing pre-processing pipelines exist for skull stripping for neonatal and preadolescent images. Development of a similar pipeline for fetal images, while a challenge, would make research surrounding fetal rs-fMRIs more accessible to the medical imaging community.

\textbf{Registration fixes positional effects of motion, not spin history or susceptibility effects}
Subject motion during rs-MRI scans affects both the recorded position and orientation of the subject as well as the established magnetic spin gradients within the skull. The DAG-based technique can correct the positional effects of motion, but it cannot correct the effects of the motion that disrupt the magnetic spin gradients. Methods for prospectively estimating subject motion exist and can be used to change slice positions in each volume during acquisition. Retrospective techniques to correct for this effect will require shot-to-shot modeling of macroscopic $B_0$ fields and are beyond the scope of the present research.

\chapter{Conclusions}
This is the second chapter of the present dissertation. It is more interesting than the first one, for it is the last one.

\section{Limitations}

\section{Future Work}

\subsection{Adult Subject Population and Images}

As the prognosis for patients with CHD improves, their life expectancy also increases. The aging CHD population presents new questions about the connection between CHD and neurocognitive challenges associated with aging. As patients age, there is an expectation that their images will contain less motion for a time. If a patient begins to show signs of cognitive impairment due to aging, it can be expected that their images will begin to contain more motion as their neurocognitive state deteriorates. 

We include a cohort of adult subjects over a wide range of ages in our study. The purpose of using images from this cohort is to demonstrate the generalizability of the DAG-based framework to adult patients as well as its use in different clinical populations. This cohort is being studied as part of an ongoing, prospective study of CHD and neurodevelopment. The data collected for these subject includes rs-fMRIs, behavioral, and clinical data from healthy and CHD adult subjects. 


%
\appendix
\chapter{Volume Registration Parameters}
\label{appendix:registration-params}

The parameters used for the registration of pairs of image volumes can be seen below.

\begin{lstlisting}[language=Python]

##
# Register a pair of image volumes
#
# Effects: save a copy of the registered image and the registration parameters
#
# @param fixedImgFn The filename of the fixed image as a string
# @param movinImgFn The filename of the moving image as a string
# @param regImgOutFn The filename as a string specifying where to save the registered moving image
# @param transformPrefix 
# @param initialize Optional parameter to specify the location of the transform matrix from the previous registration
# @param regType Optional parameter to specify the type of registration to use (affine ['Affine'] or nonlinear ['Syn']) Default: nonlinear
def registerVolumes(fixedImgFn, movinImgFn, regImgOutFn, transformPrefix, initialize=None, regtype='nonlinear'):
    # Registration set up: for both Affine and SyN transforms
    reg = Registration()
    reg.inputs.fixed_image = fixedImgFn
    reg.inputs.moving_image = movinImgFn
    reg.inputs.output_transform_prefix = transformPrefix 
    reg.inputs.interpolation = 'NearestNeighbor'
    reg.inputs.dimension = 3
    reg.inputs.write_composite_transform = False 
    reg.inputs.collapse_output_transforms = False
    reg.inputs.initialize_transforms_per_stage = False
    reg.inputs.num_threads = 100
    reg.inputs.output_warped_image = regImgOutFn

    # Registration set up: Specify certain parameters for the Affine registration step
    reg.inputs.transforms = ['Affine']
    reg.inputs.transform_parameters = [(2.0,)]
    reg.inputs.number_of_iterations = [[1500, 200]] 
    reg.inputs.metric = ['CC'] 
    reg.inputs.metric_weight = [1]
    reg.inputs.radius_or_number_of_bins = [5] 
    reg.inputs.convergence_threshold = [1.e-8]
    reg.inputs.convergence_window_size = [20]
    reg.inputs.smoothing_sigmas = [[1,0]]
    reg.inputs.sigma_units = ['vox']
    reg.inputs.shrink_factors = [[2,1]]
    reg.inputs.use_estimate_learning_rate_once = [True]
    reg.inputs.use_histogram_matching = [True] # This is the default, but specify it anyway

    # Registration set up: nonlinear transforms only
    if regtype == 'nonlinear':
        reg.inputs.transforms.append('SyN')
        reg.inputs.transform_parameters.append((0.25, 3.0, 0.0))
        reg.inputs.number_of_iterations.append([100, 50, 30])
        reg.inputs.metric.append('CC')
        reg.inputs.metric_weight.append(1)
        reg.inputs.radius_or_number_of_bins.append(5)
        reg.inputs.convergence_threshold.append(1.e-9)
        reg.inputs.convergence_window_size.append(20)
        reg.inputs.smoothing_sigmas.append([2,1,0])
        reg.inputs.sigma_units.append('vox')
        reg.inputs.shrink_factors.append([3,2,1])
        reg.inputs.use_estimate_learning_rate_once.append(True)
        reg.inputs.use_histogram_matching.append(True) # This is the default value, but specify it anyway

    # If the registration is initialized, set a few more parameters
    if initialize is not None:
        reg.inputs.initial_moving_transform = initialize
        reg.inputs.invert_initial_moving_transform = False

    # Keep the user updated with the status of the registration
    print("Starting", regtype, "registration for", regImgOutFn)
    
    # Run the registration
    reg.run()
    
    # Keep the user updated with the status of the registration
    print("Finished", regtype, "registration for", regImgOutFn)

\end{lstlisting}
\chapter{Comprehensive Selection of Results Tables and Figures}
\label{appendix:results}

\section{Simulated Data}

\subsection{Volume Registration: Power Thresholds}

\begin{table}[]
\centering
\caption{Results from the t-tests comparing the counts for the numbers of images meeting the FD, DVARS, and FD and DVARS thresholds for sequence type $S_1$ and sequence type $S_2$.}
\label{tab:spectr-power-ttest}
\begin{tabular}{|c|c|c|c|}
\hline
\textbf{Sequence Type 1 ($S_1$)} &
  \textbf{Original} &
  \textbf{Original} &
  \textbf{\begin{tabular}[c]{@{}c@{}}Traditionally \\ Registered\end{tabular}} \\ \hline
\textbf{Sequence Type 2 ($S_2$)} &
  \textbf{\begin{tabular}[c]{@{}c@{}}Traditionally\\ Registered\end{tabular}} &
  \textbf{\begin{tabular}[c]{@{}c@{}}DAG\\ Registered\end{tabular}} &
  \textbf{\begin{tabular}[c]{@{}c@{}}DAG\\ Registered\end{tabular}} \\ \hline
\begin{tabular}[c]{@{}c@{}}P($S_1$ and $S_2$ have \\ same FD counts)\end{tabular} &
  1.05 E -16 &
  4.49 E -11 &
  0.127 \\ \hline
\begin{tabular}[c]{@{}c@{}}P($S_1$ and $S_2$ have \\ same DVARS counts)\end{tabular} &
  0.941 &
  0.941 &
  1.0 \\ \hline
\begin{tabular}[c]{@{}c@{}}P($S_1$ and $S_2$ have \\ same FD and DVARS counts)\end{tabular} &
  0.590 &
  0.486 &
  0.872 \\ \hline
\end{tabular}
\end{table}

\begin{table}[]
\centering
\caption{The number of subjects whose sequences of types $S_1$ and $S_2$ had different FD distributions.}
\label{tab:spectr-fd-kstest}
\begin{tabular}{|c|c|c|c|}
\hline
\textbf{\begin{tabular}[c]{@{}c@{}}\# Sequences \\ Type 1 ($S_1$)\end{tabular}} &
  \textbf{\begin{tabular}[c]{@{}c@{}}\# Sequences \\ Type 2 ($S_2$)\end{tabular}} &
  \textbf{\begin{tabular}[c]{@{}c@{}}\# Sequences \\ p \textless 0.05\end{tabular}} &
  \textbf{\begin{tabular}[c]{@{}c@{}}\# Sequences \\ p \textless 0.005\end{tabular}} \\ \hline
Original                                                            & \begin{tabular}[c]{@{}c@{}}Traditionally\\ Registered\end{tabular} & 90 & 90 \\ \hline
Original                                                            & \begin{tabular}[c]{@{}c@{}}DAG\\ Registered\end{tabular}           & 90 & 90 \\ \hline
\begin{tabular}[c]{@{}c@{}}Traditionally \\ Registered\end{tabular} & \begin{tabular}[c]{@{}c@{}}DAG\\ Registered\end{tabular}           & 40 & 27 \\ \hline
\end{tabular}
\end{table}

\begin{table}[]
\centering
\caption{The number of subjects whose sequences of types $S_1$ and $S_2$ had different DVARS distributions.}
\label{tab:spectr-dvars-kstest}
\begin{tabular}{|c|c|c|c|}
\hline
\textbf{\begin{tabular}[c]{@{}c@{}}\# Sequences \\ Type 1 ($S_1$)\end{tabular}} &
  \textbf{\begin{tabular}[c]{@{}c@{}}\# Sequences \\ Type 2 ($S_2$)\end{tabular}} &
  \textbf{\begin{tabular}[c]{@{}c@{}}\# Sequences \\ p \textless 0.05\end{tabular}} &
  \textbf{\begin{tabular}[c]{@{}c@{}}\# Sequences \\ p \textless 0.005\end{tabular}} \\ \hline
Original                                                            & \begin{tabular}[c]{@{}c@{}}Traditionally\\ Registered\end{tabular} & 90 & 90 \\ \hline
Original                                                            & \begin{tabular}[c]{@{}c@{}}DAG\\ Registered\end{tabular}           & 90 & 90 \\ \hline
\begin{tabular}[c]{@{}c@{}}Traditionally \\ Registered\end{tabular} & \begin{tabular}[c]{@{}c@{}}DAG\\ Registered\end{tabular}           & 3  & 0  \\ \hline
\end{tabular}
\end{table}



\subsection{Volume Registration: Sequence Duration Motion}

\begin{table}[]
\centering
\caption{Results of t-tests comparing the descriptive statistics of the correlation ratio matrices for the simulated data.}
\label{tab:spectr-cr-ttest}
\begin{tabular}{|c|c|c|c|}
\hline
\textbf{Sequence Type 1 ($S_1$)} &
  \textbf{Original} &
  \textbf{Original} &
  \textbf{\begin{tabular}[c]{@{}c@{}}Traditionally \\ Registered\end{tabular}} \\ \hline
\textbf{Sequence Type 2 ($S_2$)} &
  \textbf{\begin{tabular}[c]{@{}c@{}}Traditionally\\ Registered\end{tabular}} &
  \textbf{\begin{tabular}[c]{@{}c@{}}DAG\\ Registered\end{tabular}} &
  \textbf{\begin{tabular}[c]{@{}c@{}}DAG\\ Registered\end{tabular}} \\ \hline
\begin{tabular}[c]{@{}c@{}}P($S_1$ and $S_2$ \\ have same minimums)\end{tabular} &
  0.3487 &
  0.3407 &
  0.9821 \\ \hline
\begin{tabular}[c]{@{}c@{}}P($S_1$ and $S_2$ \\ have same 1st quartile)\end{tabular} &
  9.750 E -113 &
  1.246 E -112 &
  0.8019 \\ \hline
\begin{tabular}[c]{@{}c@{}}P($S_1$ and $S_2$ \\ have same medians)\end{tabular} &
  5.288 E -88 &
  5.409 E -88 &
  0.9997 \\ \hline
\begin{tabular}[c]{@{}c@{}}P($S_1$ and $S_2$ \\ have same 3rd quartiles)\end{tabular} &
  6.534 E -81 &
  6.730 E -81 &
  0.9577 \\ \hline
\begin{tabular}[c]{@{}c@{}}P($S_1$ and $S_2$ \\ have same maximums)\end{tabular} &
  2.536 E -98 &
  6.180 E -103 &
  0.4068 \\ \hline
\end{tabular}
\end{table}

\begin{table}[]
\centering
\caption{Results of t-tests comparing the descriptive statistics of the Dice matrices for the simulated data.}
\label{tab:spectr-dice-ttest}
\begin{tabular}{|c|c|c|c|}
\hline
\textbf{Sequence Type 1 ($S_1$)} &
  \textbf{Original} &
  \textbf{Original} &
  \textbf{\begin{tabular}[c]{@{}c@{}}Traditionally \\ Registered\end{tabular}} \\ \hline
\textbf{Sequence Type 2 ($S_2$)} &
  \textbf{\begin{tabular}[c]{@{}c@{}}Traditionally\\ Registered\end{tabular}} &
  \textbf{\begin{tabular}[c]{@{}c@{}}DAG\\ Registered\end{tabular}} &
  \textbf{\begin{tabular}[c]{@{}c@{}}DAG\\ Registered\end{tabular}} \\ \hline
\begin{tabular}[c]{@{}c@{}}P($S_1$ and $S_2$ \\ have same minimums)\end{tabular} &
  9.976 E -105 &
  2.520 E -110 &
  0.3778 \\ \hline
\begin{tabular}[c]{@{}c@{}}P($S_1$ and $S_2$ \\ have same 1st quartile)\end{tabular} &
  5.225 E -93 &
  5.582 E -93 &
  0.931 \\ \hline
\begin{tabular}[c]{@{}c@{}}P($S_1$ and $S_2$ \\ have same medians)\end{tabular} &
  1.988 E -104 &
  2.158 E -104 &
  0.9578 \\ \hline
\begin{tabular}[c]{@{}c@{}}P($S_1$ and $S_2$ \\ have same 3rd quartiles)\end{tabular} &
  1.679 E -131 &
  2.190 E -131 &
  0.842 \\ \hline
\begin{tabular}[c]{@{}c@{}}P($S_1$ and $S_2$ \\ have same maximums)\end{tabular} &
  1.0 &
  1.0 &
  1.0 \\ \hline
\end{tabular}
\end{table}

\clearpage

\begin{table}[]
\centering
\caption{Results of t-tests comparing the descriptive statistics of the MI matrices for the simulated data.}
\label{tab:spectr-mi-ttest}
\begin{tabular}{|c|c|c|c|}
\hline
\textbf{Sequence Type 1 ($S_1$)} &
  \textbf{Original} &
  \textbf{Original} &
  \textbf{\begin{tabular}[c]{@{}c@{}}Traditionally \\ Registered\end{tabular}} \\ \hline
\textbf{Sequence Type 2 ($S_2$)} &
  \textbf{\begin{tabular}[c]{@{}c@{}}Traditionally\\ Registered\end{tabular}} &
  \textbf{\begin{tabular}[c]{@{}c@{}}DAG\\ Registered\end{tabular}} &
  \textbf{\begin{tabular}[c]{@{}c@{}}DAG\\ Registered\end{tabular}} \\ \hline
\begin{tabular}[c]{@{}c@{}}P($S_1$ and $S_2$ \\ have same minimums)\end{tabular} &
  5.016 E -114 &
  6.328 E -126 &
  0.5397 \\ \hline
\begin{tabular}[c]{@{}c@{}}P($S_1$ and $S_2$ \\ have same 1st quartile)\end{tabular} &
  4.68 E -105 &
  7.90 E -105 &
  0.995 \\ \hline
\begin{tabular}[c]{@{}c@{}}P($S_1$ and $S_2$ \\ have same medians)\end{tabular} &
  1.65 E -97 &
  3.57 E -97 &
  0.994 \\ \hline
\begin{tabular}[c]{@{}c@{}}P($S_1$ and $S_2$ \\ have same 3rd quartiles)\end{tabular} &
  1.065 E -84 &
  2.374 E -84 &
  0.974 \\ \hline
\begin{tabular}[c]{@{}c@{}}P($S_1$ and $S_2$ \\ have same maximums)\end{tabular} &
  0.00473 &
  0.00794 &
  0.8761 \\ \hline
\end{tabular}
\end{table}

\clearpage

\section{Preadolescent Cohort}

\subsection{Volume Registration: Power Thresholds}

\begin{table}[]
\centering
\caption{Results from the t-tests comparing the counts for the numbers of images meeting the FD, DVARS, and FD and DVARS thresholds for sequence type $S_1$ and sequence type $S_2$.}
\label{tab:pread-power-ttest}
\begin{tabular}{|c|c|c|c|}
\hline
\textbf{Sequence Type 1 ($S_1$)} &
  \textbf{Original} &
  \textbf{Original} &
  \textbf{\begin{tabular}[c]{@{}c@{}}Traditionally \\ Registered\end{tabular}} \\ \hline
\textbf{Sequence Type 2 ($S_2$)} &
  \textbf{\begin{tabular}[c]{@{}c@{}}Traditionally\\ Registered\end{tabular}} &
  \textbf{\begin{tabular}[c]{@{}c@{}}DAG\\ Registered\end{tabular}} &
  \textbf{\begin{tabular}[c]{@{}c@{}}DAG\\ Registered\end{tabular}} \\ \hline
\begin{tabular}[c]{@{}c@{}}P($S_1$ and $S_2$ have \\ same FD counts)\end{tabular} &
  2.81 E -16 &
  2.35 E -16 &
  0.998 \\ \hline
\begin{tabular}[c]{@{}c@{}}P($S_1$ and $S_2$ have \\ same DVARS counts)\end{tabular} &
  9.43 E -12 &
  5.30 E -12 &
  0.950 \\ \hline
\begin{tabular}[c]{@{}c@{}}P($S_1$ and $S_2$ have \\ same FD and DVARS counts)\end{tabular} &
  1.12 E -11 &
  5.60 E -12 &
  0.938 \\ \hline
\end{tabular}
\end{table}



%\begin{figure}[ht]
%\centering
%\includegraphics[width=1\textwidth]{6/figures/similarity-mat-sample.png}
%\caption{Examples of the three similarity matrices. Lighter colors represent more desirable values.}
%\label{fig:sim-mat-sample}
%\end{figure}

\clearpage

\subsection{Volume Registration: Sequence Duration Motion}

\begin{table}[]
\centering
\caption{Results of t-tests comparing the descriptive statistics of the Dice matrices for the preadolescent data.}
\label{tab:preads-dice-ttest}
\begin{tabular}{|c|c|c|c|}
\hline
\textbf{Sequence Type 1 ($S_1$)} &
  \textbf{Original} &
  \textbf{Original} &
  \textbf{\begin{tabular}[c]{@{}c@{}}Traditionally \\ Registered\end{tabular}} \\ \hline
\textbf{Sequence Type 2 ($S_2$)} &
  \textbf{\begin{tabular}[c]{@{}c@{}}Traditionally\\ Registered\end{tabular}} &
  \textbf{\begin{tabular}[c]{@{}c@{}}DAG\\ Registered\end{tabular}} &
  \textbf{\begin{tabular}[c]{@{}c@{}}DAG\\ Registered\end{tabular}} \\ \hline
\begin{tabular}[c]{@{}c@{}}P($S_1$ and $S_2$ \\ have same minimums)\end{tabular} &
  0.770 &
  0.695 &
  0.916 \\ \hline
\begin{tabular}[c]{@{}c@{}}P($S_1$ and $S_2$ \\ have same 1st quartile)\end{tabular} &
  0.976 &
  0.906 &
  0.880 \\ \hline
\begin{tabular}[c]{@{}c@{}}P($S_1$ and $S_2$ \\ have same medians)\end{tabular} &
  0.883 &
  0.562 &
  0.643 \\ \hline
\begin{tabular}[c]{@{}c@{}}P($S_1$ and $S_2$ \\ have same 3rd quartiles)\end{tabular} &
  0.000343 &
  0.000586 &
  0.390 \\ \hline
\begin{tabular}[c]{@{}c@{}}P($S_1$ and $S_2$ \\ have same maximums)\end{tabular} &
  1.0 &
  1.0 &
  1.0 \\ \hline
\end{tabular}
\end{table}

\clearpage

\begin{table}[]
\centering
\caption{Results of t-tests comparing the descriptive statistics of the MI matrices for the preadolescent data.}
\label{tab:preads-mi-ttest}
\begin{tabular}{|c|c|c|c|}
\hline
\textbf{Sequence Type 1 ($S_1$)} &
  \textbf{Original} &
  \textbf{Original} &
  \textbf{\begin{tabular}[c]{@{}c@{}}Traditionally \\ Registered\end{tabular}} \\ \hline
\textbf{Sequence Type 2 ($S_2$)} &
  \textbf{\begin{tabular}[c]{@{}c@{}}Traditionally\\ Registered\end{tabular}} &
  \textbf{\begin{tabular}[c]{@{}c@{}}DAG\\ Registered\end{tabular}} &
  \textbf{\begin{tabular}[c]{@{}c@{}}DAG\\ Registered\end{tabular}} \\ \hline
\begin{tabular}[c]{@{}c@{}}P($S_1$ and $S_2$ \\ have same minimums)\end{tabular} &
  0.624 &
  0.718 &
  0.896 \\ \hline
\begin{tabular}[c]{@{}c@{}}P($S_1$ and $S_2$ \\ have same 1st quartile)\end{tabular} &
  0.489 &
  0.497 &
  0.992 \\ \hline
\begin{tabular}[c]{@{}c@{}}P($S_1$ and $S_2$ \\ have same medians)\end{tabular} &
  0.364 &
  0.324 &
  0.928 \\ \hline
\begin{tabular}[c]{@{}c@{}}P($S_1$ and $S_2$ \\ have same 3rd quartiles)\end{tabular} &
  0.121 &
  0.0882 &
  0.851 \\ \hline
\begin{tabular}[c]{@{}c@{}}P($S_1$ and $S_2$ \\ have same maximums)\end{tabular} &
  0.946 &
  0.932 &
  0.987 \\ \hline
\end{tabular}
\end{table}

\clearpage

%-----------------------------------------------------------------
\section{Neonatal Cohort}

\subsection{Volume Registration: Power Thresholds}

\begin{figure}[t]
	\centering
	\begin{subfigure}{0.4\textwidth}
		\centering
		\includegraphics[width=1.0\textwidth]{6/figures/neonates-bold-fd-150.png}
		\caption{FD of Original Sequences.}
	\end{subfigure}
	\hspace{0.05\textwidth}
	\begin{subfigure}{0.4\textwidth}
		\centering
		\includegraphics[width=1.0\textwidth]{6/figures/neonates-bold-dvars-150.png}
		\caption{DVARS of Original Sequences.}
	\end{subfigure}
	
	\begin{subfigure}{0.4\textwidth}
		\centering
		\includegraphics[width=1.0\textwidth]{6/figures/neonates-trad-fd-150.png}
		\caption{FD of Traditionally Registered Sequences.}
	\end{subfigure}
	\hspace{0.05\textwidth}
	\begin{subfigure}{0.4\textwidth}
		\centering
		\includegraphics[width=1.0\textwidth]{6/figures/neonates-trad-dvars-150.png}
		\caption{DVARS of Traditionally Registered Sequences.}
	\end{subfigure}
	
	\begin{subfigure}{0.4\textwidth}
		\centering
		\includegraphics[width=1.0\textwidth]{6/figures/neonates-dag-fd-150.png}
		\caption{FD of DAG-Registered Sequences.}
	\end{subfigure}
	\hspace{0.05\textwidth}
	\begin{subfigure}{0.4\textwidth}
		\centering
		\includegraphics[width=1.0\textwidth]{6/figures/neonates-dag-dvars-150.png}
		\caption{DVARS of DAG-Registered Sequences.}
	\end{subfigure}
\caption{The means and standard deviations of the FD and DVARS metrics for all neonatal images both before and after registration.}
\label{fig:neonate-power-dists}
\end{figure}

\begin{table}[]
\centering
\caption{The number and percentage of image volumes across all sequences in the neonatal cohort which meet the usability thresholds of FD \textless 0.2 mm and DVARS \textless 2.5\%.}
\label{tab:neonate-power-thresh}
\begin{tabular}{|c|c|c|c|}
\hline
\textbf{Threshold Met} &
  \textbf{\begin{tabular}[c]{@{}c@{}}Original\\  Sequences\end{tabular}} &
  \textbf{\begin{tabular}[c]{@{}c@{}}Traditionally Registered \\ Sequences\end{tabular}} &
  \textbf{\begin{tabular}[c]{@{}c@{}}DAG-Registered \\ Sequences\end{tabular}} \\ \hline
FD (count)    & 16495  & 14264 & 14173 \\ \hline
DVARS (count) & 16820  & 13903 & 13752 \\ \hline
Both (count)  & 15332  & 12837 & 12684 \\ \hline
FD (\%)       & 69.59  & 60.18 & 59.79 \\ \hline
DVARS(\%)     & 70.96  & 58.65 & 58.02 \\ \hline
Both (\%)     & 64.68  & 54.16 & 53.51 \\ \hline
\end{tabular}
\end{table}

\begin{table}[]
\centering
\caption{Results from the t-tests comparing the counts for the numbers of images meeting the FD, DVARS, and FD and DVARS thresholds for sequence type $S_1$ and sequence type $S_2$.}
\label{tab:neonate-power-ttest}
\begin{tabular}{|c|c|c|c|}
\hline
\textbf{Sequence Type 1 ($S_1$)} &
  \textbf{Original} &
  \textbf{Original} &
  \textbf{\begin{tabular}[c]{@{}c@{}}Traditionally \\ Registered\end{tabular}} \\ \hline
\textbf{Sequence Type 2 ($S_2$)} &
  \textbf{\begin{tabular}[c]{@{}c@{}}Traditionally\\ Registered\end{tabular}} &
  \textbf{\begin{tabular}[c]{@{}c@{}}DAG\\ Registered\end{tabular}} &
  \textbf{\begin{tabular}[c]{@{}c@{}}DAG\\ Registered\end{tabular}} \\ \hline
\begin{tabular}[c]{@{}c@{}}P($S_1$ and $S_2$ have \\ same FD counts)\end{tabular} &
  0.0110 &
  0.00813 &
  0.924 \\ \hline
\begin{tabular}[c]{@{}c@{}}P($S_1$ and $S_2$ have \\ same DVARS counts)\end{tabular} &
  0.00163 &
  0.000942 &
  0.880 \\ \hline
\begin{tabular}[c]{@{}c@{}}P($S_1$ and $S_2$ have \\ same FD and DVARS counts)\end{tabular} &
  0.00779 &
  0.00475 &
  0.879 \\ \hline
\end{tabular}
\end{table}

\begin{table}[]
\centering
\caption{The number of subjects whose sequences of types $S_1$ and $S_2$ had different FD distributions.}
\label{tab:neonate-fd-kstest}
\begin{tabular}{|c|c|c|c|}
\hline
\textbf{\begin{tabular}[c]{@{}c@{}}\# Sequences \\  Type 1 ($S_1$)\end{tabular}} &
  \textbf{\begin{tabular}[c]{@{}c@{}}\# Sequences \\ Type 2 ($S_2$)\end{tabular}} &
  \textbf{\begin{tabular}[c]{@{}c@{}}\# Sequences \\ p \textless 0.05\end{tabular}} &
  \textbf{\begin{tabular}[c]{@{}c@{}}\# Sequences \\ p \textless 0.005\end{tabular}} \\ \hline
Original                                                            & \begin{tabular}[c]{@{}c@{}}Traditionally\\ Registered\end{tabular} & 36	 & 32 \\ \hline
Original                                                            & \begin{tabular}[c]{@{}c@{}}DAG\\ Registered\end{tabular}           & 42 & 38 \\ \hline
\begin{tabular}[c]{@{}c@{}}Traditionally \\ Registered\end{tabular} & \begin{tabular}[c]{@{}c@{}}DAG\\ Registered\end{tabular}           & 13 & 5 \\ \hline
\end{tabular}
\end{table}

\begin{table}[]
\centering
\caption{The number of subjects whose sequences of types $S_1$ and $S_2$ had different DVARS distributions.}
\label{tab:neonate-dvars-kstest}
\begin{tabular}{|c|c|c|c|}
\hline
\textbf{\begin{tabular}[c]{@{}c@{}}\# Sequences \\ Type 1($S_1$)\end{tabular}} &
  \textbf{\begin{tabular}[c]{@{}c@{}}\# Sequences \\ Type 2 ($S_2$)\end{tabular}} &
  \textbf{\begin{tabular}[c]{@{}c@{}}\# Sequences \\ p \textless 0.05\end{tabular}} &
  \textbf{\begin{tabular}[c]{@{}c@{}}\# Sequences \\ p \textless 0.005\end{tabular}} \\ \hline
Original                                                            & \begin{tabular}[c]{@{}c@{}}Traditionally\\ Registered\end{tabular} & 44 & 39 \\ \hline
Original                                                            & \begin{tabular}[c]{@{}c@{}}DAG\\ Registered\end{tabular}           & 45 & 44 \\ \hline
\begin{tabular}[c]{@{}c@{}}Traditionally \\ Registered\end{tabular} & \begin{tabular}[c]{@{}c@{}}DAG\\ Registered\end{tabular}           & 11  & 5  \\ \hline
\end{tabular}
\end{table}

\clearpage

\subsection{Volume Registration: Sequence Duration Motion}

\begin{figure}
\centering
\includegraphics[width=0.5\textwidth]{6/figures/neonates-dice-box.png}
\caption{Boxplots of the values of all Dice matrices for the original sequences, the traditionally registered sequences, and the DAG-registered sequences for the neonatal cohort.}
\label{fig:neonates-dice-box}
\end{figure}

\begin{table}[]
\centering
\caption{Results of t-tests comparing the descriptive statistics of the Dice matrices for the neonatal cohort.}
\label{tab:neonates-dice-ttest}
\begin{tabular}{|c|c|c|c|}
\hline
\textbf{Sequence Type 1 ($S_1$)} &
  \textbf{Original} &
  \textbf{Original} &
  \textbf{\begin{tabular}[c]{@{}c@{}}Traditionally \\ Registered\end{tabular}} \\ \hline
\textbf{Sequence Type 2 ($S_2$)} &
  \textbf{\begin{tabular}[c]{@{}c@{}}Traditionally\\ Registered\end{tabular}} &
  \textbf{\begin{tabular}[c]{@{}c@{}}DAG\\ Registered\end{tabular}} &
  \textbf{\begin{tabular}[c]{@{}c@{}}DAG\\ Registered\end{tabular}} \\ \hline
\begin{tabular}[c]{@{}c@{}}P($S_1$ and $S_2$ \\ have same minimums)\end{tabular} &
  0.523 &
  0.542 &
  0.977 \\ \hline
\begin{tabular}[c]{@{}c@{}}P($S_1$ and $S_2$ \\ have same 1st quartile)\end{tabular} &
  0.468 &
  0.515 &
  0.933 \\ \hline
\begin{tabular}[c]{@{}c@{}}P($S_1$ and $S_2$ \\ have same medians)\end{tabular} &
  0.329 &
  0.292 &
  0.937 \\ \hline
\begin{tabular}[c]{@{}c@{}}P($S_1$ and $S_2$ \\ have same 3rd quartiles)\end{tabular} &
  0.149 &
  0.115 &
  0.890 \\ \hline
\begin{tabular}[c]{@{}c@{}}P($S_1$ and $S_2$ \\ have same maximums)\end{tabular} &
  1.0 &
  1.0 &
  1.0 \\ \hline
\end{tabular}
\end{table}

\clearpage

\begin{figure}
\centering
\includegraphics[width=0.5\textwidth]{6/figures/neonates-mi-box.png}
\caption{Boxplots of the values of all MI matrices for the original sequences, the traditionally registered sequences, and the DAG-registered sequences for the neonatal cohort.}
\label{fig:neonates-mi-box}
\end{figure}

\begin{table}[]
\centering
\caption{Results of t-tests comparing the descriptive statistics of the MI matrices for the neonatal data.}
\label{tab:neonates-mi-ttest}
\begin{tabular}{|c|c|c|c|}
\hline
\textbf{Sequence Type 1 ($S_1$)} &
  \textbf{Original} &
  \textbf{Original} &
  \textbf{\begin{tabular}[c]{@{}c@{}}Traditionally \\ Registered\end{tabular}} \\ \hline
\textbf{Sequence Type 2 ($S_2$)} &
  \textbf{\begin{tabular}[c]{@{}c@{}}Traditionally\\ Registered\end{tabular}} &
  \textbf{\begin{tabular}[c]{@{}c@{}}DAG\\ Registered\end{tabular}} &
  \textbf{\begin{tabular}[c]{@{}c@{}}DAG\\ Registered\end{tabular}} \\ \hline
\begin{tabular}[c]{@{}c@{}}P($S_1$ and $S_2$ \\ have same minimums)\end{tabular} &
  0.853 &
  0.874 &
  0.978 \\ \hline
\begin{tabular}[c]{@{}c@{}}P($S_1$ and $S_2$ \\ have same 1st quartile)\end{tabular} &
  0.794 &
  0.809 &
  0.985 \\ \hline
\begin{tabular}[c]{@{}c@{}}P($S_1$ and $S_2$ \\ have same medians)\end{tabular} &
  0.762 &
  0.758 &
  0.996 \\ \hline
\begin{tabular}[c]{@{}c@{}}P($S_1$ and $S_2$ \\ have same 3rd quartiles)\end{tabular} &
  0.755 &
  0.743 &
  0.987 \\ \hline
\begin{tabular}[c]{@{}c@{}}P($S_1$ and $S_2$ \\ have same maximums)\end{tabular} &
  0.956 &
  0.938 &
  0.982 \\ \hline
\end{tabular}
\end{table}

\clearpage

\begin{table}[]
\centering
\caption{The number of neonatal subjects whose sequences of types $S_1$ and $S_2$ had different MI distributions.}
\label{tab:neonates-mi-kstest}
\begin{tabular}{|c|c|c|c|}
\hline
\textbf{\begin{tabular}[c]{@{}c@{}}\# Sequences \\ Type 1 ($S_1$)\end{tabular}} &
  \textbf{\begin{tabular}[c]{@{}c@{}}\# Sequences \\ Type 2 ($S_2$)\end{tabular}} &
  \textbf{\begin{tabular}[c]{@{}c@{}}\# Sequences \\ p \textless 0.05\end{tabular}} &
  \textbf{\begin{tabular}[c]{@{}c@{}}\# Sequences \\ p \textless 0.005\end{tabular}} \\ \hline
Original                                                            & \begin{tabular}[c]{@{}c@{}}Traditionally\\ Registered\end{tabular} & 66 & 64 \\ \hline
Original                                                            & \begin{tabular}[c]{@{}c@{}}DAG\\ Registered\end{tabular}           & 71 & 69 \\ \hline
\begin{tabular}[c]{@{}c@{}}Traditionally \\ Registered\end{tabular} & \begin{tabular}[c]{@{}c@{}}DAG\\ Registered\end{tabular}           & 64 & 64  \\ \hline
\end{tabular}
\end{table}

\begin{table}[]
\centering
\caption{The number of neonatal subjects whose sequences of types $S_1$ and $S_2$ had different Dice distributions.}
\label{tab:neonates-dice-kstest}
\begin{tabular}{|c|c|c|c|}
\hline
\textbf{\begin{tabular}[c]{@{}c@{}}\# Sequences \\ Type 1 ($S_1$)\end{tabular}} &
  \textbf{\begin{tabular}[c]{@{}c@{}}\# Sequences \\ Type 2 ($S_2$)\end{tabular}} &
  \textbf{\begin{tabular}[c]{@{}c@{}}\# Sequences \\ p \textless 0.05\end{tabular}} &
  \textbf{\begin{tabular}[c]{@{}c@{}}\# Sequences \\ p \textless 0.005\end{tabular}} \\ \hline
Original                                                            & \begin{tabular}[c]{@{}c@{}}Traditionally\\ Registered\end{tabular} & 67 & 66 \\ \hline
Original                                                            & \begin{tabular}[c]{@{}c@{}}DAG\\ Registered\end{tabular}           & 71 & 69 \\ \hline
\begin{tabular}[c]{@{}c@{}}Traditionally \\ Registered\end{tabular} & \begin{tabular}[c]{@{}c@{}}DAG\\ Registered\end{tabular}           & 68 & 64  \\ \hline
\end{tabular}
\end{table}

\section{Fetal Cohort}

\subsection{Brain}

\subsubsection{Volume Registration: Power Thresholds}

\begin{figure}[t]
	\centering
	\begin{subfigure}{0.4\textwidth}
		\centering
		\includegraphics[width=1.0\textwidth]{6/figures/fetal-brain-bold-fd-150.png}
		\caption{FD of Original Sequences.}
	\end{subfigure}
	\hspace{0.05\textwidth}
	\begin{subfigure}{0.4\textwidth}
		\centering
		\includegraphics[width=1.0\textwidth]{6/figures/fetal-brain-bold-dvars-150.png}
		\caption{DVARS of Original Sequences.}
	\end{subfigure}
	
	\begin{subfigure}{0.4\textwidth}
		\centering
		\includegraphics[width=1.0\textwidth]{6/figures/fetal-brain-trad-fd-150.png}
		\caption{FD of Traditionally Registered Sequences.}
	\end{subfigure}
	\hspace{0.05\textwidth}
	\begin{subfigure}{0.4\textwidth}
		\centering
		\includegraphics[width=1.0\textwidth]{6/figures/fetal-brain-trad-dvars-150.png}
		\caption{DVARS of Traditionally Registered Sequences.}
	\end{subfigure}
	
	\begin{subfigure}{0.4\textwidth}
		\centering
		\includegraphics[width=1.0\textwidth]{6/figures/fetal-brain-dag-fd-150.png}
		\caption{FD of DAG-Registered Sequences.}
	\end{subfigure}
	\hspace{0.05\textwidth}
	\begin{subfigure}{0.4\textwidth}
		\centering
		\includegraphics[width=1.0\textwidth]{6/figures/fetal-brain-dag-dvars-150.png}
		\caption{DVARS of DAG-Registered Sequences.}
	\end{subfigure}
\caption{The means and standard deviations of the FD and DVARS metrics for all fetal brain images both before and after registration.}
\label{fig:fetal-brain-power-dists}
\end{figure}

\begin{table}[]
\centering
\caption{The number and percentage of image volumes across all sequences in the fetal brain image data set which meet the usability thresholds of FD \textless 0.2 mm and DVARS \textless 2.5\%.}
\label{tab:fetal-brain-power-thresh}
\begin{tabular}{|c|c|c|c|}
\hline
\textbf{Threshold Met} &
  \textbf{\begin{tabular}[c]{@{}c@{}}Original\\  Sequences\end{tabular}} &
  \textbf{\begin{tabular}[c]{@{}c@{}}Traditionally Registered \\ Sequences\end{tabular}} &
  \textbf{\begin{tabular}[c]{@{}c@{}}DAG-Registered \\ Sequences\end{tabular}} \\ \hline
FD (count)    & 575    & 581   & 561 \\ \hline
DVARS (count) & 7      & 84    & 7 \\ \hline
Both (count)  & 7      & 80    & 7 \\ \hline
FD (\%)       & 4.775  & 4.854 & 4.659 \\ \hline
DVARS(\%)     & 0.058  & 0.702 & 0.058 \\ \hline
Both (\%)     & 0.058  & 0.668 & 0.058 \\ \hline
\end{tabular}
\end{table}

\begin{table}[]
\centering
\caption{Results from the t-tests comparing the counts for the numbers of images meeting the FD, DVARS, and FD and DVARS thresholds for fetal brain sequence type $S_1$ and sequence type $S_2$.}
\label{tab:fetal-brain-power-ttest}
\begin{tabular}{|c|c|c|c|}
\hline
\textbf{Sequence Type 1 ($S_1$)} &
  \textbf{Original} &
  \textbf{Original} &
  \textbf{\begin{tabular}[c]{@{}c@{}}Traditionally \\ Registered\end{tabular}} \\ \hline
\textbf{Sequence Type 2 ($S_2$)} &
  \textbf{\begin{tabular}[c]{@{}c@{}}Traditionally\\ Registered\end{tabular}} &
  \textbf{\begin{tabular}[c]{@{}c@{}}DAG\\ Registered\end{tabular}} &
  \textbf{\begin{tabular}[c]{@{}c@{}}DAG\\ Registered\end{tabular}} \\ \hline
\begin{tabular}[c]{@{}c@{}}P($S_1$ and $S_2$ have \\ same FD counts)\end{tabular} &
  0.811 &
  0.926 &
  0.883 \\ \hline
\begin{tabular}[c]{@{}c@{}}P($S_1$ and $S_2$ have \\ same DVARS counts)\end{tabular} &
  0.159 &
  1.0 &
  0.159 \\ \hline
\begin{tabular}[c]{@{}c@{}}P($S_1$ and $S_2$ have \\ same FD and DVARS counts)\end{tabular} &
  0.159 &
  1.0 &
  0.159 \\ \hline
\end{tabular}
\end{table}

Table \ref{tab:fetal-brain-power-ttest} shows the results of a set of t-tests which determine if the distribution of metric X for sequence type $S_1$ is the same as the distribution of metric X for sequence type $S_2$. Fetal brain.

\begin{table}[]
\centering
\caption{The number of subjects whose sequences of types $S_1$ and $S_2$ had different FD distributions according to the Kolmogorov-Smirnov test.}
\label{tab:fetal-brain-fd-kstest}
\begin{tabular}{|c|c|c|c|}
\hline
\textbf{\begin{tabular}[c]{@{}c@{}}\# Sequences \\ Type 1 ($S_1$)\end{tabular}} &
  \textbf{\begin{tabular}[c]{@{}c@{}}\# Sequences \\ Type 2 ($S_2$)\end{tabular}} &
  \textbf{\begin{tabular}[c]{@{}c@{}}\# Sequences \\ p \textless 0.05\end{tabular}} &
  \textbf{\begin{tabular}[c]{@{}c@{}}\# Sequences \\ p \textless 0.005\end{tabular}} \\ \hline
Original                                                            & \begin{tabular}[c]{@{}c@{}}Traditionally\\ Registered\end{tabular} & 14 & 9 \\ \hline
Original                                                            & \begin{tabular}[c]{@{}c@{}}DAG\\ Registered\end{tabular}           & 2  & 2 \\ \hline
\begin{tabular}[c]{@{}c@{}}Traditionally \\ Registered\end{tabular} & \begin{tabular}[c]{@{}c@{}}DAG\\ Registered\end{tabular}           & 13 & 9 \\ \hline
\end{tabular}
\end{table}

\begin{table}[]
\centering
\caption{The number of subjects whose sequences of types $S_1$ and $S_2$ had different DVARS distributions according to the Kolmogorov-Smirnov test.}
\label{tab:fetal-brain-dvars-kstest}
\begin{tabular}{|c|c|c|c|}
\hline
\textbf{\begin{tabular}[c]{@{}c@{}}\# Sequences \\ Type 1 ($S_1$)\end{tabular}} &
  \textbf{\begin{tabular}[c]{@{}c@{}}\# Sequences \\ Type 2 ($S_2$)\end{tabular}} &
  \textbf{\begin{tabular}[c]{@{}c@{}}\# Sequences \\ p \textless 0.05\end{tabular}} &
  \textbf{\begin{tabular}[c]{@{}c@{}}\# Sequences \\ p \textless 0.005\end{tabular}} \\ \hline
Original                                                            & \begin{tabular}[c]{@{}c@{}}Traditionally\\ Registered\end{tabular} & 3 & 3 \\ \hline
Original                                                            & \begin{tabular}[c]{@{}c@{}}DAG\\ Registered\end{tabular}           & 2 & 1 \\ \hline
\begin{tabular}[c]{@{}c@{}}Traditionally \\ Registered\end{tabular} & \begin{tabular}[c]{@{}c@{}}DAG\\ Registered\end{tabular}           & 2  & 2  \\ \hline
\end{tabular}
\end{table}

\clearpage

\subsubsection{Volume Registration: Sequence Duration Motion}

\begin{figure}
\centering
\includegraphics[width=0.5\textwidth]{6/figures/fetal-brain-dice-box.png}
\caption{Boxplots of the values of all Dice matrices for the original sequences, the traditionally registered sequences, and the DAG-registered sequences for the fetal-brain images.}
\label{fig:fetal-brain-dice-box}
\end{figure}

\begin{table}[]
\centering
\caption{Results of t-tests comparing the descriptive statistics of the Dice matrices for the fetal brain data.}
\label{tab:fetal-brain-dice-ttest}
\begin{tabular}{|c|c|c|c|}
\hline
\textbf{Sequence Type 1 ($S_1$)} &
  \textbf{Original} &
  \textbf{Original} &
  \textbf{\begin{tabular}[c]{@{}c@{}}Traditionally \\ Registered\end{tabular}} \\ \hline
\textbf{Sequence Type 2 ($S_2$)} &
  \textbf{\begin{tabular}[c]{@{}c@{}}Traditionally\\ Registered\end{tabular}} &
  \textbf{\begin{tabular}[c]{@{}c@{}}DAG\\ Registered\end{tabular}} &
  \textbf{\begin{tabular}[c]{@{}c@{}}DAG\\ Registered\end{tabular}} \\ \hline
\begin{tabular}[c]{@{}c@{}}P($S_1$ and $S_2$ \\ have same minimums)\end{tabular} &
  0.259 &
  0.932 &
  0.289 \\ \hline
\begin{tabular}[c]{@{}c@{}}P($S_1$ and $S_2$ \\ have same 1st quartile)\end{tabular} &
  0.254 &
  0.996 &
  0.252 \\ \hline
\begin{tabular}[c]{@{}c@{}}P($S_1$ and $S_2$ \\ have same medians)\end{tabular} &
  0.658 &
  0.970 &
  0.686 \\ \hline
\begin{tabular}[c]{@{}c@{}}P($S_1$ and $S_2$ \\ have same 3rd quartiles)\end{tabular} &
  0.973 &
  0.921 &
  0.896 \\ \hline
\begin{tabular}[c]{@{}c@{}}P($S_1$ and $S_2$ \\ have same maximums)\end{tabular} &
  1.0 &
  1.0 &
  1.0 \\ \hline
\end{tabular}
\end{table}

\clearpage

\begin{figure}
\centering
\includegraphics[width=0.5\textwidth]{6/figures/fetal-brain-mi-box.png}
\caption{Boxplots of the values of all MI matrices for the original sequences, the traditionally registered sequences, and the DAG-registered sequences for the fetal brain images.}
\label{fig:fetal-brain-mi-box}
\end{figure}

\begin{table}[]
\centering
\caption{Results of t-tests comparing the descriptive statistics of the MI matrices for the fetal brain data.}
\label{tab:fetal-brain-mi-ttest}
\begin{tabular}{|c|c|c|c|}
\hline
\textbf{Sequence Type 1 ($S_1$)} &
  \textbf{Original} &
  \textbf{Original} &
  \textbf{\begin{tabular}[c]{@{}c@{}}Traditionally \\ Registered\end{tabular}} \\ \hline
\textbf{Sequence Type 2 ($S_2$)} &
  \textbf{\begin{tabular}[c]{@{}c@{}}Traditionally\\ Registered\end{tabular}} &
  \textbf{\begin{tabular}[c]{@{}c@{}}DAG\\ Registered\end{tabular}} &
  \textbf{\begin{tabular}[c]{@{}c@{}}DAG\\ Registered\end{tabular}} \\ \hline
\begin{tabular}[c]{@{}c@{}}P($S_1$ and $S_2$ \\ have same minimums)\end{tabular} &
  0.673 &
  0.845 &
  0.816 \\ \hline
\begin{tabular}[c]{@{}c@{}}P($S_1$ and $S_2$ \\ have same 1st quartile)\end{tabular} &
  0.765 &
  0.963 &
  0.798 \\ \hline
\begin{tabular}[c]{@{}c@{}}P($S_1$ and $S_2$ \\ have same medians)\end{tabular} &
  0.764 &
  0.963 &
  0.798 \\ \hline
\begin{tabular}[c]{@{}c@{}}P($S_1$ and $S_2$ \\ have same 3rd quartiles)\end{tabular} &
  0.760 &
  0.959 &
  0.797 \\ \hline
\begin{tabular}[c]{@{}c@{}}P($S_1$ and $S_2$ \\ have same maximums)\end{tabular} &
  0.539 &
  0.999 &
  0.539 \\ \hline
\end{tabular}
\end{table}

\clearpage

\begin{table}[]
\centering
\caption{The number of subjects whose sequences of types $S_1$ and $S_2$ had different MI distributions.}
\label{tab:fetal-brain-mi-kstest}
\begin{tabular}{|c|c|c|c|}
\hline
\textbf{\begin{tabular}[c]{@{}c@{}}\# Sequences \\ Type 1 ($S_1$)\end{tabular}} &
  \textbf{\begin{tabular}[c]{@{}c@{}}\# Sequences \\ Type 2 ($S_2$)\end{tabular}} &
  \textbf{\begin{tabular}[c]{@{}c@{}}\# Sequences \\ p \textless 0.05\end{tabular}} &
  \textbf{\begin{tabular}[c]{@{}c@{}}\# Sequences \\ p \textless 0.005\end{tabular}} \\ \hline
Original                                                            & \begin{tabular}[c]{@{}c@{}}Traditionally\\ Registered\end{tabular} & 13 & 10 \\ \hline
Original                                                            & \begin{tabular}[c]{@{}c@{}}DAG\\ Registered\end{tabular}           & 12 & 9 \\ \hline
\begin{tabular}[c]{@{}c@{}}Traditionally \\ Registered\end{tabular} & \begin{tabular}[c]{@{}c@{}}DAG\\ Registered\end{tabular}           & 14 & 9  \\ \hline
\end{tabular}
\end{table}

\begin{table}[]
\centering
\caption{The number of subjects whose sequences of types $S_1$ and $S_2$ had different Dice distributions.}
\label{tab:fetal-brain-dice-kstest}
\begin{tabular}{|c|c|c|c|}
\hline
\textbf{\begin{tabular}[c]{@{}c@{}}\# Sequences \\ Type 1 ($S_1$)\end{tabular}} &
  \textbf{\begin{tabular}[c]{@{}c@{}}\# Sequences \\ Type 2 ($S_2$)\end{tabular}} &
  \textbf{\begin{tabular}[c]{@{}c@{}}\# Sequences \\ p \textless 0.05\end{tabular}} &
  \textbf{\begin{tabular}[c]{@{}c@{}}\# Sequences \\ p \textless 0.005\end{tabular}} \\ \hline
Original                                                            & \begin{tabular}[c]{@{}c@{}}Traditionally\\ Registered\end{tabular} & 11 & 9 \\ \hline
Original                                                            & \begin{tabular}[c]{@{}c@{}}DAG\\ Registered\end{tabular}           & 7 & 7 \\ \hline
\begin{tabular}[c]{@{}c@{}}Traditionally \\ Registered\end{tabular} & \begin{tabular}[c]{@{}c@{}}DAG\\ Registered\end{tabular}           & 10  & 8  \\ \hline
\end{tabular}
\end{table}

\clearpage

\subsection{Placenta}

\subsubsection{Volume Registration: Power Thresholds}

\begin{figure}[t]
	\centering
	\begin{subfigure}{0.4\textwidth}
		\centering
		\includegraphics[width=1.0\textwidth]{6/figures/fetal-placenta-bold-fd-150.png}
		\caption{FD of Original Sequences.}
	\end{subfigure}
	\hspace{0.05\textwidth}
	\begin{subfigure}{0.4\textwidth}
		\centering
		\includegraphics[width=1.0\textwidth]{6/figures/fetal-placenta-bold-dvars-150.png}
		\caption{DVARS of Original Sequences.}
	\end{subfigure}
	
	\begin{subfigure}{0.4\textwidth}
		\centering
		\includegraphics[width=1.0\textwidth]{6/figures/fetal-placenta-trad-fd-150.png}
		\caption{FD of Traditionally Registered Sequences.}
	\end{subfigure}
	\hspace{0.05\textwidth}
	\begin{subfigure}{0.4\textwidth}
		\centering
		\includegraphics[width=1.0\textwidth]{6/figures/fetal-placenta-trad-dvars-150.png}
		\caption{DVARS of Traditionally Registered Sequences.}
	\end{subfigure}
	
	\begin{subfigure}{0.4\textwidth}
		\centering
		\includegraphics[width=1.0\textwidth]{6/figures/fetal-placenta-dag-fd-150.png}
		\caption{FD of DAG-Registered Sequences.}
	\end{subfigure}
	\hspace{0.05\textwidth}
	\begin{subfigure}{0.4\textwidth}
		\centering
		\includegraphics[width=1.0\textwidth]{6/figures/fetal-placenta-dag-dvars-150.png}
		\caption{DVARS of DAG-Registered Sequences.}
	\end{subfigure}
\caption{The means and standard deviations of the FD and DVARS metrics for all placenta images both before and after registration.}
\label{fig:fetal-placenta-power-dists}
\end{figure}

\begin{table}[]
\centering
\caption{The number and percentage of image volumes across all sequences in the fetal placenta image data set which meet the usability thresholds of FD \textless 0.2 mm and DVARS \textless 2.5\%.}
\label{tab:fetal-placenta-power-thresh}
\begin{tabular}{|c|c|c|c|}
\hline
\textbf{Threshold Met} &
  \textbf{\begin{tabular}[c]{@{}c@{}}Original\\  Sequences\end{tabular}} &
  \textbf{\begin{tabular}[c]{@{}c@{}}Traditionally Registered \\ Sequences\end{tabular}} &
  \textbf{\begin{tabular}[c]{@{}c@{}}DAG-Registered \\ Sequences\end{tabular}} \\ \hline
FD (count)    & 10017  & 4113   & 4005   \\ \hline
DVARS (count) & 17     & 1056   & 1624   \\ \hline
Both (count)  & 17     & 599    & 996    \\ \hline
FD (\%)       & 43.646 & 44.245 & 45.020 \\ \hline
DVARS(\%)     & 0.170  & 11.36  & 18.255 \\ \hline
Both (\%)     & 0.170  & 6.444  & 11.196 \\ \hline
\end{tabular}
\end{table}

\begin{table}[]
\centering
\caption{Results from the t-tests comparing the counts for the numbers of images meeting the FD, DVARS, and FD and DVARS thresholds for fetal placenta sequence type $S_1$ and sequence type $S_2$.}
\label{tab:fetal-placenta-power-ttest}
\begin{tabular}{|c|c|c|c|}
\hline
\textbf{Sequence Type 1 ($S_1$)} &
  \textbf{Original} &
  \textbf{Original} &
  \textbf{\begin{tabular}[c]{@{}c@{}}Traditionally \\ Registered\end{tabular}} \\ \hline
\textbf{Sequence Type 2 ($S_2$)} &
  \textbf{\begin{tabular}[c]{@{}c@{}}Traditionally\\ Registered\end{tabular}} &
  \textbf{\begin{tabular}[c]{@{}c@{}}DAG\\ Registered\end{tabular}} &
  \textbf{\begin{tabular}[c]{@{}c@{}}DAG\\ Registered\end{tabular}} \\ \hline
\begin{tabular}[c]{@{}c@{}}P($S_1$ and $S_2$ have \\ same FD counts)\end{tabular} &
  0.519 &
  0.350 &
  0.775 \\ \hline
\begin{tabular}[c]{@{}c@{}}P($S_1$ and $S_2$ have \\ same DVARS counts)\end{tabular} &
  5.38 E -6 &
  5.65 E -9 &
  0.101 \\ \hline
\begin{tabular}[c]{@{}c@{}}P($S_1$ and $S_2$ have \\ same FD and DVARS counts)\end{tabular} &
  5.18 E -6 &
  1.92 E -8 &
  0.0571 \\ \hline
\end{tabular}
\end{table}

\begin{table}[]
\centering
\caption{The number of placental images whose sequences of types $S_1$ and $S_2$ had different FD distributions according to the Kolmogorov-Smirnov test.}
\label{tab:fetal-placenta-fd-kstest}
\begin{tabular}{|c|c|c|c|}
\hline
\textbf{\begin{tabular}[c]{@{}c@{}}\# Sequences \\ Type 1 ($S_1$)\end{tabular}} &
  \textbf{\begin{tabular}[c]{@{}c@{}}\# Sequences \\ Type 2 ($S_2$)\end{tabular}} &
  \textbf{\begin{tabular}[c]{@{}c@{}}\# Sequences \\ p \textless 0.05\end{tabular}} &
  \textbf{\begin{tabular}[c]{@{}c@{}}\# Sequences \\ p \textless 0.005\end{tabular}} \\ \hline
Original                                                            & \begin{tabular}[c]{@{}c@{}}Traditionally\\ Registered\end{tabular} & 21 & 21 \\ \hline
Original                                                            & \begin{tabular}[c]{@{}c@{}}DAG\\ Registered\end{tabular}           & 32 & 30 \\ \hline
\begin{tabular}[c]{@{}c@{}}Traditionally \\ Registered\end{tabular} & \begin{tabular}[c]{@{}c@{}}DAG\\ Registered\end{tabular}           & 31 & 20 \\ \hline
\end{tabular}
\end{table}

\begin{table}[]
\centering
\caption{The number of placental images whose sequences of types $S_1$ and $S_2$ had different DVARS distributions according to the Kolmogorov-Smirnov test.}
\label{tab:fetal-placenta-dvars-kstest}
\begin{tabular}{|c|c|c|c|}
\hline
\textbf{\begin{tabular}[c]{@{}c@{}}\# Sequences \\ Type 1 ($S_1$)\end{tabular}} &
  \textbf{\begin{tabular}[c]{@{}c@{}}\# Sequences \\ Type 2 ($S_2$)\end{tabular}} &
  \textbf{\begin{tabular}[c]{@{}c@{}}\# Sequences \\ p \textless 0.05\end{tabular}} &
  \textbf{\begin{tabular}[c]{@{}c@{}}\# Sequences \\ p \textless 0.005\end{tabular}} \\ \hline
Original                                                            & \begin{tabular}[c]{@{}c@{}}Traditionally\\ Registered\end{tabular} & 20 & 20 \\ \hline
Original                                                            & \begin{tabular}[c]{@{}c@{}}DAG\\ Registered\end{tabular}           & 29 & 29 \\ \hline
\begin{tabular}[c]{@{}c@{}}Traditionally \\ Registered\end{tabular} & \begin{tabular}[c]{@{}c@{}}DAG\\ Registered\end{tabular}           & 19 & 19  \\ \hline
\end{tabular}
\end{table}

\clearpage

\subsubsection{Volume Registration: Sequence Duration Motion}

\begin{figure}
\centering
\includegraphics[width=0.5\textwidth]{6/figures/fetal-placenta-dice-box.png}
\caption{Boxplots of the values of all Dice matrices for the original sequences, the traditionally registered sequences, and the DAG-registered sequences for the placenta images.}
\label{fig:fetal-placenta-dice-box}
\end{figure}

\begin{table}[]
\centering
\caption{Results of t-tests comparing the descriptive statistics of the Dice matrices for the fetal placenta data.}
\label{tab:fetal-placenta-dice-ttest}
\begin{tabular}{|c|c|c|c|}
\hline
\textbf{Sequence Type 1 ($S_1$)} &
  \textbf{Original} &
  \textbf{Original} &
  \textbf{\begin{tabular}[c]{@{}c@{}}Traditionally \\ Registered\end{tabular}} \\ \hline
\textbf{Sequence Type 2 ($S_2$)} &
  \textbf{\begin{tabular}[c]{@{}c@{}}Traditionally\\ Registered\end{tabular}} &
  \textbf{\begin{tabular}[c]{@{}c@{}}DAG\\ Registered\end{tabular}} &
  \textbf{\begin{tabular}[c]{@{}c@{}}DAG\\ Registered\end{tabular}} \\ \hline
\begin{tabular}[c]{@{}c@{}}P($S_1$ and $S_2$ \\ have same minimums)\end{tabular} &
  6.39 E -5 &
  3.43 E -7 &
  0.257 \\ \hline
\begin{tabular}[c]{@{}c@{}}P($S_1$ and $S_2$ \\ have same 1st quartile)\end{tabular} &
  6.54 E -5 &
  2.35 E -6 &
  0.257 \\ \hline
\begin{tabular}[c]{@{}c@{}}P($S_1$ and $S_2$ \\ have same medians)\end{tabular} &
  0.000310 &
  9.46 E -5 &
  0.816 \\ \hline
\begin{tabular}[c]{@{}c@{}}P($S_1$ and $S_2$ \\ have same 3rd quartiles)\end{tabular} &
  0.096 &
  0.104 &
  0.902 \\ \hline
\begin{tabular}[c]{@{}c@{}}P($S_1$ and $S_2$ \\ have same maximums)\end{tabular} &
  1.0 &
  1.0 &
  1.0 \\ \hline
\end{tabular}
\end{table}

\clearpage

\begin{figure}
\centering
\includegraphics[width=0.5\textwidth]{6/figures/fetal-placenta-mi-box.png}
\caption{Boxplots of the values of all MI matrices for the original sequences, the traditionally registered sequences, and the DAG-registered sequences for the placenta images.}
\label{fig:fetal-placenta-mi-box}
\end{figure}

\begin{table}[]
\centering
\caption{Results of t-tests comparing the descriptive statistics of the MI matrices for the fetal placenta data.}
\label{tab:fetal-placenta-mi-ttest}
\begin{tabular}{|c|c|c|c|}
\hline
\textbf{Sequence Type 1 ($S_1$)} &
  \textbf{Original} &
  \textbf{Original} &
  \textbf{\begin{tabular}[c]{@{}c@{}}Traditionally \\ Registered\end{tabular}} \\ \hline
\textbf{Sequence Type 2 ($S_2$)} &
  \textbf{\begin{tabular}[c]{@{}c@{}}Traditionally\\ Registered\end{tabular}} &
  \textbf{\begin{tabular}[c]{@{}c@{}}DAG\\ Registered\end{tabular}} &
  \textbf{\begin{tabular}[c]{@{}c@{}}DAG\\ Registered\end{tabular}} \\ \hline
\begin{tabular}[c]{@{}c@{}}P($S_1$ and $S_2$ \\ have same minimums)\end{tabular} &
  0.00980 &
  0.00168 &
  0.536 \\ \hline
\begin{tabular}[c]{@{}c@{}}P($S_1$ and $S_2$ \\ have same 1st quartile)\end{tabular} &
  0.00742 &
  0.00132 &
  0.548 \\ \hline
\begin{tabular}[c]{@{}c@{}}P($S_1$ and $S_2$ \\ have same medians)\end{tabular} &
  0.00713 &
  0.00116 &
  0.535 \\ \hline
\begin{tabular}[c]{@{}c@{}}P($S_1$ and $S_2$ \\ have same 3rd quartiles)\end{tabular} &
  0.00807 &
  0.00118 &
  0.506 \\ \hline
\begin{tabular}[c]{@{}c@{}}P($S_1$ and $S_2$ \\ have same maximums)\end{tabular} &
  0.00145 &
  8.28 E -6 &
  0.212 \\ \hline
\end{tabular}
\end{table}

\clearpage

\begin{table}[]
\centering
\caption{The number of subjects whose placenta sequences of types $S_1$ and $S_2$ had different MI distributions.}
\label{tab:fetal-placenta-mi-kstest}
\begin{tabular}{|c|c|c|c|}
\hline
\textbf{\begin{tabular}[c]{@{}c@{}}\# Sequences \\ Type 1 ($S_1$)\end{tabular}} &
  \textbf{\begin{tabular}[c]{@{}c@{}}\# Sequences \\ Type 2 ($S_2$)\end{tabular}} &
  \textbf{\begin{tabular}[c]{@{}c@{}}\# Sequences \\ p \textless 0.05\end{tabular}} &
  \textbf{\begin{tabular}[c]{@{}c@{}}\# Sequences \\ p \textless 0.005\end{tabular}} \\ \hline
Original                                                            & \begin{tabular}[c]{@{}c@{}}Traditionally\\ Registered\end{tabular} & 22 & 22 \\ \hline
Original                                                            & \begin{tabular}[c]{@{}c@{}}DAG\\ Registered\end{tabular}           & 39 & 39 \\ \hline
\begin{tabular}[c]{@{}c@{}}Traditionally \\ Registered\end{tabular} & \begin{tabular}[c]{@{}c@{}}DAG\\ Registered\end{tabular}           & 38 & 38  \\ \hline
\end{tabular}
\end{table}

\begin{table}[]
\centering
\caption{The number of subjects whose placenta sequences of types $S_1$ and $S_2$ had different Dice distributions.}
\label{tab:fetal-placenta-dice-kstest}
\begin{tabular}{|c|c|c|c|}
\hline
\textbf{\begin{tabular}[c]{@{}c@{}}\# Sequences \\ Type 1 ($S_1$)\end{tabular}} &
  \textbf{\begin{tabular}[c]{@{}c@{}}\# Sequences \\ Type 2 ($S_2$)\end{tabular}} &
  \textbf{\begin{tabular}[c]{@{}c@{}}\# Sequences \\ p \textless 0.05\end{tabular}} &
  \textbf{\begin{tabular}[c]{@{}c@{}}\# Sequences \\ p \textless 0.005\end{tabular}} \\ \hline
Original                                                            & \begin{tabular}[c]{@{}c@{}}Traditionally\\ Registered\end{tabular} & 22 & 22 \\ \hline
Original                                                            & \begin{tabular}[c]{@{}c@{}}DAG\\ Registered\end{tabular}           & 39 & 39 \\ \hline
\begin{tabular}[c]{@{}c@{}}Traditionally \\ Registered\end{tabular} & \begin{tabular}[c]{@{}c@{}}DAG\\ Registered\end{tabular}           & 30 & 30  \\ \hline
\end{tabular}
\end{table}

\clearpage

\section{Characterizing Motion}

\subsection{Age Groups}

\begin{figure}[t]
	\centering
	\begin{subfigure}{0.49\textwidth}
		\centering
		\includegraphics[width=1.0\textwidth]{6/figures/agegroup-bold-fd-sns-agg.png}
		\caption{FD}
	\end{subfigure}
	\begin{subfigure}{0.49\textwidth}
		\centering
		\includegraphics[width=1.0\textwidth]{6/figures/agegroup-bold-dvars-sns-agg.png}
		\caption{DVARS}
	\end{subfigure}
	
	\begin{subfigure}{0.49\textwidth}
		\centering
		\includegraphics[width=1.0\textwidth]{6/figures/agegroup-bold-dice-sns-agg.png}
		\caption{Dice}
	\end{subfigure}
	\begin{subfigure}{0.49\textwidth}
		\centering
		\includegraphics[width=1.0\textwidth]{6/figures/agegroup-bold-mi-sns-agg.png}
		\caption{MI}
	\end{subfigure}
\caption{The preadolescent, neonatal, and fetal images clustered by each metric using agglomerative clustering and labeled by age group.}
\label{fig:mochar-ages-sns-agg}
\end{figure}

\clearpage

\subsection{CHD and Control}

\begin{figure}[t]
	\centering
	\begin{subfigure}{0.49\textwidth}
		\centering
		\includegraphics[width=1.0\textwidth]{6/figures/cohort-bold-fd-sns-agg.png}
		\caption{FD}
	\end{subfigure}
	\begin{subfigure}{0.49\textwidth}
		\centering
		\includegraphics[width=1.0\textwidth]{6/figures/cohort-bold-dvars-sns-agg.png}
		\caption{DVARS}
	\end{subfigure}
	
	\begin{subfigure}{0.49\textwidth}
		\centering
		\includegraphics[width=1.0\textwidth]{6/figures/cohort-bold-dice-sns-agg.png}
		\caption{Dice}
	\end{subfigure}
	\begin{subfigure}{0.49\textwidth}
		\centering
		\includegraphics[width=1.0\textwidth]{6/figures/cohort-bold-mi-sns-agg.png}
		\caption{MI}
	\end{subfigure}
\caption{The preadolescent, neonatal, and fetal images clustered by each metric using agglomerative clustering and labeled by CHD/Control status.}
\label{fig:mochar-cohorts-sns-agg}
\end{figure}
%\chapter{Proposed Timeline}

\begin{table}[h]
\begin{tabular}{p{.2\textwidth}|p{.8\textwidth}}
\textbf{Month} & \multicolumn{1}{l}{\textbf{Objectives}}                                                                         \\ \hline
June 2019      & Begin coordinating with lab members to find data for all subjects from all cohorts. \\
               & Generate simulated phantom images.                                                                               \\
               & Determine organization system for images and clinical data for each cohort.                                      \\
               & Determine organization system for registered images and results.                                                 \\ \hline
July 2019      & Continue obtaining and organizing data, if needed.                                                               \\
               & Run both volume registrations and the motion correction on preadolescent cohort                                  \\
               & Structure analysis of volume registration analyses.                                                              \\
               & Submit volume registration feasibility paper.                                                                    \\ \hline
August 2019    & Analyze corrected preadolescent images.                                                                          \\
               & Run volume registrations and the motion correction pipeline on the whole neonatal cohort.                        \\
               & Outline preadolescent motion correction paper.                                                                   \\

\end{tabular}
\end{table}

\begin{table}[t]
\begin{tabular}{p{.2\textwidth}|p{.8\textwidth}}
\textbf{Month} & \multicolumn{1}{l}{\textbf{Objectives}}                                                                         \\ \hline
September 2019 & Analyze corrected neonatal images. \\
               & Run volume registrations and the motion correction pipeline on the fetal images. \\
               & Write preadolescent motion correction paper.                                                                     \\ \hline
October 2019   & \textbf{*Update on project status to committee.} \\                                             
               & Run volume registrations and the motion correction pipeline on the ADNI images.                                  \\
               & Analyze corrected fetal images.                                                                                  \\
               & Begin unsupervised machine learning to identify patterns of motion in fetal, neonatal, and preadolescent images.  
\\ \hline
November 2019  & Run volume registrations and the motion correction pipeline on the phantom images.                                   \\
              & Run volume registrations and the motion correction pipeline on the adult CHD images.                                 \\
   & Continue identifying patterns of motion in fetal, neonatal, and preadolescent images. Also incorporate adult images. \\ \hline

December 2019  & Analyze corrected phantom and adult CHD images.                                                                      \\
       & Outline paper on age-based patterns of motion, descriptions of motion, and quantification of motion.                 \\
            & Identify clinically valuable outcomes to explore in conjunction to motion.                                           \\ \hline
January 2019  & \textbf{*Schedule defense meeting for March 2019.}                                                                   \\
          & Analyze associations between motion and clinical variables.                                                          \\
           & Write paper on age-based patterns of motion, descriptions of motion, and quantification of motion.                   \\ \hline

February 2019  & Finish any remaining analyses                                                                                        \\
              & Outline paper on clinical variables and motion patterns                                                              \\
             & Revise dissertation                                                                                                  \\ \hline
March 2019     & Write paper on clinical variables and motion patterns                                                                \\
              & \textbf{*Defend dissertation.}                                                                                       \\
\end{tabular}
\end{table}





%\chapter{Raw data}
%
\safebibliography{sources}          
%\safebibliography is used the same way as \bibliography, but gives pittetd a greater chance to succeed in formatting the bibliography when non-standard BibTeX styles are used.
\end{document}
