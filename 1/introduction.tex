\chapter{Introduction}

%Real goal is to develop a method of registering fetal brain and placental images so that we can further examine the relationship between placental oxygen levels and fetal brain development. Longitudinally, this technique can be used to determine how placental oxygen flow and fetal brain development impact a patient over the course of his or her life. Once the relationship between the placenta and fetal brain development is better understood, we can determine a set of neuroprotective interventions to employ for at-risk patients before they are born.

Resting-state functional magnetic resonance imaging (rs-fMRI) measures the blood oxygen level dependent signal in an organ or organ system. This property makes rs-fMRI an invaluable tool for evaluating a patient's neurodevelopmental status or examining functional networks in his brain. To gather enough data to fully evaluate these networks, a series of image volumes must be acquired over a period of several minutes. In a standard rs-fMRI, one new image volume is obtained approximately once every two to three seconds. To gather high quality data on such a short timescale, the rs-fMRI suffers from two major limitations: rs-fMR images have low physical resolution and are highly susceptible to motion. The first limitation can be addressed by obtaining an MR image with high physical resolution and registering the rs-fMRI to this structural image, but the second limitation requires the patient to remain as still as possible for the entire duration of the scan. This task is particularly difficult for populations of certain ages and populations who suffer from conditions that affect neurodevelopment. As a result, it is common for an image from a member of one of these populations to contain too much motion to be used in clinical or research applications.

Various clinical, behavioral, and technical protocols have been developed in an attempt to prevent patient motion from impacting the acquired rs-fMR image. Sedation can be used to immobilize a patient during a scan, but it requires additional personal to perform safely and it involves a great time commitment from the patient. Sedation is also not recommended for use in young children and fetal patients. Behavioral and educational techniques can be employed to prepare a patient for stressors he may experience during an rs-fMRI scan, but these approaches do not prevent the patient from moving out of boredom, discomfort, or distress. Several groups have developed techniques to compensate for motion as the image is acquired, but these techniques often require additional MR compatible equipment and can only be utilized during the scan. After a rs-fMR image is acquired, however, it is possible to reduce the positional effects of motion in the image sequence.

Many methods have been developed to mitigate the effects of motion after the rs-fMRI is acquired. While different post-acquisition motion correction pipelines utilize different processing techniques, they generally begin with global volume registration. Global volume registration is the process used to align all volumes in a rs-fMRI sequence into the same physical space. Traditionally, all volumes in the sequence are registered directly to one volume. This approach can be effective in images where the subject remains relatively still throughout the duration of the scan, but is not as successful in images containing high quantities of patient movement.

We have developed an alternative volume registration framework which takes into account the spatiotemporal relationships between sequential volumes in the rs-fMRI sequence and uses these relationships during the registration process. We have demonstrated the feasibility of this technique on a high-motion neonatal brain rs-fMRI data set and compared it to the traditional registration framework. Herein, we evaluate it further in the context of a complete motion correction pipeline across several patient populations. 


While correcting motion within an rs-fMRI is important both for clinical use and research applications, we are also interested in the motion itself. In addition to evaluating a global volume registration framework in the context of a fully motion correction pipeline, we also investigate the relationships between a patient's motion and their clinical outcomes, specifically to further the study of congenital heart disease (CHD) across the lifetime of the patient. %This analysis is valuable for each of our subject populations for different reasons. For the preadolescent patients, we are investigating the relationship between in-scan motion, neurocognitive development, and congenital heart disease status. In the fetal images, we investigate the relationship between FETAL BRAIN DEVELOPMENT AND PLACENTAL GROWTH. % need to run this part by Ashok.
%These investigations use both supervised and unsupervised machine learning techniques to determine what relationships a computer can detect between these pieces of information.


Our aims for this project are as follows:
\begin{itemize}
\item \textbf{Aim 1.} Evaluate the impact of global volume registration within a complete motion correction pipeline in simulated and clinical data.
\item \textbf{Aim 2.} Study the motion patterns in the different populations to formally describe age-group or clinical status related motion patterns.
\item \textbf{Aim 3.} Employ machine learning techniques to (a) measure the impact of motion on image harmonization in multi-center studies, and (b) evaluate the relationship between motion and cognitive, clinical, and behavioral outcomes of CHD patients.
\end{itemize}


We have a large set of neurological rs-fMRIs for both healthy control and CHD neonatal, preadolescent, and adult subjects. We also have a set of neurological and placental rs-fMRIs for fetal patients. We will apply both the tradition and novel registration frameworks to all images in our different cohorts and evaluate the impact of each framework on each image after passing it through a complete motion correction pipeline. The original and registered images will be used to address the aims discussed in this chapter.