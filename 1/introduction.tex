\chapter{Introduction}

Resting-state functional magnetic resonance imaging (rs-fMRI) measures the blood oxygen level dependent signal in an organ or organ system. This property makes rs-fMRI an invaluable tool for evaluating a patient's neurodevelopmental status or examining functional networks in his brain. To gather enough data to fully evaluate these networks, a series of image volumes must be acquired over a period of several minutes. In a standard rs-fMRI (?), one new image volume is obtained approximately once every two to three seconds. To gather high quality data on such a short timescale, the rs-fMRI suffers from two major limitations: rs-fMR images have low physical resolution and are highly susceptible to motion. The first limitation can be addressed by obtaining an MR image with high physical resolution and registering the rs-fMRI to this structural image, but the second limitation requires the patient to remain as still as possible for the entire duration of the scan. This task is particularly difficult for populations of certain ages or populations who suffer from conditions that affect neurodevelopment. As a result, it is common for an image from a member of one of these populations to contain too much motion to be used in clinical or research applications.

Various behavioral and XX protocols have been developed in an attempt to prevent patients from moving during MRI scans, though many of these protocols are not applicable to younger populations. In particular, a neonate or fetus cannot understand instructions to stay still, and young children who can understand the command have difficulty following it. Sedation is not advisable for these young populations. After a rs-fMR image is acquired, however, it is possible to reduce the positional effects of motion in the image sequence.

Limitations of traditional methods

DAG-based registration

Apply DAG-based registration to neonates and preadolesents

Real goal is to develop a method of registering fetal brain and placental images so that we can further examine the relationship between placental oxygen levels and fetal brain development. Longitudinally, this technique can be used to determine how placental oxygen flow and fetal brain development impact a patient over the course of his or her life. Once the relationship between the placenta and fetal brain development is better understood, we can determine a set of neuroprotective interventions to employ for at-risk patients before they are born.