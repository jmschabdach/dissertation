\chapter{DATA}

The data used to test the hypothesis and aims introduced in the previous chapter are drawn from three subject populations: neonatal subjects, preadolescent subjects, and fetal subjects.

All data was obtained through studies approved by the IRB at the Children's Hospital of Pittsburgh of UPMC and the University of Pittsburgh. The data is stored and accessed in compliance with all HIPPA policies.

\section{Neonatal Subjects}

\subsection{Subject Population}

Neonatal subjects are recruited as part of a prospective observational study. The subjects were scanned using a 3T Skyra (Siemans AG, Erlangen, Germany). They were unsedated during the scans and a ``feed and bundle'' protocol was used to prevent motion during the scans \cite{Windram2011}. The newborns were positioned in the coil to minimize head tilting. Newborns were fitted with earplugs (Quiet Earplugs; Sperian Hearing Protection, San Diego, CA) and neonatal ear muffs (MiniMuffs; Natus, San Carlos, CA). An MR-compatible vital signs monitoring system (Veris, MEDRAD, Inc. Indianola, PA) was used to monitor neonatal vital signs. All scans were performed using a multi-channel head coil. The parameters for the resting-state BOLD MR scans were FOV=240 mm and TE/TR=32/2020 ms with interplane resolution of 4x4 mm, slice thickness of 4 mm, and 4 mm space between slices. The acquired images contained 150 volumes where each volume consisted of 64x64x32 voxels$^3$.

\subsection{Data Processing}

For each subject in a cohort of 74 healthy neonatal subjects, at least one rs-fMRI was acquired. The images underwent motion correction using a pipeline developed by Power et al. \cite{Power2014}. The motion corrected image were compared to Power et al.'s thresholds of acceptability for their FD and DVARS metrics \cite{Power2014}. Of the 74 subjects in this cohort, 17 subjects had rs-fMRIs which did not meet Power et al.'s usability criteria. These high motion images were used in our study.

\section{Preadolescent Subjects}

\subsection{Subject Population}

\subsection{Data Processing}

\section{Fetal Subjects}

\subsection{Subject Population}

\subsection{Data Processing}