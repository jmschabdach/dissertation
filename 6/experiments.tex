\chapter{AIMS}
\label{ch:experiments}

\section{Volume Registration In A Motion Correction Pipeline}

All images in each data set and cohort underwent both types of registration independently. The registered and original images are compared to the Power et. al. usability thresholds. The results at this stage answer the question of whether or not the DAG-based registration technique is more effective than the traditional registration technique for reducing motion in the initial step of a motion correction pipeline.  

Next, each pair of registered images will undergo a motion correction via an independent component analysis (ICA) pipeline outlined by Beckmann and Smith and implemented as FMRIB's MELODIC tool \cite{Beckmann2004}. The results of this experiment show how the DAG-based framework fits into an existing, comprehensive motion correction pipeline. 

\subsection{Simulated Phantom}

The phantom experiments will be used to probe the volume registration technique. By applying the DAG-based and traditional registration techniques to the base phantom sequence, we will be able to evaluate the degrees of positional and signal change errors each technique may introduce into the registration process. After determining the baseline error, we will apply both registration techniques to the BOLD phantom sequence. The registered versions of the BOLD phantom sequence will be compared to each other and to the original BOLD phantom sequence to determine how well each registration retains the BOLD signal.

This particular experiment will be one of the first to investigate how much true BOLD signal is preserved through motion correction. One of the major drawbacks to existing motion correction pipelines is that they remove signal along with noise. In clinical data, there is no way to know the ground truth signal contained within the image; however, simulated phantom images have a de facto known ground truth signal. The design for this experiment can be used to evaluate how much BOLD signal is recovered by other motion correction pipelines, and how close the recovered signal is to the signal of interest.

\subsection{Human Phantom}

The human phantom images from all sites will be used both as a set of true healthy control adults and as examples of low motion images of the same subjects taken at multiple sites.

\subsection{Clinical Images}
%For each subject in a cohort of 74 healthy neonatal subjects, at least one rs-fMRI was acquired. The images underwent motion correction using a pipeline developed by Power et al. \cite{Power2014}. The motion corrected image were compared to Power et al.'s thresholds of acceptability for their FD and DVARS metrics \cite{Power2014}. Of the 74 subjects in this cohort, 17 subjects had rs-fMRIs which did not meet Power et al.'s usability criteria. These high motion images were used in our study.

\textbf{Neonatal Cohort.} Our set of neonatal subjects includes a cohort of 74 healthy neonates. Each subject in this cohort underwent an MRI scan, and the rs-fMRIs obtained during this process were compared to Power et al.'s positional and signal change usability thresholds. Of the 74 subjects, 17 of them had rs-fMRIs which did not meet the usability criteria. These high motion images were used to test the feasibility of the DAG-based volume registration framework. 

These images were ideal for the feasibility study for three reasons. First, the neonates were healthy, which eliminates disease status as a confounding variable in the analysis of the registered images. Second, the neonates in this study were scanned using a feed and sleep protocol. Because the neonates were asleep during the scan, they generally did not move very much. The high-motion neonates are an obvious exception to this concept, but many of the high-motion images contained long periods where the subject was stationary. Evaluating the DAG-based framework on data with various patterns of motion and different periods of low and high motion allowed us to explore the effects of the DAG-based algorithm in different combinations of motion features. Third, these images were too corrupted by motion to be used in other analyses. Applying both the DAG-based framework and the traditional registration framework to these images provided the opportunity to compare the performances of both registration frameworks to each other in the context of the usability gold standard thresholds. 

\textbf{Preadolescent Cohort.} The multicenter imaging study of preadolescent subjects provides a unique opportunity to evaluate the efficacy of the DAG-based framework on a large subject cohort containing variable amounts of motion. The outcome of this experiment will be used in the next experiment to determine if there are any site-specific or vendor-specific variables influencing patient motion.

\textbf{Adult Cohorts.} The adult cohorts encompass many clinical outcomes and a wider age range than the other clinical populations. 

\textbf{Fetal Cohort.} As the fetal subjects have both neurological and placental images, their data will be used to examine the impact of volume registration on different organ types.

\section{Describing Motion Patterns}

In previous sections we describe practical knowledge about how fetal, neonatal, preadolescent, and adult patients move in different ways during MRI scans. We wish to quantify these movement patterns using the metrics described in Chapter \ref{ch:methods} and identify appropriate terminology that can be used to describe them.

\section{Identifying Motion Patterns}

Machine learning techniques can be used to classify images as belonging to different groups, but many of these techniques use difficult to interpret ``black box'' logic. In some cases, examining the logic behind a classification reveals patterns in a dataset which a human missed but a computer detected. These patterns can be helpful for improving human classification of the images, but they may also be based on artifacts which were not filtered out during preprocessing.

\subsection{Unsupervised Learning Techniques}

Unsupervised machine learning techniques use the features of a collection of data samples to divide the data samples into groups of similar samples. They are considered unsupervised 

Agglomerative clustering

Spectral clustering

Support vector machine

\subsection{Supervised Learning Techniques}

Supervised learning techniques use features of a collection of data samples

Regression: logistic to identify healthy vs. CHD, linear to predict clinical values

K-nearest neighbors to identify patients with similar motion patterns







\subsection{Demographic-Related Motion Patterns} % in different populations to formally describe age-group or clinical status related motion patterns.

%The purpose of this experiment is to address the following questions. Are there any patterns in motion that are similar (a) within age groups, (b) within groups scanned at the same site, or (c) within broad clinical groups? Are these patterns due to spurious signals?

To ensure that there are no confounding signals in our datasets, we first use unsupervised machine learning techniques to identify correlations between subject images and their demographic data. The techniques we will use are several types of clustering (agglomerative, k-means, and spectral) as well as principle component analysis (PCA) and regression. Features of the images before and after registration will be used as training data for each model and different demographic features will be used as the true classes. %The demographic data for each subject includes the subject's age at the time of scan, gender, race, dominant hand, and scan site. (NOTE: THAT SENTENCE IS FOR MULTISITE STUDY DATA, NEED TO SPECIFY, ALSO NEED TO GET ALL CLINICAL DATA.)

\textbf{Phantom Images.} The phantom images are included in this analysis, though no significant results are expected other than potential site specific results.

\textbf{Clinical Cohorts.} Any demographic features which influence the division of patients into groups will be reported and accounted for during later analyses. After identifying and accounting for demographic groups, we will expand the analysis to clinical and behavioral outcomes.


\subsection{Clinical-Related Motion Patterns} %Employ machine learning techniques to (a) measure the impact of motion on image harmonization in multi-center studies, and (b) evaluate the relationship between motion and cognitive, clinical, and behavioral outcomes of CHD patients.

In addition to evaluating the effects of the DAG-based framework within the context of a motion correction pipeline, the registered images are used to explore the relationship between motion and clinical outcomes. Unsupervised machine learning techniques such as agglomerative clustering and k-means clustering are applied to the data. The results of the clustering techniques elucidate whether there are patterns in motion specific to certain patient groups. These groups could include patients with similar clinical outcomes, patients from the same site, or potentially other clinical or demographic groups.

%\section{Other potential areas I'm thinking about}

%\textbf{Machine Learning for Optimal Motion Correction} Start with a classification module for identifying severity of motion between template volume, previous volume(s), and current volume. The classifications will be based either on the patterns identified in Aim 2, or on the positional and signal change differences between the volumes of interest.

%After the severity of the motion reflected in a volume is determined...

%\textbf{Aim 4: Does Motion Correction Recover True Signal?} Hinted at earlier in first section of chapter, should it get its own section?