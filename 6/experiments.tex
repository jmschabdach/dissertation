\chapter{EXPERIMENTS}

\section{Experiments}

\subsection{Simulated Phantom}

The phantom experiments will be used to probe the volume registration technique. By applying the DAG-based and traditional registration techniques to the base phantom sequence, we will be able to evaluate the degrees of positional and signal change errors each technique may introduce into the registration process. After determining the baseline error, we will apply both registration techniques to the BOLD phantom sequence. The registered versions of the BOLD phantom sequence will be compared to each other and to the original BOLD phantom sequence to determine how well each registration retains the BOLD signal.


\subsection{Feasibility Experiment: Neonatal Subjects}

%For each subject in a cohort of 74 healthy neonatal subjects, at least one rs-fMRI was acquired. The images underwent motion correction using a pipeline developed by Power et al. \cite{Power2014}. The motion corrected image were compared to Power et al.'s thresholds of acceptability for their FD and DVARS metrics \cite{Power2014}. Of the 74 subjects in this cohort, 17 subjects had rs-fMRIs which did not meet Power et al.'s usability criteria. These high motion images were used in our study.

Our set of neonatal subjects includes a cohort of 74 healthy neonates. Each subject in this cohort underwent an MRI scan, and the rs-fMRIs obtained during this process were compared to Power et al.'s positional and signal change usability thresholds. Of the 74 subjects, 17 of them had rs-fMRIs which did not meet the usability criteria. These high motion images were used to test the feasibility of the DAG-based volume registration framework. 

These images were ideal for the feasibility study for two reasons. First, the neonates were healthy, which eliminates disease status as a confounding variable in the analysis of the registered images. Second, the neonates in this study were scanned using a feed and sleep protocol. Because the neonates were asleep during the scan, they generally did not move very much. The high-motion neonates are an obvious exception to this concept, but many of the high-motion images contained long periods where the subject was stationary. Evaluating the DAG-based framework on data with various patterns of motion and different periods of low and high motion allowed us to explore the effects of the DAG-based algorithm in different combinations of motion features. Third, these images were too corrupted by motion to be used in other analyses. Applying both the DAG-based framework and the traditional registration framework to these images provided the opportunity to compare the performances of both registration frameworks to each other in the context of the usability gold standard thresholds. 

\subsection{Motion Correction and Evaluation Experiments: Preadolescent and Adult Subjects}

The multicenter imaging study provides a unique opportunity to evaluate the efficacy of the DAG-based framework on a large subject cohort. The images from all sites first undergo both types of registration independently. The registered and original images are compared to the Power etl. al. usability thresholds. The results at this stage answer the question of whether or not the results of the feasibility study can generalize to a larger cohort of subjects from a different clinical population. Next, each pair of registered images will undergo a complete motion correction pipeline. We use the ICA pipeline outlined by XXX and implemented in XXX tool. The results at this stage of the experiment show how the DAG-based framework fits into an existing, comprehensive motion correction pipeline. 

In addition to evaluating the effects of the DAG-based framework within the context of a motion correction pipeline, the registered preadolescent images are used to explore the relationship between motion and clinical outcomes. Unsupervised machine learning techniques such as agglomerative clustering and k-means clustering are applied to the data. The results of the clustering techniques elucidate whether there are patterns in motion specific to certain patient groups. These groups could include patients with similar clinical outcomes, patients from the same site, or potentially other clinical or demographic groups.

\subsection{Fetal Motion: Fetal Brain and Liver}


\section{Aim 1: DAG-based Image Registration}

Pending paper: 

\section{Aim 2: Detecting Population-Dependent Patterns of Motion}

Machine learning techniques can be used to classify images as belonging to different groups, but many of these techniques use difficult to interpret ``black box'' logic. In some cases, examining the logic behind a classification reveals patterns in a dataset which a human missed but a computer detected. One example is a system trained to differentiate between pictures of red foxes and pictures of artic foxes. Its classification accuracy was XX\%, but further examination revealed that the system used the presence of snow in an image led the system to label that image as containing an artic fox. CITATION NEEDED

To ensure that there are no confounding signals such as those in the previous example are present in our datasets, we first use unsupervised machine learning techniques to identify correlations between subject images and their demographic data. The techniques we will use are several types of clustering (agglomerative, k-means, and spectral) as well as principle component analysis (PCA) and regression. Features of the images before and after registration will be used as training data for each model and different demographic features will be used as the true classes. The demographic data for each subject includes the subject's age at the time of scan, gender, race, dominant hand, and scan site. (NOTE: THAT SENTENCE IS FOR MULTISITE STUDY DATA, NEED TO SPECIFY.)

Any demographic features which influence the division of patients into groups will be reported and accounted for during later analyses.

After identifying and , clinical, and behavioral outcome data. 

Use pediatric, neonatal, and fetal images

Proposed topic for a paper: different subject populations exhibit different patterns of motion according to age and clinical factors. 

\section{Aim 3: Machine Learning for Optimal Motion Correction}

Start with a classification module for identifying severity of motion between template volume, previous volume(s), and current volume. The classifications will be based either on the patterns identified in Aim 2, or on the positional and signal change differences between the volumes of interest.

After the severity of the motion reflected in a volume is determined, 

\section{Aim 4: Does Motion Correction Recover True Signal?}


\section{Tools}

Cite nipypy, ANTs, FSL, etc. here

\section{Metrics and Analyses}

Power et al. thresholds

Correlation ratio matrix

Dice coefficients?