\documentclass[12pt]{article}

\begin{document}

\title{Multiorgan Volume Registration}
\author{Jenna Schabdach, Dr. Rafael Ceschin, Dr. Ashok Panigrahy}
\maketitle

\begin{itemize}
\item Motion is a problem in resting state fMRIs
\item Various protocols have been developed to prevent motion in neonatal, pediatric, and adult patients, but these protocols are not compatible for fetal patients
\item A motion monitoring software tool has been developed by X for the purpose of evaluating patient motion during an MRI; however, it only monitors and does nothing to actively correct the effects of motion
\item A number of post acquisition motion correction pipelines exist, and they all have their own strengths and limitations
\item Note that all motion correction pipelines begin with volume registration
\item Traditional volume registration chooses one volume in the image sequence as the template volume and aligns all volumes to it. This technique can result in failed volume registrations in image sequences containing large amounts of motion.
\item Recently, we proposed a volume registration method that accounts for spatio-temporal relationships between neighboring volumes in an image sequence.
\item We applied this technique to a set of (description of SVR subjects here)
%\item Rather than propose a new post acquisition motion correction pipeline, we choose to use a novel volume

\end{itemize}

%Breast cancer is the second leading cause of death for women in the United States. In 2016 alone, over 250,000 new cases will be diagnosed. The current standard of care for breast cancer screening is primarily dependent on mammography. Magnetic resonance imaging (MRI) is extremely valuable for diagnosing patients who have dense or scarred breast tissue; however, the integration of MRI is often not covered by insurance for screening purposes and is, therefore, often not conducted. Radiographic alternatives to MRI are avoided to prevent risks associated with increased radiation exposure. Volumetric ultrasonic imaging, while inadequate for standalone screening, is a potential solution for complex screening cases if the modality can be fused with mammography projections. The additional information provided by volumetric ultrasound allows for clearer discrimination between soft tissues possessing similar radiographic densities. This project aims to fuse volumetric ultrasound images with mammography scans by developing a new registration algorithm. This algorithm first reduces the dimensionality of the volumetric ultrasound to that of the target mammography projection. The ultrasound projections are then used to align the ultrasound volume and mammography images in three dimensional space and account for differences in deformation between the two modalities by framing the registration as an optimization problem.

\end{document}
