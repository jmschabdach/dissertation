\chapter{DISCUSSION}
\label{ch:discussion}

\section{Volume Registration}

\subsection{Simulated Cohort}

\subsection{Preadolescent Cohort}

\subsection{Neonatal Cohort}

\subsection{Fetal Cohort}

The fetal scans were manually segmented, the masks used in the segmentation were created to be uniform across the whole sequence. The intention behind this process is to remove all voxel values not associated with the organ of interest. However, fetal motion is highly variable. It is possible for a subject to rotate in any direction. The subject may drastically change position in the middle of the scan, possibly several times. The manually created masks were developed using a software tool which allows 3D image masks to be applied to an entire 4D image sequence. The masks were required to be created to ensure the fetal brain or placenta would be inside the masked area at all times. The masks may cover some area that does not belong to the organ of interest.

This limitation is unique to the fetal images. Existing pre-processing pipelines exist for skull stripping for neonatal and preadolescent images. Development of a similar pipeline for fetal images, while a challenge, would make research surrounding fetal rs-fMRIs more accessible to the medical imaging community.

Alternatively, the field of computer vision offers some SOMETHING.

\textbf{Registration fixes positional effects of motion, not spin history or susceptibility effects}
Subject motion during rs-MRI scans affects both the recorded position and orientation of the subject as well as the established magnetic spin gradients within the skull. The DAG-based technique can correct the positional effects of motion, but it cannot correct the effects of the motion that disrupt the magnetic spin gradients. Methods for prospectively estimating subject motion exist and can be used to change slice positions in each volume during acquisition. Retrospective techniques to correct for this effect will require shot-to-shot modeling of macroscopic $B_0$ fields and are beyond the scope of the present research.

\section{Characterizing Motion}

The models used to characterize motion were able to identify the differences between general patient age group based on the metrics used to measure motion in the original sequences.

\subsection{Limitations}

\subsection{Future Work}

Additional analyses could be performed to further evaluate the computer detectable differences in patient groups. The models presented in the previous chapter were generated each using a single metric type. Each metric only measures one property of the image volumes. It is possible that combinations of metrics measuring different properties could be used to better separate patient groups. For example, the combination of the FD values which measure the positional changes due to motion and the DVARS values which measure overall signal changes could be combined to comprehensively categorize subjects based on the effects of motion, BOLD signal change, and background noise.



\section{Relation to Existing Work}

To the best of our knowledge, the only other study that has used a variant of the DAG-based method was performed by Liao et al \cite{Liao2016}. Liao et al’s dataset consisted of 10 fetal rs-fMRIs. In each of these sequences, the fetal brain, fetal liver, and placenta were manually segmented in the first volume of the sequence as well as in five other randomly chosen volumes. These overlap of these manual segmentations before and after registration as measured using the Dice coefficient was used to quantify the amount of motion in each sequence. Even though the Dice coefficients increase more in each sequence after Liao et al.’s registration than after traditional registration, their measure of positional change fails to quantify any changes in position between any other pairs of volumes that do not have manual segmentations. 

Satterthwaite et al. note that motion is often correlated with patient age in adolescent populations and specifically designed a study of adolescents ages 8-23 such that patient age and motion were uncorrelated (Satterthwaite 2012a?).