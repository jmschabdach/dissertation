\chapter{Background}
The topics treated in this chapter can be somewhat obscure. For humanitarian considerations, the chapter will be subdivided.

\section{RESTING-STATE NETWORKS}

The idea of a neuronal network which operated when a person is at rest was proposed in 2001, and then confirmed in 2003 \cite{Raichle2001} \cite{Greicius2003}. Resting-state networks are recorded using resting-state functional magnetic resonance images (rs-fMRIs). rs-fMRIs are sequences of image volumes acquired over a period of a few minutes while the patient is in a task-free state. The image volumes themselves have relatively low spatial resolution when compared to structural MRIs, but their temporal resolution is significantly higher as a new volume is acquired every two to three seconds. Each volume records the blood oxygen level dependent (BOLD) signals within the brain at that point in time. 

The BOLD signals in rs-fMRI image sequences are analyzed using a process called functional connectivity analysis. Functional connectivity analysis identifies patterns and networks of brain activity. Because the patient is not performing a specific task during a rs-fMRI acquisition, these resting-state networks have the potential to reveal valuable information about a patient's neurodevelopmental status. Some functional connectivity analysis studies have lead to the discoveries of links between specific disruptions in these naturally occurring networks and neurodevelopmental diseases such as autism and attention deficit hyperactivity disorder \cite{Assaf2010} \cite{Zang2007}. With further refinements of both acquisition techniques and characterization of these functional networks, clinicians may be able to use rs-fMRI in early detection protocols to evaluate the neurodevelopmental status of infants and neonates, and in personalized care by identifying patients who may benefit from certain therapies or neuroprotective interventions.

\section{MEASURING THE EFFECTS OF MOTION}

Due to their low spatial and high temporal resolutions, rs-fMRIs are highly susceptible to motion. Even the smallest movement can alter the position of the patient enough to cause the voxels to record signals from different brain regions and tissue types. Even if the movement does not significantly change the recorded position of the subject, it impacts the established spin gradients, which introduces artifacts into the image sequence. Movements cause the orientation of existing spin gradients to change, and the gradients require time to realign to the magnetic field. This recovery time often results in a decrease in the global signal in frames obtained over the following 8-10 seconds, which can affect the functional connectivity analysis \cite{Power2014}.

The effects of motion on rs-fMRIs can be clearly divided into two categories: the effect on patient position and the effect on the recorded BOLD signal.

The effect of motion on patient position is measured in terms of the difference in position between temporally neighboring image volumes. The difference in position is determined using metrics calculated by performing rigid volume registration on the two volumes. 
In rigid volume registration, one volume is chosen as the reference volume and the other is considered the moving volume. The reference volume remains stationary while the moving volume is translated and rotated in three-dimensional space on top of it. The registration is considered successfully complete when the position of the patient in the moving volume matches the position in the reference volume. The three translation and three rotation parameters used to achieve this alignment are used to calculate the positional change between the image volumes, which is often called the framewise displacement (FD). 

Several researchers have proposed different methods for calculating the FD. Power et al., Jenkinson et al., and Dosenbach et al. each propose a slightly different method for calculating the FD \cite{Power2012} \cite{Jenkinson2002} \cite{Dosenbach2017}. All three FD calculations produce correlated metrics: the FD metric proposed by Power et al. produces measurements approximately twice as large as the metric proposed by Jenkinson et al., and Dosenbach et al. reported a high correlation between their FD and Power’s FD \cite{Yan2013a} \cite{Dosenbach2017}. 

The effects of motion on the BOLD signal are a little more difficult to measure. They occur because motion disrupts the magnetic spin gradients present in the patient during the scan. The spin gradients need time to recover to the correct magnetic field orientation, and up to eight to ten seconds may pass before the recovery is complete \cite{Power2014}. While the spin gradients are reorienting, the recorded BOLD signal may vary between temporally neighboring volumes. These changes can be measured using the temporal derivative of the variance in the BOLD signal intensity (DVARS) between the  frames \cite{Power2012}.

Even though the effects of motion on the patient position and the recorded signal can be measured, we still need gold standard criteria to determine whether an image containing motion can be used. Patients move slightly due to breathing and cardiac function, and the BOLD signal naturally fluctuates over time. Some motion is expected; however, we need to know how much motion can be present in the image before it is considered to be corrupted by it. Power et al. established thresholds for FD and DVARS to determine the usability of a pair of images:
\begin{itemize}
\item FD less than or equal to 0.2 mm from previous volume, and
\item DVARS less than or equal to 25 units on a normalized scale of [0, 1000] signal units \cite{Power2014}
\end{itemize}

Image volumes that meet these criteria are considered to be low-motion. van Dijk et al. established that approximately five minutes of low-motion data is sufficient for use in functional connectivity analysis \cite{VanDijk2012}. Unfortunately, it is often difficult to obtain enough low-motion data from patients to use in these analyses.

\section{MOTION PREVENTION}

Various techniques and protocols have been developed to prevent patients from moving during the image acquisition process. Not all of these techniques are suitable for all patient populations, and some techniques have been designed specifically for certain populations populations.

\subsection{Sedation}

Sedation can be used to help a patient tolerate an MRI scan. Murphy and Brunberg retrospectively analyzed seven weeks of data from the MR department and found that 14.2\% of their adult patients some form of sedation \cite{Murphy1997}. In a study about claustrophobia and MR acquisitions, ELEPHANTS report that out of 55,734 patients who underwent MRI scans, a total of 1,004 patients experienced claustrophobia and 610 of these patients required intravenous sedation before their scans \cite{Dewey2007}. Even though sedation allowed the patients mentioned in this paragraph to undergo an MRI scan, the authors of both studies note that sedation can result in adverse events and advise the reader to avoid patient sedation if possible.


Sedation can be used with pediatric patients, though the risks are more significant than with adult patients. Studies have shown that sedation for pediatric imaging can lead to hypoxemia and inappropriate sedation levels during image acquisition \cite{Malviya2000}. Some pediatric patients can also expect ``motor imbalance and gastrointestinal effects,'' as well as agitation and restless for a period of hours after waking from sedation.

A report from the American Academy of Pediatrics and the American Academy of Pediatric Dentistry outlines the minimum set of criteria needed for a pediatric patient to be sedated for a procedure \cite{Cote2016}:
\begin{itemize}
\item The patient must be a suitable candidate for sedation based on their medical history and medical needs.
\item At least one responsible person must be with the patient at the medical factility, though the report recommends that two adults are present for patients who use car seats to travel to and from the facility. This practice ensures that one adult can monitor the patient after the procedure while the other adult drives.
\item The clinician administering the sedation must have immediate access to emergency facilities, personnel, and equipment and should monitor the patient for adverse events including respiratory events, seizures, vomiting, and allergic reactions.
\item There must be a clear protocol outlined for immediate access to these emergency services.
\item Emergency equipment and drugs appropriate for the patient's size and age must be immediately available in case the patient needs to be resuscitated.
\item Informed consent must be obtained prior to the procedure.
\item Instructions for what to expect and how to transport the patient home safely must be provided to the patient's responsible adult.
\item The patient may be held at the facility for prolonged monitoring after the procedure.
\item The patient's food and drink intake prior to the procedure should be taken into account to minimize the risk of pulmonary aspiration.
\item The patient's health status must be evaluated and verified by the sedation team prior to the procedure.
\item The information about the procedure must be correctly documented.
\item The facility should have a dedicated recovery area, and the status of the patient should be recorded when he is discharged. The patient should not be discharged if his level of consciousness and oxygen saturation do not meet recognized guidelines.
\end{itemize}
\noindent This report clearly states that the levels of monitoring suggested within should serve as minimum levels of involvement: clinicians should increase patient monitoring as needed for complex cases. Rutman has a similar and detailed perspective on patient monitoring during and after sedation, suggesting that two independent medical personnel should be present during the scan and one should be present until the patient is discharged \cite{Rutman2009}. Rutman also notes that all sedation and monitoring equipment must be MR compatible, which is a simple but important safety constraint. This constraint may make sedation less advisable if the appropriate equipment is not available.

Sedation in neonatal and infant populations is not recommended. The  U.~S.~Food and Drug Administration (FDA) issued a warning in late 2016 about repeated use of sedation or general anesthesia in patients under three years of age or in pregnant women in their third trimester \cite{FDA2016}. The warning states that while a single, relatively short exposure to sedative and anesthetic drugs is unlikely to impact the patient, the effects of prolonged exposure to these drugs are still being studied. Studies of sedative and anesthetic drugs in multiple animal models have shown that these drugs can lead to loss of nerve cells in the brain when the animals undergo prolonged, repeated exposure to them during period of brain development. More data is needed to determine if this effect translates to humans.

\subsection{Education, Distraction, and Behavioral Techniques}

Educational material can be used to help the patient understand what to expect during an MRI scan as well as to teach the patient different behavioral coping strategies. The education materials can be used either before arrival at the imaging facility or upon arriving at the imaging facility. 

% Adult patients
Most of the formal literature focuses on educational, distraction, and behavioral techniques to use during pediatric MRI scans. Many of the following approaches could be adapted for use with adults.

% Pediatric patients
In a review of the available literature, Alexander found several commonly used techniques to educate, comfort, and distract pediatric patients during radiology procedures. Tools such as educational coloring books and short videos can expose patients to the types of equipment they can expect to see using a familiar, engaging medium. Pediatric patients can learn coping strategies to employ during the scan such as breathing techniques, imagery, and positive statements. Alexander notes that allowing a pediatric patient to choose a behavioral coping strategy gives the patient a sense of control and may encourage the patient to cooperate during the MRI acquisition \cite{Alexander2012}.

Mock scanners and MRI simulators can also help the patient feel more comfortable during the scan. Barnea-Goraly et al. showed that both a commerical MRI simulator and a low-tech mock scanner desensitized pediatric patients between four and ten years of age to the MRI scanner with the results that 92.3\% of the acquired images could be used in high-resolution anatomical studies \cite{Barnea-Goraly2014}. 

% distraction
During the MRI acquisition, headphones with music or stories and MR compatible video goggles can distract patients \cite{Alexander2012} \cite{Barnea-Goraly2014} \cite{Harned2001}. Garcia-Palacios et al. performed a case study comparing the efficacy of music and immersive virtual reality tools as distractions during a mock scan \cite{Garcia-Palacios2007}. They suggest that immersive virtual reality may help decrease patient anxiety during a scan more effectively than music alone. %As current virtual reality technology improves, it may join headphones and MR compatible video goggles as an available distraction method.

Another source of distraction for pediatric patients could be the patient's parent or parents. Having a parent involved with the scanning process may calm the patient and encourage him to cooperate; however, patients can become distressed if their parent is distressed \cite{Alexander2012}. 

% Neonates


%Education techniques can be used with pediatric and adult patients to help the patients learn what to expect during a scan, though it does not prevent the patient from moving during the scan itself [Alexander (2012); Barnea-Goraly et al. (2014)].

%Playing videos or audio, or allowing the patient to use a virtual reality setup during the scan can help reduce anxiety in pediatric and adult patients, though the motion reduction techniques of these distraction tactics are questionable [Harned and Strain (2001); Khan et al. (2007); Garcia-Palacios et al. (2007)].

%For scans of neonatal and pediatric patients, parental involvement can help calm the patient, but parental distress can further upset an anxious patient and complicate the scanning process [Alexander (2012)]. 

\subsection{Sleep Methods for Neonates}

%In pediatrics especially, sedation requires additional hospital personnel to observe the patient and prevent adverse effects, and it may require the patient to remain in the hospital for some time after the scan to ensure the patient does not experience any adverse events.

%Efforts have been made to develop proactive methods for mitigating the effects of motion in rs-fMRI. There are clinical protocols that can be used to try to prevent motion from occurring during acquisition, but these protocols are either not effective, not safe, or not compatible with rs-fMRI. 
% 

%A few researchers have explored the area of prospective motion correction during the image acquisition. 

%Dosenbach et al. have developed a tool to evaluate motion in rs-fMRI sequences as they are acquired [Dosenbach et al. (2017)]. It registers each frame to the initial frame of the rs-fMRI sequence immediately after the new frame is recorded. The parameters produced by this registration are used to calculate the framewise displacement between pairs of frames, which is then compared to a set of displacement thresholds associated with the scan quality. The number of frames that meet each threshold is used to determine how many more frames are needed to obtain five minutes of low-motion frames. This method for assessing the quality of a scan in real time is useful for ensuring images are acquired with a sufficient number of low-motion frames. It can also aid the technologists in determining whether to prematurely terminate a scan, which may be desirable if the amount of time needed to obtain enough low-motion frames is greater than the amount of time remaining for the patient in the scanner. However, this method cannot be used to recover motion-corrupted data in existing repositories.

\section{MOTION CORRECTION}

%Due to the inconsistency in the success of clinical protocols in preventing subject motion, the primary focus of motion mitigation research is development of post-acquisition techniques. The most commonly used techniques include global volume registration, denoising, and filtering.

%After the sequence undergoes global rigid volume registration and the motion estimates are calculated, denoising and filtering techniques are often applied. Denoising techniques consist of regressions of various confound variables. Regression of the global signal (global signal regression, GSR) corrects for variance between temporal signals within a voxel and for the mean BOLD signal across all voxels [Power et al. (2014); Satterthwaite et al. (2013); Yan et al. (2013a); Yan et al. (2013b)]. GSR has been shown to reduce spuriously increased long-distance correlations in functional connectivity studies, but may inadvertently weaken shorter-distance connections [Jo et al. (2013); Power et al. (2014); Satterthwaite et al. (2012)]. Other regression parameters that have been investigated include the six realignment parameters and their first-order derivatives [Power et al. (2012); Satterthwaite et al. (2012); van Dijk et al. (2012)], realignment parameters from surrounding timpoints [Patriat et al. (2017); Power et al. (2014); Satterthwaite et al. (2013); Yan et al. (2013b)],signals from white matter or cerebral spinal fluid [Power et al. (2014); Satterthwaite et al. (2013); Yan et al. (2013b); Jo et al. (2010)], and components identified using principal or independent component analysis [Pruim et al. (2015); Salimi-Khorshidi et al. (2014); Behzadi et al. (2007)]. Regression of each of these sets of parameters has been shown to reduce the effects of motion in the sequence but not remove them entirely [Power et al. (2015); Parkes et al. (2017)]. 

%Filtering, which is also referred to as censoring, involves the identification and removal or interpolation of frames containing high quantities of motion. Two popular techniques are scrubbing and spike regression. Power et al.’s scrubbing technique removes frames with more than 0.2 mm of FD [Power et al. (2012)]. Spike regression identifies frames with large FD and replaces them with interpolated volumes [Satterthwaite et al. (2013)]. Unfortunately, these filtering techniques ultimately result in the loss of data as frames are removed from the sequence. A third technique called despiking detects signal spikes at the voxel level and interpolates over the spikes [Jo et al. (2013); Patel et al. (2014)]. Despiking does not remove frames, but could accidentally remove valuable signals. 

\section{VOLUME REGISTRATION}

Liao et al. suggested that a rs-fMRI sequence could be viewed as a hidden Markov model, and reflected this idea in their suggested registration framework \cite{Liao2016}. Their framework uses the transformation of the previous volume to the reference volume as the initial transformation for the current volume and the reference volume. 



%\section{MAGNETIC RESONANCE IMAGING}
%
%Magnetic resonance imaging is a technique used in medicine to obtain in vivo images of different organ systems. The patient is placed in a static magnetic field and electromagnetic pulses are used to briefly induce a secondary magnetic field.
%
%One type of MRI is resting state functional MRI (rs-fMRI). rs-fMRI is 
%
%
%\section{THE CHALLENGE OF MOTION}%            Remember to capitalize the sections (otherwise, the bookmark will be lowercase)
%
%Patient movement is a challenge across all modalities of medical imaging, but rs-fMRI is particularly sensitive to it. Various preventative, 
%
%An rs-fMRI is considered usable if it contains at least 5 minutes of low motion data.
%
%\subsection{•}
%
%
%
%\subsection{First subsection of the section}
%This is well-known topic, and we shall discuss it no more.
%\subsection{Second subsection of the section}
%This is a very complicated topic and we shall discuss it in our next paper.\textsl{•}
