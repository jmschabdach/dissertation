\chapter{Background}
The topics treated in this chapter can be somewhat obscure. For humanitarian considerations, the chapter will be subdivided.

\section{RESTING-STATE NETWORKS}

The idea of a neuronal network which operated when a person is at rest was proposed in 2001, and then confirmed in 2003 \cite{Raichle2001} \cite{Greicius2003}. Resting-state networks are recorded using resting-state functional magnetic resonance images (rs-fMRIs). rs-fMRIs are sequences of image volumes acquired over a period of a few minutes while the patient is in a task-free state. The image volumes themselves have relatively low spatial resolution when compared to structural MRIs, but their temporal resolution is significantly higher as a new volume is acquired every two to three seconds. Each volume records the blood oxygen level dependent (BOLD) signals within the brain at that point in time. 

The BOLD signals in rs-fMRI image sequences are analyzed using a process called functional connectivity analysis. Functional connectivity analysis identifies patterns and networks of brain activity. Because the patient is not performing a specific task during a rs-fMRI acquisition, these resting-state networks have the potential to reveal valuable information about a patient's neurodevelopmental status. Some functional connectivity analysis studies have lead to the discoveries of links between specific disruptions in these naturally occurring networks and neurodevelopmental diseases such as autism and attention deficit hyperactivity disorder \cite{Assaf2010} \cite{Zang2007}. With further refinements of both acquisition techniques and characterization of these functional networks, clinicians may be able to use rs-fMRI in early detection protocols to evaluate the neurodevelopmental status of infants and neonates, and in personalized care by identifying patients who may benefit from certain therapies or neuroprotective interventions.

\section{MEASURING THE EFFECTS OF MOTION}

Due to their low spatial and high temporal resolutions, rs-fMRIs are highly susceptible to motion. Even the smallest movement can alter the position of the patient enough to cause the voxels to record signals from different brain regions and tissue types. Even if the movement does not significantly change the recorded position of the subject, it impacts the established spin gradients, which introduces artifacts into the image sequence. Movements cause the orientation of existing spin gradients to change, and the gradients require time to realign to the magnetic field. This recovery time often results in a decrease in the global signal in frames obtained over the following 8-10 seconds, which can affect the functional connectivity analysis \cite{Power2014}.

The effects of motion on rs-fMRIs can be clearly divided into two categories: the effect on patient position and the effect on the recorded BOLD signal.

The effect of motion on patient position is measured in terms of the difference in position between temporally neighboring image volumes. The difference in position is determined using metrics calculated by performing rigid volume registration on the two volumes. 
In rigid volume registration, one volume is chosen as the reference volume and the other is considered the moving volume. The reference volume remains stationary while the moving volume is translated and rotated in three-dimensional space on top of it. The registration is considered successfully complete when the position of the patient in the moving volume matches the position in the reference volume. The three translation and three rotation parameters used to achieve this alignment are used to calculate the positional change between the image volumes, which is often called the framewise displacement (FD). 

Several researchers have proposed different methods for calculating the FD. Power et al., Jenkinson et al., and Dosenbach et al. each propose a slightly different method for calculating the FD \cite{Power2012} \cite{Jenkinson2002} \cite{Dosenbach2017}. All three FD calculations produce correlated metrics: the FD metric proposed by Power et al. produces measurements approximately twice as large as the metric proposed by Jenkinson et al., and Dosenbach et al. reported a high correlation between their FD and Power’s FD \cite{Yan2013a} \cite{Dosenbach2017}. 

The effects of motion on the BOLD signal are a little more difficult to measure. They occur because motion disrupts the magnetic spin gradients present in the patient during the scan. The spin gradients need time to recover to the correct magnetic field orientation, and up to eight to ten seconds may pass before the recovery is complete \cite{Power2014}. While the spin gradients are reorienting, the recorded BOLD signal may vary between temporally neighboring volumes. These changes can be measured using the temporal derivative of the variance in the BOLD signal intensity (DVARS) between the  frames \cite{Power2012}.

Even though the effects of motion on the patient position and the recorded signal can be measured, we still need gold standard criteria to determine whether an image containing motion can be used. Patients move slightly due to breathing and cardiac function, and the BOLD signal naturally fluctuates over time. Some motion is expected; however, we need to know how much motion can be present in the image before it is considered to be corrupted by it. Power et al. established thresholds for FD and DVARS to determine the usability of a pair of images:
\begin{itemize}
\item FD less than or equal to 0.2 mm from previous volume, and
\item DVARS less than or equal to 25 units on a normalized scale of [0, 1000] signal units \cite{Power2014}
\end{itemize}

Image volumes that meet these criteria are considered to be low-motion. van Dijk et al. established that approximately five minutes of low-motion data is sufficient for use in functional connectivity analysis \cite{VanDijk2012}. Unfortunately, it is often difficult to obtain enough low-motion data from patients to use in these analyses.

\section{MOTION PREVENTION}

Various techniques and protocols have been developed to prevent patients from moving during the image acquisition process. Not all of these techniques are suitable for all patient populations, and some techniques have been designed specifically for certain populations populations.

\subsection{Sedation}

Sedation can be used to help a patient tolerate an MRI scan. Murphy and Brunberg retrospectively analyzed seven weeks of data from the MR department and found that 14.2\% of their adult patients some form of sedation \cite{Murphy1997}. In a study about claustrophobia and MR acquisitions, ELEPHANTS report that out of 55,734 patients who underwent MRI scans, a total of 1,004 patients experienced claustrophobia and 610 of these patients required intravenous sedation before their scans \cite{Dewey2007}. Even though sedation allowed the patients mentioned in this paragraph to undergo an MRI scan, the authors of both studies note that sedation can result in adverse events and advise the reader to avoid patient sedation if possible.


Sedation can be used with pediatric patients, though the risks are more significant than with adult patients. Studies have shown that sedation for pediatric imaging can lead to hypoxemia and inappropriate sedation levels during image acquisition \cite{Malviya2000}. Some pediatric patients can also expect ``motor imbalance and gastrointestinal effects,'' as well as agitation and restless for a period of hours after waking from sedation.

A report from the American Academy of Pediatrics and the American Academy of Pediatric Dentistry outlines the minimum set of criteria needed for a pediatric patient to be sedated for a procedure \cite{Cote2016}:
\begin{itemize}
\item The patient must be a suitable candidate for sedation based on their medical history and medical needs.
\item At least one responsible person must be with the patient at the medical factility, though the report recommends that two adults are present for patients who use car seats to travel to and from the facility. This practice ensures that one adult can monitor the patient after the procedure while the other adult drives.
\item The clinician administering the sedation must have immediate access to emergency facilities, personnel, and equipment and should monitor the patient for adverse events including respiratory events, seizures, vomiting, and allergic reactions.
\item There must be a clear protocol outlined for immediate access to these emergency services.
\item Emergency equipment and drugs appropriate for the patient's size and age must be immediately available in case the patient needs to be resuscitated.
\item Informed consent must be obtained prior to the procedure.
\item Instructions for what to expect and how to transport the patient home safely must be provided to the patient's responsible adult.
\item The patient may be held at the facility for prolonged monitoring after the procedure.
\item The patient's food and drink intake prior to the procedure should be taken into account to minimize the risk of pulmonary aspiration.
\item The patient's health status must be evaluated and verified by the sedation team prior to the procedure.
\item The information about the procedure must be correctly documented.
\item The facility should have a dedicated recovery area, and the status of the patient should be recorded when he is discharged. The patient should not be discharged if his level of consciousness and oxygen saturation do not meet recognized guidelines.
\end{itemize}
\noindent This report clearly states that the levels of monitoring suggested within should serve as minimum levels of involvement: clinicians should increase patient monitoring as needed for complex cases. Rutman has a similar and detailed perspective on patient monitoring during and after sedation, suggesting that two independent medical personnel should be present during the scan and one should be present until the patient is discharged \cite{Rutman2009}. Rutman also notes that all sedation and monitoring equipment must be MR compatible, which is a simple but important safety constraint. This constraint may make sedation less advisable if the appropriate equipment is not available.

Sedation in neonatal and infant populations is not recommended. The  U.~S.~Food and Drug Administration (FDA) issued a warning in late 2016 about repeated use of sedation or general anesthesia in patients under three years of age or in pregnant women in their third trimester \cite{FDA2016}. The warning states that while a single, relatively short exposure to sedative and anesthetic drugs is unlikely to impact the patient, the effects of prolonged exposure to these drugs are still being studied. Studies of sedative and anesthetic drugs in multiple animal models have shown that these drugs can lead to loss of nerve cells in the brain when the animals undergo prolonged, repeated exposure to them during period of brain development. More data is needed to determine if this effect translates to humans.

\subsection{Education, Distraction, and Behavioral Techniques}

Educational material can be used to help the patient understand what to expect during an MRI scan as well as to teach the patient different behavioral coping strategies. The education materials can be used either before arrival at the imaging facility or upon arriving at the imaging facility. 

% Adult patients
Most of the formal literature focuses on educational, distraction, and behavioral techniques to use during pediatric MRI scans. Many of the following approaches could be adapted for use with adults.

% Pediatric patients
In a review of the available literature, Alexander found several commonly used techniques to educate, comfort, and distract pediatric patients during radiology procedures. Tools such as educational coloring books and short videos can expose patients to the types of equipment they can expect to see using a familiar, engaging medium. Pediatric patients can learn coping strategies to employ during the scan such as breathing techniques, imagery, and positive statements. Alexander notes that allowing a pediatric patient to choose a behavioral coping strategy gives the patient a sense of control and may encourage the patient to cooperate during the MRI acquisition \cite{Alexander2012}.

Mock scanners and MRI simulators can also help the patient feel more comfortable during the scan. Barnea-Goraly et al. showed that both a commerical MRI simulator and a low-tech mock scanner desensitized pediatric patients between four and ten years of age to the MRI scanner with the results that 92.3\% of the acquired images could be used in high-resolution anatomical studies \cite{Barnea-Goraly2014}. 

% distraction
During the MRI acquisition, headphones with music or stories and MR compatible video goggles can distract patients \cite{Alexander2012} \cite{Barnea-Goraly2014} \cite{Harned2001}. Khan et al. found that a relatively simple moving light show can be helpful in distracting younger patients \cite{Khan2007}. Garcia-Palacios et al. performed a case study comparing the efficacy of music and immersive virtual reality tools as distractions during a mock scan \cite{Garcia-Palacios2007}. They suggest that immersive virtual reality may help decrease patient anxiety during a scan more effectively than music alone. %As current virtual reality technology improves, it may join headphones and MR compatible video goggles as an available distraction method.

Another source of distraction for pediatric patients could be the patient's parent or parents. Having a parent involved with the scanning process may calm the patient and encourage him to cooperate; however, parental distress can further upset an anxious patient and complicate the scanning process \cite{Alexander2012}. 

These techniques for educating the patient and helping the patient cope with the anxiety that can come with an MRI scan all depend on the ability of the patient to understand instructions and communicate with the scan team. Due to the gap in communication abilities, these techniques are not useful for young patients such as neonates, infants, and toddlers. Other patient populations, such as those with developmental delays and neurobehavioral disorders, may also have difficulty adhering to these protocols. Even in patients with developed and intact communication skills, the techniques outlined here do not actively prevent the patient from moving during the scan: they only help the patient feel more comfortable with the MRI environment.

\subsection{Feed and Sleep Protocols}

Neither sedation nor educational and behavioral techniques are appropriate to use with neonatal patients, but rs-fMRIs in neonates and infants are invaluable  in studying early brain development and neurological diseases \cite{Smyser2015}. A set of protocols have been developed specifically for scanning neonates without sedation. These protocols are referred to as ``feed and sleep'' or ``feed and bundle'' protocols.

Windram et al. describe a protocol in which the infant is deprived of food for four hours prior to the scan \cite{Windram2011}. At the scanning facility, the patient is fed by his mother, swaddled, and placed in a vacuum-bag immobilizer for the duration of the scan. 

Rather than deprive the patient of food prior to the scan, Gale et al.'s protocol recommends timing the scan so that the patient is fed after arrival on site and less than 45 minutes before the scan \cite{Gale2013}. The patient's ears are protected from the noise of the MR scanner by a layer of dental putty, followed by headphones, and held in place by a hat. The patient is the swaddled and placed in the scanner once he is asleep. Additional foam padding is used to cushion the patient's head and provides extra noise protection.

Mathur et al. describe a protocol similar to the previous two: the patient's feeding schedule is adjusted so that he feeds 30-45 minutes before the scan time, and he is swaddled, given ear protection, and placed in a vacuum-bag immobilizer \cite{Mathur2008}.

These protocols are generally successful: when performed correctly, the neonatal patient usually sleeps for the duration of the MRI scan. However, the patient may shift slightly while asleep or may wake up and move mid-scan.

% application of rs-fMRI to evaluate neonates neurodevelopmental status of infants and neonates, and in personalized care by identifying patients who may benefit from certain therapies or neuroprotective interventions \cite{Smyser2015}.

\section{PROSPECTIVE MOTION CORRECTION}

Since motion cannot be completely eliminated from rs-fMRI scans, different approaches have developed for correcting for the effects of motion after the scan. These approaches can be divided into two groups: those that monitor the patient's motion during the scan and those that work solely on the acquired sequences.

\subsection{Optical Motion Correction}

Several groups have developed methods for actively accounting for changes in the patient's position during an MRI scan. Optical-based methods record the patient's position using a combination of markers placed on the patient and one or more MR compatible optical cameras placed the scanner bore. The changes in the patient position from one time point to the next are used to update the MR parameters in real-time. Real-time updates of the MR parameters result in less spatial and spin-history effects of motion in the acquired sequences.

The first report of successful prospective motion correction using optical cameras and markers was by Zaitsev et al. in 2006 \cite{Zaitsev2006}. Their dual camera system was located outside of the MRI scanner and focused on the patient inside the system. Four reflective markers were attached to a modified mouthpiece originally designed for patient immobilization. Changes in the translation and rotation of the patient were recorded and processed during the exam. The processed changes were sent in real-time to the MRI scanner which used them to update the gradient orientations and RF frequencies and phases at every time point during the acquisition process.

Aksoy et al. simplify this approach by using a single in-bore optical camera and replacing the 3D markers with a small 2D chessboard grid \cite{Aksoy2008}. Properties intrinsic to the camera as well as information about the camera's placement within the MRI scanner were recorded prior as part of a calibration process. During the scan, patient movements recorded using the optical camera were used to calculate the relationship between the patient's position at the current time point in the physical space and the patient's position at the initial time point in the MR space. The transformation needed to translate between these two positions was calculated on a laptop and passed to the MRI scanner to correct for motion in real-time.

% Need one more optical example...
The camera used to record the position of the chessboard is mounted on the head coil. If the patient moves his head significantly, the camera will only be able to record the position of part of the chessboard marker. This limitation makes it difficult for the computer vision processing to identify the independent features on the standard chessboard. Forman et al. modified the chessboard marker to improve its flexibility \cite{Forman2011}. To differentiate between the different blocks in the chessboard, they added a unique, machine readable symbol to each black block in the chessboard. The symbols were chosen to be unique even in the event of rotation so that the identification of each block would be robust to rotation movements. The chessboard marker was embedded with MR-detectable agar so that the position of the marker could be detected in the MRI scan as well as by the in-bore camera. At each point during the scan, the image recorded by the in-bore camera was sent to a computer independent from the MRI controller. The independent computer detected the blocks of the chessboard and identified their spatial locations using the symbols contained within them. Their positions were checked by confirming the locations of the symbols with respect to each other. The confirmed locations of the corners of the black boxes were used to estimate the position of the patient, which was then sent to the MRI controller so that the magnetic gradients and RF hardware could be updated for the time point. The authors note that the latency of the system is a significant limitation to their system, but overall they experienced an increase in the accuracy of the estimates of the patient's position.

% Limitation: MR safe equipment
% Limitation: measurements must be made  
% Limitation: only rigid body motion. 
The methods discussed above have a few limitations due to the optical camera setups. For precise real-time motion correction, the camera or cameras must be carefully placed so that the position of the marker on the patient can be recorded. They must have a clear line of sight, which means they will be in the same room as the MRI scanner, if not within the scanner bore. The cameras and markers must be MR compatible, and the positions of the cameras and markers in physical space relative to the visual markers on the patient must be known. These positions are key for the calculations used to measure the motions. Even if the motion measurements are accurate, the changes in position that are recorded and used to adapt the scan parameters will only be true for rigid body motion of the body part to which the markers are attached: any distortion of soft tissue will not be accurately accounted for during the motion correction. 

\subsection{Within-Image Motion Correction}

Dosenbach et al. have developed a tool to evaluate motion in rs-fMRI sequences as they are acquired \cite{Dosenbach2017}. It registers each frame to the initial frame of the rs-fMRI sequence immediately after the new frame is recorded. The parameters produced by this registration are used to calculate the framewise displacement between pairs of frames, which is then compared to a set of displacement thresholds associated with the scan quality. The number of frames that meet each threshold is used to determine how many more frames are needed to obtain five minutes of low-motion frames. This method for assessing the quality of a scan in real time is useful for ensuring images are acquired with a sufficient number of low-motion frames. It can also aid the technologists in determining whether to prematurely terminate a scan, which may be desirable if the amount of time needed to obtain enough low-motion frames is greater than the amount of time remaining for the patient in the scanner. 


\subsection{General Limitations of Prospective Motion Correction}

Prospective motion correction actively changes the image as it is acquired. In order to view a scan not impacted by prospective motion correction, the patient must undergo a second scan. It may be wise to build the second image acquisition into the same scan period as the prospectively motion corrected scan: unsuccessful prospective motion correction has the potential to drastically corrupt the acquired scan \cite{Zaitsev2017}.

MacClaren et al. note that while prospective motion correction reduces imhomogeneities in the $B_0$ field, the $B_0$ field will change when the patient moves \cite{MacClaren2013}.

Both types of prospective motion correction introduce a delay into the scanning process. The delay is due to the additional processing of some metrics to determine the patient's position, the transmission of these metrics to the MR scanner, and the adjustments the scanner makes to its next set of measurements.

Though prospective motion correction has great power for managing motion during a scan, it cannot be used to recover motion-corrupted data in existing repositories.


\section{RETROSPECTIVE MOTION CORRECTION}

Many groups have put significant effort into developing techniques for motion correction after the scan is acquired. Here, we discuss several commonly techniques including global volume registration, denoising, and filtering.

\subsection{Global Volume Registration}



% A number of post-acquisition methods have been developed specifically to correct for distortions due to the impact of motion on the magnetic field. The usability of these dynamic distortion correction methods has been studied in a few specific cases, but their generalizability has yet to be confirmed in a broader range of fMRI studies \cite{Zaitsev2017}.

\subsection{Denoising}

%After the sequence undergoes global rigid volume registration and the motion estimates are calculated, denoising and filtering techniques are often applied. Denoising techniques consist of regressions of various confound variables. Regression of the global signal (global signal regression, GSR) corrects for variance between temporal signals within a voxel and for the mean BOLD signal across all voxels [Power et al. (2014); Satterthwaite et al. (2013); Yan et al. (2013a); Yan et al. (2013b)]. GSR has been shown to reduce spuriously increased long-distance correlations in functional connectivity studies, but may inadvertently weaken shorter-distance connections [Jo et al. (2013); Power et al. (2014); Satterthwaite et al. (2012)]. Other regression parameters that have been investigated include the six realignment parameters and their first-order derivatives [Power et al. (2012); Satterthwaite et al. (2012); van Dijk et al. (2012)], realignment parameters from surrounding timpoints [Patriat et al. (2017); Power et al. (2014); Satterthwaite et al. (2013); Yan et al. (2013b)],signals from white matter or cerebral spinal fluid [Power et al. (2014); Satterthwaite et al. (2013); Yan et al. (2013b); Jo et al. (2010)], and components identified using principal or independent component analysis [Pruim et al. (2015); Salimi-Khorshidi et al. (2014); Behzadi et al. (2007)]. Regression of each of these sets of parameters has been shown to reduce the effects of motion in the sequence but not remove them entirely [Power et al. (2015); Parkes et al. (2017)]. 


\subsection{Filtering}

%Filtering, which is also referred to as censoring, involves the identification and removal or interpolation of frames containing high quantities of motion. Two popular techniques are scrubbing and spike regression. Power et al.’s scrubbing technique removes frames with more than 0.2 mm of FD [Power et al. (2012)]. Spike regression identifies frames with large FD and replaces them with interpolated volumes [Satterthwaite et al. (2013)]. Unfortunately, these filtering techniques ultimately result in the loss of data as frames are removed from the sequence. A third technique called despiking detects signal spikes at the voxel level and interpolates over the spikes [Jo et al. (2013); Patel et al. (2014)]. Despiking does not remove frames, but could accidentally remove valuable signals. 

\section{VARIANTS OF GLOBAL VOLUME REGISTRATION}

Liao et al. suggested that a rs-fMRI sequence could be viewed as a hidden Markov model, and reflected this idea in their suggested registration framework \cite{Liao2016}. Their framework uses the transformation of the previous volume to the reference volume as the initial transformation for the current volume and the reference volume. 