\chapter{Neurodevelopment, Congenital Heart Disease, and Functional Connectivity}
The topics treated in this chapter can be somewhat obscure. For humanitarian considerations, the chapter will be subdivided.

\section{CONGENITAL HEART DEFECTS}

Congenital heart defects and congenital heart disease (CHD) both refer to defects in the heart or the vessels around the heart which formed during the fetal period. CHD can affect any combination of heart chambers and blood vessels with varying degrees of severity, which makes it complex to treat effectively.

Heart defects affect how blood moves into, through, and away from the heart. They can prevent the cardiopulmonary system as a whole from functioning correctly.

\subsection{From Heart Disease and Stroke Update 2016}
Children with CHD at 19-fold risk for stroke (216)

ICD-9 has 25 CHD codes, but many more lesions exist that are not well defined by the ICD-9 or ICD-10 codes due to the large variance in CHD presentations/physiology

Incidence in US: 4-10 per 1000, approximately 0.8\% of live births (10)
Incidence in Europe: 6.9/1000 live births
Incidence in Asia: 9.3 per 1000 live births (11)
Incidence metrics vary depending on how old the patient is when the CHD is detected: more severe cases present at earlier stages, but minor defects may not become apparant until the patient is an adult. As screening tools become more effective, it is expected that the incidence rates will increase as defects are detected earlier.
See (16)

% Causes of CHD: genetic syndromes, single gene mutations, environmental exposure, and unknown
Risk factors: genetic syndromes, twins, maternal exposures (smoking during first trimester, especially heavy smokers, binge drinking, binge drinking and smoking, obesity, gestational GM, Folate deficiency, high altitude), and paternal exposures (phthalates, anesthesia, sympathomimetic medications, pesticides, and solvents)

% Overall prevalence
Prevalence: Development of better medical procedures and medications have lead to a decrease in mortality of CHD patients. As of 2000, it is estimated between 800,000 and 850,000 adults and about 859,000 children in the US have CHD. In the same year, Canada reported a CHD prevalance of 11.89 per 1,000 children and 4.09 per 1,000 adults. A report from the Bhabha Atmic Research Centre Hospital in Mumbai estimates the prevalence of CHD at birth to be about 13.28 per 1,000 live births between 2006 and 2011. Information about the prevalence of CHD in other countries is sparse and therefore not reported here.

Cost: high for certain defects, medium costs for surgical interventions for other lesions, still processing

Complications: heart failure, infections?

Mortality: still processing information. Overall, the mortality for CHD patients is declining. 

% Table of some common ``forms'' of CHD and their prevalence

% When are patients diagnosed?
% Complications and risks
% Expected lifespan
% Treatment plan
% Financial burden

\subsection{From Long-Term Risk of Hemorrhagic Stroke in Young Patients with CHD}
Giang et al performed a study comparing the prevalence of cardiac conditions in patients with and without CHD born between 1970 and 1993 in Sweden. They found that patients who had a CHD diagnosis were at about eight times higher risk for intracerebral hemorrhage and subarachnoid hemorrhage than their non-CHD counterparts. The CHD patients were also more likely to suffer from arrhythmia and heart failure.

\section{CHD AND NEURODEVELOPMENT}

Recent research has found that there is a link between CHD and neurodevelopment.

To address:
\begin{itemize}
\item Common combinations
\item Joint treatment?
\item Additional risks?
\item Joint financial and emotional burden on caretakers? 
\item CHD, neuro, and aging? Dementia/Alzheimer's?
\end{itemize}

\section{RESTING-STATE NETWORKS}

The idea of a neuronal network which operated when a person is at rest was proposed in 2001, and then confirmed in 2003 \cite{Raichle2001} \cite{Greicius2003}. Resting-state networks are recorded using resting-state functional magnetic resonance images (rs-fMRIs). rs-fMRIs are sequences of image volumes acquired over a period of a few minutes while the patient is in a task-free state. The image volumes themselves have relatively low spatial resolution when compared to structural MRIs, but their temporal resolution is significantly higher as a new volume is acquired every two to three seconds. Each volume records the blood oxygen level dependent (BOLD) signals within the brain at that point in time. 

The BOLD signals in rs-fMRI image sequences are analyzed using a process called functional connectivity analysis. Functional connectivity analysis identifies patterns and networks of brain activity. Because the patient is not performing a specific task during a rs-fMRI acquisition, these resting-state networks have the potential to reveal valuable information about a patient's neurodevelopmental status. Some functional connectivity analysis studies have lead to the discoveries of links between specific disruptions in these naturally occurring networks and neurodevelopmental diseases such as autism and attention deficit hyperactivity disorder \cite{Assaf2010} \cite{Zang2007}. With further refinements of both acquisition techniques and characterization of these functional networks, clinicians may be able to use rs-fMRI in early detection protocols to evaluate the neurodevelopmental status of infants and neonates, and in personalized care by identifying patients who may benefit from certain therapies or neuroprotective interventions.
