\chapter{rs-fMRIs and Patient Motion}
\label{ch:mri}

This chapter discusses rs-fMRIs and how they are affected by patient motion. Specific topics include the structure of rs-fMRIs, sources of motion, current methods for preventing and managing motion in rs-fMRIs, quantifying motion, and usability criteria for using images corrupted by motion.

\section{Structure of an rs-fMRI}

A rs-fMRI scan produces a four dimensional image series. The first three dimensions are length, width, and depth and encompass the physical space occupied by the patient's head. The information in these three dimensions is interpreted as a three dimensional, volumetric image. The fourth dimension is time. The temporal dimension interacts with the spatial dimensions such that the contents of that image volume change with time. This concept of a 4D image can be illustrated in two different ways. The first is an ordered list of 3D image volumes. The second is a single 3D image volume where the value of each voxel is a temporal signal. 

rs-fMRIs are discrete representations of continuous data. A new image volume of the patient's brain is acquired every two to three seconds. The image volume is composed of a three dimensional version of a pixel called a voxel (volume element). Just as the ``distance'' between each image volume encompasses a certain amount of time, each voxel encompasses a small volume of physical space. The transformations between the continuous physical and temporal dimensions and the discrete physical and temporal dimensions are the spatial and temporal resolutions. 

An rs-fMRI is considered to have relatively low spatial resolution but high temporal resolution. The physical size of a single voxel seems small at about 4 mm$^3$, but this resolution is not granular enough to capture details about activity within small structures of the brain. The activity information recorded during a rs-fMRI must be combined with the detailed anatomic information from a structural MRI to know precisely which areas of the brain are active at each point in time. A structural MRI volume takes much longer to acquire than a rs-fMRI volume, which can be obtained every two to three seconds. Unfortunately, the patient's position and neural activity can change faster than the image volume can be acquired. As a result, a temporal resolution of two to three seconds is not fast enough to actively compensate for sources of noise which confound the BOLD signal. 

\section{*Sources of Motion in rs-fMRIs}

During every medical imaging scan, the patient will naturally perform small, automatic movements due to regular bodily functions. Minuscule movements caused by cardiac activity may disrupt scans with high spatial resolution or with high sensitivity to the movement of blood molecules. Larger movements caused by respiration result in motion artifacts in images of the thoracic and abdominal cavities. 

Other motions occur on a larger and more conscious scale. It is important to note that different populations may exhibit more of certain macro-motions than others. The patient may fidget or shift his gaze when he becomes bored in the scanner. If the patient falls asleep during a scan, there may be slight movement as the body relaxes and retenses if the patient wakes. Certain MRI protocols are known to produce loud sounds: during one of these protocols, the patient may become surprised and react by jumping. Additionally, clausterphobic patients or patients who feel secure around specific people that are not allowed in the scanner room may become agitated. 


\section{Effects of Motion}

Due to their low spatial and high temporal resolutions, rs-fMRIs are highly susceptible to all types of motion outlined in the previous section. The effects of motion on rs-fMRIs can be clearly divided into two categories: the effect on patient position and the effect on the recorded BOLD signal.

\subsection{Positional Effects of Motion}

The smallest motion can alter the position of the patient enough to cause the voxels to record signals from different brain regions and tissue types. The technique used for analyzing rs-fMRIs, called functional connectivity analysis, assumes that the contents of one voxel at two different time points both contain signal from a single point in the brain. This assumption is vital in the process of inferring networks of neuronal activity. 

The effect of motion on patient position is measured in terms of the difference in the positions of the contents of temporally neighboring image volumes. The difference in position is determined using metrics calculated by performing rigid volume registration on the two volumes. 
In rigid volume registration, one volume is chosen as the reference volume and the other is considered the moving volume. The reference volume remains stationary while the moving volume is translated and rotated in three-dimensional space on top of it. The registration is considered complete when the position of the patient in the moving volume matches the position in the reference volume. % Maybe move these sentences to methods?

The moving volume can undergo linear or nonlinear transformations. Linear transformations include translation, rotation, and affine transformations along all three spatial dimensions as well as a scaling transformation. These transformations move the image volume as a whole: all voxels in the moving image remain in the same location relative to their neighbors. On the other hand, nonlinear transformations have the ability to warp the contents of the moving volume so that it better matches the contents of the reference volume. Nonlinear transformations are more complex than linear transformation. They involve additional image processing steps such as smoothing and voxel interpolation.

Even in cases when nonlinear transformations are used, the registration process begins with the translation and rotation transformations. The three translation and three rotation parameters used to achieve the best alignment are used to calculate the positional change between the image volumes. The positional change between temporally neighboring volumes is called the framewise displacement (FD).  

Several researchers have proposed slightly different methods for calculating the FD. Power et al., Jenkinson et al., and Dosenbach et al. each propose a slightly different method for calculating the FD \cite{Power2012} \cite{Jenkinson2002} \cite{Dosenbach2017}. All three FD calculations produce correlated metrics: the FD metric proposed by Power et al. produces measurements approximately twice as large as the metric proposed by Jenkinson et al., and Dosenbach et al. reported a high correlation between their FD and Power’s FD \cite{Yan2013a} \cite{Dosenbach2017}. Herein, we use Power et al.'s version of the FD metric.

\subsection{Signal Effects of Motion}

In addition to changing the recorded position of the subject, motion impacts the established spin gradients, which introduces artifacts into the image sequence.

%According to principles of quantum mechanics, elementary particles intrinsically have two types of angular momentum: orbital angular momentum and spin angular momentum. The same types of elementary molecules have the same spin angular momentum, which is denoted by a spin quantum number specific to that particle type. A spin quantum number 

During an ideal MRI scan, the patient is sitting in the scanner and all molecules are aligned with the primary magnetic field $B_0$ in a relaxed state. Then, a radiofrequency (RF) pulse is applied to the field. The purpose of the pulse is to excite the molecules in a certain volume of physical space to orient the molecules to align to a secondary field in a different plane. When the pulse ends, the molecules precess back to their orientation in $B_0$. As they do, their small magnetic fields induce electric currents on the RF coil. The currents are received by the scanner as signals in frequency space. The volume of the space intended to be excited is known, and the signal produced by the induced electric current is used in conjunction to reconstruct the image in voxel space.

However, when the patient moves, the volume of space which was thought to be excited is not actually excited: some other volume of space, which may or may not overlap with the intended volume of space, is excited instead. Because the MRI scanner has no way to know this assumption is not true, it does not know that not all of the molecules in its intended area are relaxed and correctly aligned to the $B_0$ field at the end of the RF pulse. The scanner proceeds with the next RF pulse, which excites a new set of ``relaxed molecules'', some of which are still excited from the previous pulse. As a result, the signals produced in the second RF pulse are different than they should be. For example, signals that are smaller than they should be result in dark shadows within motion affected volumes of the sequence.

The previous few paragraphs in this section describe how motion disrupts the magnetic spin gradients present in the patient during an rs-fMRI scan. The spin gradients need time to recover to the correct magnetic field orientation, and up to eight to ten seconds may pass before the recovery is complete \cite{Power2014}. While the spin gradients are reorienting, the recorded BOLD signal will vary more than usual between temporally neighboring volumes. These variations are more difficult to quantify than the positional effects of motion.

One popular metric to measure changes in the BOLD signal due to patient motion was developed by Smyser et al. in 2010. Their metric is called DVARS, which measures the temporal \textbf{d}erivative of the root mean squared \textbf{var}iance over the voxels between two volumes \cite{Smyser2010}. Power et al. explain the steps to calculate DVARS in a separate study \cite{Power2012}. The DVARS value is calculated in two steps. The first step uses backward differences to approximate the derivative of the BOLD signal change between volumes $J_i$ and $J_{i-1}$ at every point $\vec{x}$ contained in both image volumes:

\begin{equation}
\frac{\partial}{\partial t} J_i(\vec{x}) \approx J_i(\vec{x}) - J_{i-1}(\vec{x}).
\end{equation}

% FD(J_i) = | \Delta d_{ix} | + | \Delta d_{iy} | + | \Delta d_{iz} | \\ + | \Delta \alpha_i | + | \Delta \beta_i | + | \Delta \gamma_i |

The second step calculates the root mean square of the approximated derivatives for all $N$ points $\vec{x}$:

\begin{equation}
DVARS(J_i) = \sqrt{ \frac{1}{N} \sum_{\vec{x} \in J_i, J_{i-1}} \left( \frac{\partial}{\partial t} J_i(\vec{x}) \right)^2 }.
\end{equation}

DVARs measures the change in BOLD signal intensity, which is highly related to motion-induced spin gradient changes. 

\section{Motion Prevention}

Various techniques and protocols have been developed to prevent patients from moving during the image acquisition process. Not all of these techniques are suitable for all patient populations, and some techniques have been designed specifically for certain populations.

\subsection{Pre-Scan: Education}

Educational material can be used to help the patient understand what to expect during an MRI scan as well as to teach the patient different behavioral coping strategies. The education materials can be used either before or upon arrival at the imaging facility. Most of the formal literature focuses on informative, distraction, and behavioral techniques to use during pediatric MRI scans, though many of the following approaches could be adapted for use with adults.

% Pediatric patients
In a review of the available literature, Alexander found several commonly used techniques to educate pediatric patients before and comfort or distract pediatric patients during radiology procedures \cite{Alexander2012}. Tools such as educational coloring books and short videos can expose patients to the types of equipment they can expect to see using a familiar, engaging medium. Pediatric patients can learn coping strategies to employ during the scan such as breathing techniques, imagery, and positive statements. Alexander notes that allowing a pediatric patient to choose a behavioral coping strategy gives the patient a sense of control and may encourage the patient to cooperate during the MRI acquisition.

Mock scanners and MRI simulators can also help the patient feel more comfortable during the scan. Barnea-Goraly et al. showed that both a commerical MRI simulator and a low-tech mock scanner desensitized pediatric patients between four and ten years of age to the MRI scanner with the results that 92.3\% of the acquired images could be used in high-resolution anatomical studies \cite{Barnea-Goraly2014}. 

% distraction
Several groups have investigated the role of auditory and visual distraction during an MRI acquisition. Headphones with music and stories or MR compatible video goggles can distract patients from the tedium of the scan \cite{Alexander2012} \cite{Barnea-Goraly2014} \cite{Harned2001}. Khan et al. found that a relatively simple moving light show can be helpful in distracting younger patients \cite{Khan2007}. Garcia-Palacios et al. performed a case study comparing the efficacy of music and immersive virtual reality tools as distractions during a mock scan \cite{Garcia-Palacios2007}. They suggest that immersive virtual reality may help decrease patient anxiety during a scan more effectively than music alone. As virtual reality technology improves, it may join headphones and MR compatible video goggles as an available distraction method.

Another helpful source of distraction for pediatric patients could be the patient's parent or parents. Having a parent involved with the scanning process may calm the patient and encourage him to cooperate; however, parental distress can further upset an anxious patient and complicate the scanning process \cite{Alexander2012}. 

These techniques for educating the patient and helping the patient cope with the anxiety that can accompany an MRI scan all depend on the ability of the patient to understand instructions and communicate with the scan team. Due to the gap in communication abilities, these techniques are not useful for young patients such as neonates, infants, toddlers, and possibly elementary school aged children. Other patient populations, such as those with developmental delays and neurobehavioral disorders, may also have difficulty adhering to these protocols. Even in patients with developed and intact communication skills, the techniques outlined here do not actively prevent the patient from moving during the scan: they only help the patient feel more comfortable with the MRI environment.

\subsection{During Scan: Sedation}

Sedation can be used to help a patient tolerate an MRI scan. Murphy and Brunberg retrospectively analyzed seven weeks of data from the MR department and found that 14.2\% of their adult patients required some form of sedation \cite{Murphy1997}. In a study about claustrophobia and MR acquisitions, Dewey et al. report that out of 55,734 patients who underwent MRI scans, a total of 1004 patients experienced claustrophobia and 610 of these patients required intravenous sedation before their scans \cite{Dewey2007}. Even though sedation allowed the patients mentioned in this paragraph to undergo an MRI scan, the authors of both studies note that sedation can result in adverse events and advise the reader to avoid patient sedation if possible.


Sedation can be used with pediatric patients, though the risks are more significant than with adult patients. Studies have shown that sedation for pediatric imaging can lead to hypoxemia and inappropriate sedation levels during image acquisition \cite{Malviya2000}. Pediatric patients can also expect ``motor imbalance and gastrointestinal effects,'' as well as agitation and restlessness for a period of hours after waking from sedation.

A report from the American Academy of Pediatrics and the American Academy of Pediatric Dentistry outlines the minimum set of criteria needed for a pediatric patient to be sedated for a procedure \cite{Cote2016}:
\begin{itemize}
\item The patient must be a suitable candidate for sedation based on their medical history and medical needs.
\item The patient's health status must be evaluated and verified by the sedation team prior to the procedure.
\item Informed consent must be obtained prior to the procedure.
\item Instructions for what to expect and how to transport the patient home safely must be provided to the patient's responsible adult.
\item At least one responsible adult must be with the patient at the medical facility, though the report recommends that two adults are present for patients who travel to and from the facility using car seats. This practice ensures that one adult can monitor the patient after the procedure while the other adult drives.
\item The patient's food and drink intake prior to the procedure should be taken into account to minimize the risk of pulmonary aspiration.
\item The clinician administering the sedation must have immediate access to emergency facilities, personnel, and equipment, and should monitor the patient for adverse events including respiratory events, seizures, vomiting, and allergic reactions.
\item There must be a clear protocol outlined for immediately accessing these emergency services.
\item Emergency equipment and drugs appropriate for the patient's size and age must be immediately available in case the patient needs to be resuscitated.
\item The information about the procedure must be correctly documented.
\item The facility should have a dedicated recovery area, and the status of the patient should be recorded when he is discharged. The patient should not be discharged if his level of consciousness and oxygen saturation do not meet recognized guidelines.
\item The patient may be held at the facility for prolonged monitoring after the procedure.
\end{itemize}
\noindent This report clearly states that the levels of monitoring suggested above should serve as minimum levels of involvement: clinicians should increase patient monitoring as needed for complex cases. Rutman has a similar and detailed perspective on patient monitoring during and after sedation, adding that two independent medical personnel should be present during the scan and one should be present until the patient is discharged \cite{Rutman2009}. Rutman also notes that all sedation and monitoring equipment must be MR compatible, which is a simple but important safety constraint. This constraint makes sedation less advisable if the appropriate equipment is not available.

Sedation in neonatal and infant populations is not recommended. The  U.~S.~Food and Drug Administration (FDA) issued a warning in late 2016 about repeated use of sedation or general anesthesia for patients under three years of age or for pregnant women during their third trimester \cite{FDA2016}. The warning states that while a single, relatively short exposure to sedative and anesthetic drugs is unlikely to impact the patient, the effects of prolonged exposure to these drugs are still being studied. Studies of sedative and anesthetic drugs in multiple animal models have shown that these drugs can lead to loss of nerve cells in the brain when the animals undergo prolonged, repeated exposure to them during period of brain development. More data is needed to determine if this effect translates to humans.

\subsection{During Scan: Feed and Sleep Protocols}

Neither sedation nor educational and behavioral techniques are appropriate to use with neonatal patients, but rs-fMRIs in neonates and infants are invaluable  in studying early brain development and neurological diseases \cite{Smyser2015}. A set of protocols have been developed specifically for scanning neonates without sedation. These protocols are referred to as ``feed and sleep'' or ``feed and bundle'' protocols.

Windram et al. describe a protocol in which the infant is deprived of food for four hours prior to the scan \cite{Windram2011}. At the scanning facility, the patient is fed by his mother, swaddled, and placed in a vacuum-bag immobilizer for the duration of the scan. 

Rather than deprive the patient of food prior to the scan, Gale et al.'s protocol recommends timing the scan so that the patient is fed after arrival on site and less than 45 minutes before the scan \cite{Gale2013}. The patient's ears are protected from the noise of the MR scanner by a layer of dental putty followed by headphones, and held in place by a hat. The patient is the swaddled and placed in the scanner once he is asleep. Additional foam padding is used to cushion the patient's head and provides extra noise protection.

Mathur et al. describe a protocol similar to the previous two: the patient's feeding schedule is adjusted so that he feeds 30-45 minutes before the scan time, and he is swaddled, given ear protection, and placed in a vacuum-bag immobilizer \cite{Mathur2008}.

When performed correctly, these protocols are generally successful and the neonatal patient will sleep for the duration of the MRI scan. However, the patient may shift slightly while asleep or may wake up and move mid-scan.

% application of rs-fMRI to evaluate neonates neurodevelopmental status of infants and neonates, and in personalized care by identifying patients who may benefit from certain therapies or neuroprotective interventions \cite{Smyser2015}.

\section{Prospective Motion Correction}

Since motion cannot be completely eliminated from rs-fMRI scans, different approaches have developed for correcting for the effects of motion after the scan. These approaches can be divided into two groups: those which monitor the patient's motion during the scan and those which work solely on the acquired sequences.

\subsection{Optical Motion Correction}

Several groups have developed methods for actively accounting for changes in the patient's position during an MRI scan. Optical-based methods record the patient's position using a combination of markers placed on the patient and one or more MR compatible optical cameras placed the scanner bore. The changes in the patient position from one time point to the next are used to update the MR parameters in real-time. Real-time updates of the MR parameters result in decreased spatial and spin-history effects of motion in the acquired sequences.

The first report of successful prospective motion correction using optical cameras and markers was by Zaitsev et al. in 2006 \cite{Zaitsev2006}. Their dual camera system was located outside of the MRI scanner and focused on the patient inside the system. Four reflective markers were attached to a modified mouthpiece originally designed for patient immobilization. Changes in the translation and rotation of the patient were recorded and processed during the exam. The processed changes were sent in real-time to the MRI scanner which used them to update the gradient orientations, RF frequencies, and RF phases at every time point during the acquisition process.

Aksoy et al. simplify this approach by using a single in-bore optical camera and replacing the 3D markers with a small 2D chessboard grid \cite{Aksoy2008}. Properties intrinsic to the camera as well as information about the camera's placement within the MRI scanner were recorded prior as part of a calibration process. During the scan, patient movements recorded using the optical camera were used to calculate the relationship between the patient's position at the current time point in the physical space and the patient's position at the initial time point in the MR space. The transformation needed to translate between these two positions was calculated on a laptop and passed to the MRI scanner to correct for motion in real-time. The camera used to record the position of the chessboard is mounted on the head coil. If the patient moves his head significantly, the camera will only be able to record the position of part of the chessboard marker. This limitation makes it difficult for the computer vision processing to identify the independent features on the standard chessboard. 

Forman et al. modified the chessboard marker to improve its flexibility \cite{Forman2011}. To differentiate between the different blocks in the chessboard, they added a unique, machine readable symbol to each black block in the chessboard. The symbols were chosen to be unique even in the event of rotation so that the identification of each block would be robust to rotation movements. The chessboard marker was embedded with MR-detectable agar so that the position of the marker could be detected in the MRI scan as well as by the in-bore camera. At each point during the scan, the image recorded by the in-bore camera was sent to a computer independent from the MRI controller. The independent computer detected the blocks of the chessboard and identified their spatial locations using the symbols contained within them. Their positions were checked by confirming the locations of the symbols with respect to each other. The confirmed locations of the corners of the black boxes were used to estimate the position of the patient, which was then sent to the MRI controller so that the magnetic gradients and RF hardware could be updated for the time point. The authors note that the latency of the system is a significant limitation to their system, but overall they experienced an increase in the accuracy of the estimates of the patient's position.

Several companies have developed commercial products for prospective motion correction in neurological images. KintetiCor's system uses a high resolution camera and a physical marker to detect motion \cite{kineticor}. The camera's resolution allows it to detect respiratory and cardiac motion through changes in skin displacement on the patient's forehead. The physical marker consists of pair of rectangles containing several concentric circles which are connected via a bridge across the nose. Any patient movement is reflected in the movement of the markers, which is also tracked through the camera. Both the camera system and the marker are MR compatible. Another company, TracInnovations, uses a stereo camera system to track all patient motion \cite{tracinnovations}. At the start of the scan, the stereo camera obtains a point cloud of the patient's position at that time. The points in the point cloud are averaged together to create a primary marker. Small facial motions, cardiac motion, and respiratory motion, are monitored using the point cloud. Larger head motions are monitored using both the point cloud and the primary marker. These two systems both allow prospective motion correction to be turned on or off: if the prospective motion correction is off, the system will still acquire the motion parameters so that the motion can be corrected retrospectively.

% Limitation: MR safe equipment
% Limitation: measurements must be made  
% Limitation: only rigid body motion. 
The methods discussed above have a few limitations due to the optical camera setups. For precise real-time motion correction, the camera or cameras must be carefully placed so that the position of the marker on the patient can be recorded. They must have a clear line of sight, which means they will be in the same room as the MRI scanner, if not within the scanner bore. The cameras and markers must be MR compatible, and the positions of the cameras and markers in physical space relative to the visual markers on the patient must be known. These positions are vital for the calculations used to measure the motions. Even if the motion measurements are accurate, the changes in position that are recorded and used to adapt the scan parameters will only be true for rigid body motion of the body part to which the markers are attached: any distortion of soft tissue will not be accurately accounted for during the motion correction unless the camera system was specifically built for and trained to do so. 
% To incorporate
% Limitation: optical trackers must have clean line of sight to subject 
% Limitation: systems using markers attached to patient suffer from imperfect attachment of the marker to the subject, where the marker slips or moves during the scan independently from the subject, 
% Look up Oliver Speck?

\subsection{External Sensors}

Cameras are not the only type of external sensor that can be used to measure motion during a rs-fMRI scan. 

There is a class of sensors which can take advantage of electrophysics properties of an MRI scanner. These sensors include wired nuclear magnetic resonance field probes, wireless inductivity coupled markers, and off-resonance markers. %CITATIONS AND DETAILS. 
The fact that these sensors directly interact with the magnetic field of the MR scanner means that protocols using these sensors must be modified to account for them. As a result of the protocol modification, the scan time might need to be extended.

%CHECK DETAILS AND CITATION 
As mentioned earlier in this chapter, respiration is a source of patient motion. Since respiration is relatively periodic, it can be monitored and accounted for within a scan protocol via gating. Gating prevents an image from being acquired unless the patient is in the expected state. In the case of respiration, the expected state is either complete inhalation or exhalation. The state of a patient's respiration can be tracked using respiration bellows. After acquiring the MRI sequence, volumes in the sequence can be grouped depending on when they were recorded in the breathing cycle. By only using volumes recorded during the same stage of the breathing cycle, the effects of respiratory motion can be mitigated.

%Cardiac: pulse oximeter (delay relative to central pulse) 
%ECG: less reliable at high field strengths (Frauenrath et al JMRI 2012; 36: 364-72
%Acoustic cardiac triggering: 
%Frauenrath et al, Investigative Radiology 2009,
%Frauenrath et al J Cardiovascular MR 2010

%Additional sensors complicate the scan setup

%Detection using RF coils? Changes conductivity and RF power (Buikman et al MRI 1988)

Ultimately, the addition of extra sensors complicate the process and set up of a rs-fMRI scan.

\subsection{Image Signal Motion Monitoring}

Dosenbach et al. have developed a tool to evaluate motion in rs-fMRI sequences as they are acquired \cite{Dosenbach2017}. It registers each volume to the initial volume of the rs-fMRI sequence immediately after the new volume is recorded. The parameters produced by this registration are used to calculate the framewise displacement between pairs of volumes, which is then compared to a set of displacement thresholds associated with the scan quality. The number of volumes that meet each threshold is used to determine how many more volumes are needed to obtain five minutes of low-motion volumes. This method for assessing the quality of a scan in real time is useful for ensuring images are acquired with a sufficient number of low-motion volumes. It can also aid the technologists in determining whether to prematurely terminate a scan, which may be desirable if the amount of time needed to obtain enough low-motion volumes is greater than the amount of time remaining for the patient in the scanner. 


\subsection{General Limitations of Prospective Motion Correction}

All types of prospective motion correction introduce a delay into the scanning process. The delay is due to the additional processing of some metrics to determine the patient's position, the transmission of these metrics to the MR scanner, and the adjustments the scanner makes to its next set of measurements. These alterations to the image acquisition during prospective motion correction actively change the image as it is acquired.
Maclaren et al. note that while prospective motion correction reduces imhomogeneities in the $B_0$ field, the $B_0$ field will still change when the patient moves and may change while the motion correction is occuring \cite{Maclaren2013}. % SO WHAT?

In order to view a scan not impacted by prospective motion correction, the patient often must undergo a second scan. It may be wise to build the second image acquisition into the same scan period as the prospectively motion corrected scan: unsuccessful prospective motion correction has the potential to drastically corrupt the acquired scan \cite{Zaitsev2017}.

Finally, though prospective motion correction has great power for managing motion during a scan, it cannot be used to recover motion-corrupted data in existing data sets.


\section{Retrospective Motion Correction}

Many groups have put significant effort into developing techniques for motion correction after the scan is acquired. Here, we discuss several commonly used techniques: volume registration, denoising, and filtering. % and give examples of pipelines which utilize these tools.

\subsection{Volume Registration}

The rs-fMR image is stored in computer memory as a set of 3D matrices. The values in corresponding cells of each matrix are considered to be aligned in this digital space (voxel space). The voxel space is defined by the imaging protocol and relates to the physical space through the spatial resolution of the image. Even though the spatial and voxel spaces for the image align, the contents of the image volumes may be misaligned due to patient movement. Because we cannot assume that an image is completely motion-free, we cannot directly compare the contents of each image volume in the rs-fMRI sequence. However, we can use image registration to align the contents of the image volumes to reduce the impact of motion on patient position.

Image registration is the process of morphing the contents of one image so that they overlap optimally with another image. The morphing operations include translation, rotation, scaling, skewing, and nonlinear adjustments. The linear and affine operations in this list should be used to perform rigid body registrations for organs such as the brain. Nonlinear operations can be used to fine-tune the alignment of more pliable organs such as the liver. All morphing operations are applied to one image repeatedly until it's contents optimally match those of the static reference image as determined by a chosen similarity metric. 

One of the earliest examples of image registration was described by Friston et al. in 1995 \cite{Friston1995}. They performed image registration on positron emission tomography (PET) scans and MRI scans of a human brain. During the registration process, one scan was designated as the ``reference'' image, which remained stationary, and the other scan was designated as the ``object'' image, which was transformed to match the reference image. Constraining the alignment process to transforming a single image into the coordinates of the other image rather than transforming both images into an independent coordinate frame simplifies the registration process.

\begin{figure}
\centering
\includegraphics[width=.7\textwidth]{2/traditional-registration.png}
\caption{The traditional approach to volume registration in an rs-fMRI sequence consists of registering all volumes in the sequence to a single reference volume.}
\label{fig:ch4:traditional-reg}
\end{figure}

When performing image registration on a sequence of image volumes, one volume must be chosen as the reference volume for the entire sequence. All other volumes in the sequence are registered to this volume. An example of this process can be seen in Figure \ref{fig:ch4:traditional-reg}. In subsequent work, Friston et al. used the first volume in the rs-fMRI sequence as the universal reference image \cite{Friston1996}. Common choices for the reference volume include the volume with the least FD to all other volumes in the sequence, a volume produced by averaging all volumes in the sequence, or the first volume in the sequence \cite{Friston1996} \cite{Liao2005}. In our implementation, we chose to use the first volume in the sequence as the reference volume.

One drawback to this traditional approach to volume registration is that it only minimizes the differences between all the image volumes in the sequence and the reference volume. The key word here is minimizes: minimizing differences between image volumes does not mean that there are no differences between the image volumes. Image registration is an optimization problem, and its goal is to find the overlap between a pair of volumes with as few differences as possible either within a defined time period or until the optimization cost does not change above a certain tolerance for a certain amount of time. These practical constraints on optimization problems mean that there may still be differences between other pairs of image volumes in the sequence that do not include the reference volume. 

Variations on Friston et al.'s framework have been developed over the last two decades. Liao et al. suggested that a rs-fMRI sequence could be viewed as a hidden Markov model, and reflected this idea in their suggested registration framework \cite{Liao2016}. They still use the first volume in the image sequence as the reference volume. Their framework uses the transformation of the previous volume to the reference volume to initialize the transformation for the current volume and the reference volume. 

It has been demonstrated that image registration across the entire image sequence reduces the effects of motion on the image sequence, though they do note that motion also effects the image due to changes in the spin history of the image. These effects are not correctable by global volume registration alone and will be discussed later in this chapter.

\subsection{Denoising}

Denoising techniques can be applied to a rs-fMRI after global volume registration is completed. They consist of regressions of various confound variables. 

Regression of the global signal (global signal regression, GSR) corrects for variance between temporal signals within a voxel and for the mean BOLD signal across all voxels \cite{Power2014} \cite{Satterthwaite2013} \cite{Yan2013} \cite{Yan2013a}. GSR has been shown to reduce spuriously increased long-distance correlations in functional connectivity studies, but may inadvertently weaken shorter-distance connections \cite{Jo2013} \cite{Power2014}  \cite{Satterthwaite2012}. 

Other regression parameters have been investigated. Commonly, the six rigid realignment parameters and their first order derivatives are suggested as regression parameters \cite{Power2012} \cite{Satterthwaite2012} \cite{VanDijk2012}. More recently, researchers have also incorporated the rigid realignment parameters from surrounding timpoints \cite{Power2014} \cite{Satterthwaite2013} \cite{Yan2013a}.

, signals from white matter or cerebral spinal fluid \cite{Power2014} \cite{Satterthwaite2013} \cite{Yan2013a} \cite{Jo2010}, and components identified using principal or independent component analysis \cite{Pruim2015} \cite{Salimi-Khorshidi2014} \cite{Behzadi2007}. Regression of each of these sets of parameters has been shown to reduce the effects of motion in the sequence but not remove them entirely \cite{Power2015} \cite{Parkes2017}. 



Patriat et al. performed a robust comparison of different regression parameters on their MotSim motion data set \cite{Patriat2017}. They included rigid realignment parameters, but also used parameters obtained by performing principle component analysis (PCA) on the image sequences. PCA generates a set of linear, uncorrelated components that reflect the main features of a patient's motion. The list of parameter combinations included 
\begin{itemize}
\item 12mot: The six rigid realignment parameters and their first derivatives,
\item 12for: The first 12 principal components of the whole brain before realignment,
\item 12back: The first 12 principal components of the whole brain after realignment,
\item 12both: The first 12 principal components of the whole brain both before and after realignment,
\item 24mot: the six rigid realignment parameters of the current volume, the six rigid realignment parameters of the previous volume, and the square of these rigid realignment parameters,
\item 24both: the first 24 principle components of the whole brain before and after realignment.
\end{itemize}

\noindent They found that the features extracted from the image sequence using PCA explained more variance in the image sequence (measured using $R^2$) than the rigid realignment parameters. They showed that increasing the number of regressors increased the amount of variance explained, but with diminishing returns. While their work is promising, their experiment was performed on a simulated data set using healthy subject data and required an accurate estimate of the subject's head motion.

\subsection{Filtering}

Filtering, which is also referred to as censoring, involves the identification and removal or interpolation of volumes containing high quantities of motion. Two popular techniques are scrubbing and spike regression. Power et al.’s scrubbing technique removes volumes with more than 0.2 mm of FD \cite{Power2012}. Spike regression identifies volumes with large FD and replaces them with interpolated volumes \cite{Satterthwaite2013}. Unfortunately, these filtering techniques ultimately result in the loss of data as volumes are removed from the sequence. A third technique called despiking detects signal spikes at the voxel level and interpolates over the spikes \cite{Jo2013} \cite{Patel2014}. Despiking does not remove volumes, but could accidentally remove valuable signals. 

\subsection{Spin History Distortion Correction}

A number of post-acquisition methods have been developed specifically to correct for distortions due to the impact of motion on the magnetic field. The usability of these dynamic distortion correction methods has been studied in a few specific cases, but their generalizability has yet to be confirmed in a broader range of fMRI studies \cite{Zaitsev2017}.

\section{Image Usability}


%The FD metric only measures the positional effects of motion, not the variations in signal in individual voxels caused by motion. Changes in signal between volumes can be measured using the temporal derivative of the variance in the BOLD signal intensity (DVARS) between neighboring volumes \cite{Power2012} \cite{Smyser2015}. 

Even though the effects of motion on the patient position and the recorded signal can be measured, we still need gold standard criteria to determine whether an image containing motion can be used. Patients move slightly due to breathing and cardiac function, and the BOLD signal naturally fluctuates over time. Some motion is expected; however, we need to know how much motion can be present in the image before it is considered to be corrupted by it. Power et al. established thresholds for FD and DVARS to determine the usability of a pair of images:
\begin{itemize}
\item FD less than or equal to 0.2 mm from previous volume, and
\item DVARS less than or equal to 25 units on a normalized scale of [0, 1000] signal units \cite{Power2014}
\end{itemize}

Image volumes that meet these criteria are considered to be low-motion.

The time span of low-motion data is highly debated. van Dijk et al. established that approximately five minutes of low-motion data is sufficient for use in functional connectivity analysis \cite{VanDijk2012}. However, a recent study by Laumann et al. suggests that at least 10 minutes of low-motion data is essential for obtaining high-quality results \cite{Laumann2015}. From a practical standpoint, it is difficult to obtain even five minutes of low motion data from certain patient populations, so radiology technicians and neuroimaging study designers are often content with the five minute time standard. 

\section{Summary}

Resting-state fMRIs are four dimensional images which record BOLD signal in active areas of the brain. The BOLD signal can be used to evaluate the functional connnectivity of different underlying networks in a patient's brain. Since rs-fMRIs are highly sensitive to motion, clinicians and psychologists have devised techniques to inform patients about what they can expect during an MRI scan as well as different coping mechanisms to help them remain calm during the scan. These techniques do not prevent the patient from moving, but approaches that do are not always appropriate to use during a rs-fMRI scan. Techniques and algorithms to prospectively and retrospectively remove motion from rs-fMRIs have also been developed, though they are not always successful in removing the effects of motion. Ultimately, the amount of motion present in the rs-fMRI sequence dictates whether or not the sequence can be used in clinical or research applications.

In the next chapter, we will discuss rs-fMRIs in the context of our chosen population of CHD patients.