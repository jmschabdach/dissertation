\chapter{DATA}
\label{ch:data}

The data used to test the hypothesis and aims introduced in the previous chapter are drawn from several subject populations. Because motion causes problems in MR images across all stages of life, we used images from cohorts of healthy and CHD fetal, neonatal, preadolescent, and adult subjects gathered in as part of ongoing studies. Data from these studies was obtained through studies approved by the IRB at the Children's Hospital of Pittsburgh of UPMC and the University of Pittsburgh. The data is stored and accessed in compliance with all HIPPA policies.

We use data from a simulated phantom as well as healthy adult human phantoms. The healthy adult human phantom data collection was also approved by the IRB at the Children's Hospital of Pittsburgh of UPMC and the University of Pittsburgh and is stored and accessed in compliance with all HIPPA policies.

The final data set we use is from the Alzheimer's Disease Neuroimaging Initiative (ADNI) database (adni.loni.usc.edu). The ADNI was launched in 2003 as a public-private partnership, led by Principal Investigator Michael W. Weiner, MD. The primary goal of ADNI has been to test whether serial magnetic resonance imaging (MRI), positron emission tomography (PET), other biological markers, and clinical and neuropsychological assessment can be combined to measure the progression of mild cognitive impairment (MCI) and early Alzheimer's disease (AD). For up-to-date information, see www.adni-info.org.

\section{Neonatal Subject Population and Images}

Neonatal subjects are recruited as part of a prospective observational study. The subjects were scanned using a 3T Skyra (Siemans AG, Erlangen, Germany). They were unsedated during the scans and a ``feed and bundle'' protocol was used to prevent motion during the scans \cite{Windram2011}. The newborns were positioned in the coil to minimize head tilting. Newborns were fitted with earplugs (Quiet Earplugs; Sperian Hearing Protection, San Diego, CA) and neonatal ear muffs (MiniMuffs; Natus, San Carlos, CA). An MR-compatible vital signs monitoring system (Veris, MEDRAD, Inc. Indianola, PA) was used to monitor neonatal vital signs. All scans were performed using a multi-channel head coil. The parameters for the resting-state BOLD MR scans were FOV=240 mm and TE/TR=32/2020 ms with interplane resolution of 4x4 mm, slice thickness of 4 mm, and 4 mm space between slices. The acquired images contained 150 volumes where each volume consisted of 64x64x32 voxels$^3$.

\section{Preadolescent Subject Population and Images}

As part of a multicenter study of CHD in preadolescents, we collected rs-fMRIs from nine sites throughout the United States. These images were of patients in the age range of XX to XX years who either had CHD or were healthy with no neurocognitive impairments. In addition to the MRI scans, subjects who participated in this study were asked to participate in other testing (GET DETAILS FROM NANCY).

\begin{itemize}
\item How were the images gathered?
\item How were the patients recruited?
\item What are the imaging protocol details?
\item What other information was collected?
\end{itemize}


\section{Adult Subject Population and Images}

As the prognosis for patients with CHD improves, their life expectancy also increases. The aging CHD population presents new questions about the connection between CHD and neurocognitive challenges associated with aging. As patients age, there is an expectation that their images will contain less motion for a time. If a patient begins to show signs of cognitive impairment due to aging, it can be expected that their images will begin to contain more motion as their neurocognitive state deteriorates. 

We include a cohort of adult subjects over a wide range of ages in our study. The purpose of using images from this cohort is to demonstrate the generalizability of the DAG-based framework to adult patients as well as its use in different clinical populations. This cohort is being studied as part of an ongoing, prospective study of CHD and neurodevelopment. The data collected for these subject includes rs-fMRIs, behavioral, and clinical data from XX healthy and XX CHD adult subjects. 

\begin{itemize}
\item How were the images gathered?
\item How were the patients recruited?
\item What are the imaging protocol details?
\item What other information was collected?
\end{itemize}

%The second adult cohort comes from the Alzheimer's Disease Neuroimaging Initiative (ADNI) dataset. The ADNI study has been working since 2004 to further Alzheimer's research by gathering, analyzing, and sharing clinical, imaging, genetic, and biochemical biomarkers from the elderly population. The group gathers data from 63 sites in the United States and Canada. During the second phase of the study, sites who have a Philips MRI system gathered resting-state fMRIs from their subjects. This data is freely available to academic researchers through the LONI Image and Data Archive.

\section{Fetal Subject Population and Images}

Fetal subjects have different constraints on their physical environment than neonates, preadolescents, and adults. As a result, they exhibit unique patterns of motion. The previous subject cohorts discussed in this chapter have the following commonalities: the subject experiences the full effects of gravity, the subject is lying on his back in an MRI scanner, and the subject's head motion is limited by the head coil within the MRI. Any motion in these images is a direct result of the subject himself moving, whether passively (cardiac motion and breathing) or actively (fidgeting or looking around).

A fetal subject is scanned in vivo. He is suspended in amniotic fluid within his mother. The amniotic fluid has buoyancy that reduces the effects of gravity and allows a fetal subject significant freedom of movement. The fetus can rotate, shift, and flip in ways that can only be accomplished when floating in a body of water. The properties of the uterus constrain the physical space in which a motion could occur, but not as much as the head coil and gravity do to the other patient cohorts. A fetus is not guaranteed to be in any specific position at the start of the scan: the scan begins when the mother is ready, not when the fetus achieves a certain pose. 

The fetal subjects underwent fetal echocardiography scans in a cardiac clinic to determine whether they were healthy or had a form of CHD. They were then scanned on an MRI scanner. Images of the fetal brain and the placenta were acquired for each subject. 

We are interested in both the fetal brain and placental images for our work because of the relationship between placenta and brain development. However, these organs have very different physical properties. The fetal brain is a rigid structure floating and moving within the amneotic fluid. It undergoes translation and rotation as a single unit due to passive and active maternal and fetal motions. The placenta, on the other hand, is anchored in place on the uterine wall. It may undergo small translations or rotations due to maternal motion, but it will respond differently to fetal motion. Fetal motions cause nonlinear deformations of the pliable placenta that can only be adequately accounted for using nonlinear registration algorithms. Nonlinear registrations have the potential to deform brain images into physically impossible shapes, so the fetal brain and placenta were manually segmented in their respective images so that each organ could undergo independent motion correction. 

The segmenters were one of a group of four researchers. While one researcher trained the other three group members, the interrater agreement between them is still being determined.

\begin{itemize}
\item Fetal patients scanned between XX and XX weeks gestational age. 
\item Imaging protocol details?
\item What other information is collected about fetus and/or mom?
\end{itemize}


\section{Simulated Phantom Images}

Every MRI scanner is different, so a stand-in model for an organ or tissue type is often used to calibrate an MRI scanner. The model is designed to have specific physical properties which mimic the physical properties of the organ or tissue. These properties can be accurately measured during the design process of this model so that the radiologist or researcher looking at images of the model can know the ground truth of the model. Because these models mimic true organs and tissues, they are called phantoms. 

We will generate a simulated phantom image using the rs-fMRI of a healthy adult male. A single volume will be selected from the rs-fMRI sequence. This volume will be duplicated to create a generated image with 150 instances of the same volume. This sequence will be our base phantom sequence. 

A copy of the base phantom sequence will be made and a subvolume in the same location of every volume will be selected. In the subvolume of each frame, a small amount of noise generated using a normal Gaussian distribution will be added to simulate changes in blood oxygen level-dependent signal over time. The noise will be generated from a normal Gaussian distribution will be added to each frame. This image sequence will be referred to as our BOLD phantom sequence.

\section{Human Phantom Images}

One of the major challenges in the medical imaging field is collecting a large enough set of data with a high enough quality to generate statistically significant results. As part of a multisite study, a set of FOUR healthy adult male subjects were scanned at NINE different sites within a period of TEN WEEKS.

These subjects are considered human phantoms because of their health adult status, but they still may contain some motion. Their images are included in this study so that we can determine the consistency and effects of the motion correction pipeline on images of the same subject across time.

\begin{itemize}
\item What machines were used?
\item What were the scanning protocols?
\item Are there any preliminary analyses/results on this data?
\end{itemize}

