\chapter{METHODS}
\label{ch:methods}

\section{Motion Correction}

In the previous chapter, we discuss several techniques used to retrospectively correct motion. Motion correction pipelines may use denoising and filtering, but all pipelines begin with volume registration. In this section, we discuss a different approach to volume registration, how it compares to traditional volume registration, and how volume registration fits into a motion correction pipeline. 

%In this section, we discuss the two registration frameworks we apply to our rs-fMRIs: the traditional global volume registration framework and the DAG-based global volume registration framework. The registration frameworks will later be evaluated in comparison to each other, but will also be evaluated in the context of a complete motion correction pipeline. The motion correction pipeline of choice, ICA, will also be discussed in this section.

\subsection{Directed Acyclic Graph Based Volume Registration}

As discussed previously, the major drawback to Friston et al.'s approach to volume registration is that it only minimized the positional differences between the reference volume and the rest of the sequence. This drawback demonstrates an inability for the traditional approach to account for relationships in the patient's position throughout the scan. Intuitively, we know that the patient's position at any volume in the scan is more similar to his position in the immediately previous or subsequent volume than to another randomly chosen volume in the image.

In our proposed framework, we wish to account for these spatiotemporal relationships between temporally neighboring volumes in the sequence. To accomplish this goal, we start by viewing the rs-fMRI sequence as a directed acyclic graph (DAG). A DAG consists of a set of nodes and edges. Each edge has a direction associated with it and connects a pair of nodes. Since a DAG contains no cycles, there is no possible path back to a node once it has been traversed. 

\begin{figure}
\centering
\includegraphics[width=.7\textwidth]{4/dag-chain.png}
\caption{A rs-fMRI can be viewed as a directed acyclic graph where each volume is a node and the edges connect from each volume $i$ to the following volume $i+1$.}
\label{ch4:fig:dag-chain}
\end{figure}

In the case of an rs-fMRI, each volume can be considered a node. The relationship between each pair of temporally neighboring volumes is represented as a directed edge connecting the node for the first volume to the node for the next volume. The acyclic nature of the DAG means that once a patient was in a specific position, he will never return to that exact same position with the exact same neurons firing. The position of the subject and his brain activity as measured by the BOLD signal may be similar in subsequent image volumes, but it will never be precisely the same. The perspectives of an rs-fMRI sequence as a set of images and of the sequence as a DAG can be seen in Figure \ref{ch4:fig:dag-chain}.

The cost of transitioning from one node to the next in our DAG has a parallel representation to the combination of the positional transformation needed to align volume $i$ to volume $i+1$ and the signal change between the volumes. This representation can be written as 

\begin{equation}
J_{i+1} = \phi_{i,i+1} J_i + \delta s_{i,i+1} + \epsilon
\end{equation}

\noindent{where $J_i$ and $J_{i+1}$ are volumes $i$ and $i+1$, $\phi_{i,i+1}$ is a matrix of transformation parameters that must be applied to $J_i$ to achieve the patient’s position in $J_{i+1}$, $\delta s_{i,i+1}$ is the natural change in BOLD signal, and $\epsilon$ is the change in BOLD signal due to motion. Currently, there is no way to estimate the natural change in BOLD signal and the change in BOLD signal due to motion without incorporating additional information about the MRI scanner and the patient that is not included in a rs-fMRI. We simplify our representation of the relationship between two volumes to}

\begin{equation}
J_{i+1} = \phi_{i,i+1} J_i + \epsilon^*
\end{equation}

\noindent{where $\epsilon^*$ is the change in the BOLD signal that cannot be accounted for after aligning the patient’s position in the two volumes. Here, we use the notation $\epsilon^*$ to represent the generic error change in BOLD signal across any pair of volumes.}

After aligning two volumes $i$ and $i+1$, we will then align volumes $i+1$ and $i+2$:

\begin{equation}
\begin{split}
J_{i+2} &= \phi_{i+1,i+2} J_{i+1} + \epsilon^* \\
&= \phi_{i+1,i+2} (\phi_{i,i+1} J_i + \epsilon^*) +\epsilon^*\\
&= \phi_{i+1,i+2} \phi_{i,i+1} J_i + \epsilon^{*'}\\
\end{split}
\end{equation}

Traditional volume registration assumes that 

\begin{equation}
\phi_{i,i+2} = \phi_{i+1,i+2} \phi_{i,i+1}
\end{equation}

\begin{figure}
\centering
\includegraphics[width=.7\textwidth]{4/dag-registration.png}
\caption{The traditional approach to volume registration in an rs-fMRI sequence consists of registering all volumes in the sequence to a single reference volume.}
\label{ch4:fig:dag-reg}
\end{figure}

\noindent{and calculates $\phi_{i,i+2}$ directly. We argue that this assumption is not true in all cases. Rather than directly calculate $\phi_{0,i}$ and use it to align volume $i$ to the reference volume as the traditional method does, we calculate each component $\phi$ that is a factor of $\phi_{0,i}$. Each component $\phi_{i,i+1}$ is combined with the preceding $\phi_{0,i}$s to recursively align volume $i+1$ to the reference volume without making the large and often inaccurate transformations required by directly calculating $\phi_{0,i+1}$.} This process is outlined in Figure \ref{ch4:fig:dag-reg}.

\subsection{Motion Correction Pipeline}

After performing volume registration to ensure the patient is in the same physical space throughout the image sequence, the image sequence may still contain artifacts due to motion. Many pipelines exist for correcting motion in registered rs-fMRIs. 

\section{Implementation: Tools and Libraries}

The registration frameworks described in this section were implemented in Python using the nipype (Neuroimaging in Python Pipelines and Interfaces) library \cite{Gorgolewski2011}. Affine volume registration was performed using ANTs (Advanced Normalization Tools) \cite{Avants2014}. The metric used to estimate the dissimilarity between the pairs of volumes being registered was cross-correlation with a local window size of 5 voxels. 

To calculate metrics, we used several existing tools. FLIRT (FMRIB’s Linear Image Registration Tool) was used to calculate the correlation ratio between each possible pair of volumes in the sequences \cite{Jenkinson2001} \cite{Jenkinson2002}. We then used the average and standard deviation of the correlation ratio distribution of each image to compare the images. We calculated the FD and DVARS metrics defined by Power et al. using the FSLMotionOutliers tool \cite{Power2012}. These metrics were calculated for each image and were used for evaluation of the efficacy of the registration frameworks.